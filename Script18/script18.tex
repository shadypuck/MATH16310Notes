\documentclass[../main.tex]{subfiles}

\pagestyle{main}
\renewcommand{\chaptermark}[1]{\markboth{\chaptername\ \thechapter}{}}
\setcounter{chapter}{17}

\begin{document}




\chapter[The Euclidean Space \texorpdfstring{$\protect\fakeboldb{\R}^{\bm{n}}$}{TEXT}]{The Euclidean Space \texorpdfstring{$\protect\fakeboldc{\R}^{\bm{n}}$}{TEXT}}\label{sct:18}
\begin{definition}\label{dfn:18.1}\marginnote{7/7:}
    The \textbf{Euclidean $\bm{n}$-space} $\R^n$ is the $n$-fold Cartesian product of $\R$. Symbolically,
    \begin{equation*}
        \R^n = \{(x_1,\dots,x_n)\mid x_1,\dots,x_n\in\R\}
    \end{equation*}
    is the set of $n$-tuples of real numbers. We often write
    \begin{equation*}
        \x = (x_1,\dots,x_n)
    \end{equation*}
    to denote an element, which is also referred to as a \textbf{vector}, in $\R^n$ and
    \begin{equation*}
        \bm{0} = (0,\dots,0)
    \end{equation*}
\end{definition}

\begin{definition}\label{dfn:18.2}
    Let $\x=(x_1,\dots,x_n),\y=(y_1,\dots,y_n)\in\R^n$ and $\lambda\in\R$. We define the following operations.
    \begin{enumerate}[label={(\alph*)}]
        \item (Addition) $\x+\y=(x_1+y_1,\dots,x_n+y_n)$.
        \item (Scalar Multiplication) $\lambda\x=(\lambda x_1,\dots,\lambda x_n)$.
    \end{enumerate}
\end{definition}

\begin{exercise}\label{exr:18.3}
    Prove that the addition on $\R^n$ satisfies FA1-FA4 (see Definition \ref{dfn:7.8}). Moreover, prove that
    \begin{enumerate}[label={VS\arabic*.}]
        \item (Associativity of Scalar Multiplication) If $\lambda,\mu\in\R$ and $\x\in\R^n$, then $(\lambda\mu)\x=\lambda(\mu\x)$.
        \item (Distributivity of Scalars) If $\lambda,\mu\in\R$ and $\x\in\R^n$, then $(\lambda+\mu)\x=\lambda\x+\mu\x$.
        \item (Distributivity of Vectors) If $\lambda\in\R$ and $\x,\y\in\R^n$, then $\lambda(\x+\y)=\lambda\x+\lambda\y$.
        \item (Scalar Multiplicative Identity) If $\x\in\R^n$, then $1\x=\x$.
    \end{enumerate}
    These eight properties together are called the \textbf{vector space axioms}.
    \begin{proof}
        To prove that $\R^n$ obeys FA1 from Definition \ref{dfn:7.8}, it will suffice to show that for all $\x,\y\in\R^n$, $\x+\y=\y+\x$. Let $\x,\y$ be arbitrary elements of $\R^n$. Then by Definition \ref{dfn:18.2},
        \begin{align*}
            \x+\y &= (x_1+y_1,\dots,x_n+y_n)\\
            &= (y_1+x_1,\dots,y_n+x_n)\\
            &= \y+\x
        \end{align*}
        as desired.\par
        To prove that $\R^n$ obeys FA2 from Definition \ref{dfn:7.8}, it will suffice to show that for all $\x,\y,\z\in\R^n$, $(\x+\y)+\z=\x+(\y+\z)$. Let $\x,\y,\z$ be arbitrary elements of $\R^n$. Then by Definition \ref{dfn:18.2},
        \begin{align*}
            (\x+\y)+\z &= (x_1+y_1,\dots,x_n+y_n)+\z\\
            &= ((x_1+y_1)+z_1,\dots,(x_n+y_n)+z_n)\\
            &= (x_1+(y_1+z_1),\dots,x_n+(y_n+z_n))\\
            &= \x+(y_1+z_1,\dots,y_n+z_n)\\
            &= \x+(\y+\z)
        \end{align*}
        as desired.\par
        To prove that $\R^n$ obeys FA3 from Definition \ref{dfn:7.8}, it will suffice to find an element $0\in\R^n$ such that $\x+0=0+\x=\x$ for all $\x\in\R^n$. Choose $\bm{0}$ to be our $0$. Let $\x$ be an arbitrary element of $\R^n$. Then by Definition \ref{dfn:18.2},
        \begin{align*}
            \x+\bm{0} &= (x_1+0,\dots,x_n+0)\\
            &= (x_1,\dots,x_n)\\
            &= \x\\
            &= (0+x_1,\dots,0+x_n)\\
            &= \bm{0}+\x
        \end{align*}
        as desired.\par
        To prove that $\R^n$ obeys FA4 from Definition \ref{dfn:7.8}, it will suffice to show that for all $\x\in\R^n$, there exists $\y\in\R^n$ such that $\x+\y=\y+\x=0$. Let $\x$ be an arbitrary element of $\R^n$. Choose $\y=(-x_1,\dots,-x_n)$. Then by Definition \ref{dfn:18.2},
        \begin{align*}
            \x+\y &= (x_1+(-x_1),\dots,x_n+(-x_n))\\
            &= (0,\dots,0)\\
            &= \bm{0}\\
            &= ((-x_1)+x_1,\dots,(-x_n)+x_n)\\
            &= \y+\x
        \end{align*}
        as desired.\par
        To prove that $\R^n$ obeys VS1, it will suffice to show that for all $\lambda,\mu\in\R$ and $\x\in\R^n$, we have $(\lambda\mu)\x=\lambda(\mu\x)$. Let $\lambda,\mu$ be arbitrary elements of $\R$, and let $\x$ be an arbitrary element of $\R^n$. Then by Definition \ref{dfn:18.2},
        \begin{align*}
            (\lambda\mu)\x &= ((\lambda\mu)x_1,\dots,(\lambda\mu)x_n)\\
            &= (\lambda(\mu x_1),\dots,\lambda(\mu x_n))\\
            &= \lambda(\mu\x)
        \end{align*}
        as desired.\par
        To prove that $\R^n$ obeys VS2, it will suffice to show that for all $\lambda,\mu\in\R$ and $\x\in\R^n$, we have $(\lambda+\mu)\x=\lambda\x+\mu\x$. Let $\lambda,\mu$ be arbitrary elements of $\R$, and let $\x$ be an arbitrary element of $\R^n$. Then by Definition \ref{dfn:18.2},
        \begin{align*}
            (\lambda+\mu)\x &= ((\lambda+\mu)x_1,\dots,(\lambda+\mu)x_n)\\
            &= (\lambda x_1+\mu x_1,\dots,\lambda x_n+\mu x_n)\\
            &= (\lambda x_1,\dots,\lambda x_n)+(\mu x_1,\dots,\mu x_n)\\
            &= \lambda\x+\mu\x
        \end{align*}
        as desired.\par
        To prove that $\R^n$ obeys VS3, it will suffice to show that for all $\lambda\in\R$ and $\x,\y\in\R^n$, we have $\lambda(\x+\y)=\lambda\x+\lambda\y$. Let $\lambda$ be an arbitrary element of $\R$, and let $\x,\y$ be arbitrary elements of $\R^n$. Then by Definition \ref{dfn:18.2},
        \begin{align*}
            \lambda(\x+\y) &= \lambda(x_1+y_1,\dots,x_n+y_n)\\
            &= (\lambda(x_1+y_1),\dots,\lambda(x_n+y_n))\\
            &= (\lambda x_1+\lambda y_1,\dots,\lambda x_n+\lambda y_n)\\
            &= (\lambda x_1,\dots,\lambda x_n)+(\lambda y_1,\dots,\lambda y_2)\\
            &= \lambda\x+\lambda\y
        \end{align*}
        as desired.\par
        To prove that $\R^n$ obeys VS4, it will suffice to show that for all $\x\in\R^n$, we have $1\x=\x$. Let $\x$ be an arbitrary element of $\R^n$. Then by Definition \ref{dfn:18.2},
        \begin{align*}
            1\x &= (1x_1,\dots,1x_n)\\
            &= (x_1,\dots,x_n)\\
            &= \x
        \end{align*}
        as desired.
    \end{proof}
\end{exercise}

\begin{remark}\label{rmk:18.4}
    Since $\R^n$ with the two operations defined as above satisfies these eight axioms, we call $\R^n$ a \textbf{vector space}.
\end{remark}

\begin{exercise}\label{exr:18.5}
    Prove that if $\x\in\R^n$, then $0\x=\bm{0}$.
    \begin{proof}
        By Definition \ref{dfn:18.2}, we have that
        \begin{align*}
            0\x &= (0x_1,\dots,0x_n)\\
            &= (0,\dots,0)\\
            &= \bm{0}
        \end{align*}
        as desired.
    \end{proof}
\end{exercise}




\end{document}