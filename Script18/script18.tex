\documentclass[../main.tex]{subfiles}

\pagestyle{main}
\renewcommand{\chaptermark}[1]{\markboth{\chaptername\ \thechapter}{}}
\setcounter{chapter}{17}

\begin{document}




\chapter[The Euclidean Space \texorpdfstring{$\protect\fakeboldb{\R}^{\bm{n}}$}{TEXT}]{The Euclidean Space \texorpdfstring{$\protect\fakeboldc{\R}^{\bm{n}}$}{TEXT}}\label{sct:18}
\marginnote{7/7:}For the next three sheets, we will be studying multivariable calculus, that is "calculus on $\R^n$." First, we need to understand the space $\R^n$.

\begin{definition}\label{dfn:18.1}
    The \textbf{Euclidean $\bm{n}$-space} $\R^n$ is the $n$-fold Cartesian product of $\R$. Symbolically,
    \begin{equation*}
        \R^n = \{(x_1,\dots,x_n)\mid x_1,\dots,x_n\in\R\}
    \end{equation*}
    is the set of $n$-tuples of real numbers. We often write
    \begin{equation*}
        \x = (x_1,\dots,x_n)
    \end{equation*}
    to denote an element, which is also referred to as a \textbf{vector}, in $\R^n$ and
    \begin{equation*}
        \bm{0} = (0,\dots,0)
    \end{equation*}
\end{definition}

\begin{definition}\label{dfn:18.2}
    Let $\x=(x_1,\dots,x_n),\y=(y_1,\dots,y_n)\in\R^n$ and $\lambda\in\R$. We define the following operations.
    \begin{enumerate}[label={(\alph*)}]
        \item (Addition) $\x+\y=(x_1+y_1,\dots,x_n+y_n)$.
        \item (Scalar Multiplication) $\lambda\x=(\lambda x_1,\dots,\lambda x_n)$.
    \end{enumerate}
\end{definition}

\begin{exercise}\label{exr:18.3}
    Prove that the addition on $\R^n$ satisfies FA1-FA4 (see Definition \ref{dfn:7.8}). Moreover, prove that
    \begin{enumerate}[label={VS\arabic*.}]
        \item (Associativity of Scalar Multiplication) If $\lambda,\mu\in\R$ and $\x\in\R^n$, then $(\lambda\mu)\x=\lambda(\mu\x)$.
        \item (Distributivity of Scalars) If $\lambda,\mu\in\R$ and $\x\in\R^n$, then $(\lambda+\mu)\x=\lambda\x+\mu\x$.
        \item (Distributivity of Vectors) If $\lambda\in\R$ and $\x,\y\in\R^n$, then $\lambda(\x+\y)=\lambda\x+\lambda\y$.
        \item (Scalar Multiplicative Identity) If $\x\in\R^n$, then $1\x=\x$.
    \end{enumerate}
    These eight properties together are called the \textbf{vector space axioms}.
    \begin{proof}
        To prove that $\R^n$ obeys FA1 from Definition \ref{dfn:7.8}, it will suffice to show that for all $\x,\y\in\R^n$, $\x+\y=\y+\x$. Let $\x,\y$ be arbitrary elements of $\R^n$. Then by Definition \ref{dfn:18.2},
        \begin{align*}
            \x+\y &= (x_1+y_1,\dots,x_n+y_n)\\
            &= (y_1+x_1,\dots,y_n+x_n)\\
            &= \y+\x
        \end{align*}
        as desired.\par
        To prove that $\R^n$ obeys FA2 from Definition \ref{dfn:7.8}, it will suffice to show that for all $\x,\y,\z\in\R^n$, $(\x+\y)+\z=\x+(\y+\z)$. Let $\x,\y,\z$ be arbitrary elements of $\R^n$. Then by Definition \ref{dfn:18.2},
        \begin{align*}
            (\x+\y)+\z &= (x_1+y_1,\dots,x_n+y_n)+\z\\
            &= ((x_1+y_1)+z_1,\dots,(x_n+y_n)+z_n)\\
            &= (x_1+(y_1+z_1),\dots,x_n+(y_n+z_n))\\
            &= \x+(y_1+z_1,\dots,y_n+z_n)\\
            &= \x+(\y+\z)
        \end{align*}
        as desired.\par
        To prove that $\R^n$ obeys FA3 from Definition \ref{dfn:7.8}, it will suffice to find an element $0\in\R^n$ such that $\x+0=0+\x=\x$ for all $\x\in\R^n$. Choose $\bm{0}$ to be our $0$. Let $\x$ be an arbitrary element of $\R^n$. Then by Definition \ref{dfn:18.2},
        \begin{align*}
            \x+\bm{0} &= (x_1+0,\dots,x_n+0)\\
            &= (x_1,\dots,x_n)\\
            &= \x\\
            &= (0+x_1,\dots,0+x_n)\\
            &= \bm{0}+\x
        \end{align*}
        as desired.\par
        To prove that $\R^n$ obeys FA4 from Definition \ref{dfn:7.8}, it will suffice to show that for all $\x\in\R^n$, there exists $\y\in\R^n$ such that $\x+\y=\y+\x=0$. Let $\x$ be an arbitrary element of $\R^n$. Choose $\y=(-x_1,\dots,-x_n)$. Then by Definition \ref{dfn:18.2},
        \begin{align*}
            \x+\y &= (x_1+(-x_1),\dots,x_n+(-x_n))\\
            &= (0,\dots,0)\\
            &= \bm{0}\\
            &= ((-x_1)+x_1,\dots,(-x_n)+x_n)\\
            &= \y+\x
        \end{align*}
        as desired.\par
        To prove that $\R^n$ obeys VS1, it will suffice to show that for all $\lambda,\mu\in\R$ and $\x\in\R^n$, we have $(\lambda\mu)\x=\lambda(\mu\x)$. Let $\lambda,\mu$ be arbitrary elements of $\R$, and let $\x$ be an arbitrary element of $\R^n$. Then by Definition \ref{dfn:18.2},
        \begin{align*}
            (\lambda\mu)\x &= ((\lambda\mu)x_1,\dots,(\lambda\mu)x_n)\\
            &= (\lambda(\mu x_1),\dots,\lambda(\mu x_n))\\
            &= \lambda(\mu\x)
        \end{align*}
        as desired.\par
        To prove that $\R^n$ obeys VS2, it will suffice to show that for all $\lambda,\mu\in\R$ and $\x\in\R^n$, we have $(\lambda+\mu)\x=\lambda\x+\mu\x$. Let $\lambda,\mu$ be arbitrary elements of $\R$, and let $\x$ be an arbitrary element of $\R^n$. Then by Definition \ref{dfn:18.2},
        \begin{align*}
            (\lambda+\mu)\x &= ((\lambda+\mu)x_1,\dots,(\lambda+\mu)x_n)\\
            &= (\lambda x_1+\mu x_1,\dots,\lambda x_n+\mu x_n)\\
            &= (\lambda x_1,\dots,\lambda x_n)+(\mu x_1,\dots,\mu x_n)\\
            &= \lambda\x+\mu\x
        \end{align*}
        as desired.\par
        To prove that $\R^n$ obeys VS3, it will suffice to show that for all $\lambda\in\R$ and $\x,\y\in\R^n$, we have $\lambda(\x+\y)=\lambda\x+\lambda\y$. Let $\lambda$ be an arbitrary element of $\R$, and let $\x,\y$ be arbitrary elements of $\R^n$. Then by Definition \ref{dfn:18.2},
        \begin{align*}
            \lambda(\x+\y) &= \lambda(x_1+y_1,\dots,x_n+y_n)\\
            &= (\lambda(x_1+y_1),\dots,\lambda(x_n+y_n))\\
            &= (\lambda x_1+\lambda y_1,\dots,\lambda x_n+\lambda y_n)\\
            &= (\lambda x_1,\dots,\lambda x_n)+(\lambda y_1,\dots,\lambda y_2)\\
            &= \lambda\x+\lambda\y
        \end{align*}
        as desired.\par
        To prove that $\R^n$ obeys VS4, it will suffice to show that for all $\x\in\R^n$, we have $1\x=\x$. Let $\x$ be an arbitrary element of $\R^n$. Then by Definition \ref{dfn:18.2},
        \begin{align*}
            1\x &= (1x_1,\dots,1x_n)\\
            &= (x_1,\dots,x_n)\\
            &= \x
        \end{align*}
        as desired.
    \end{proof}
\end{exercise}

\begin{remark}\label{rmk:18.4}
    Since $\R^n$ with the two operations defined as above satisfies these eight axioms, we call $\R^n$ a \textbf{vector space}.
\end{remark}

\begin{exercise}\label{exr:18.5}
    Prove that if $\x\in\R^n$, then $0\x=\bm{0}$.
    \begin{proof}
        By Definition \ref{dfn:18.2}, we have that
        \begin{align*}
            0\x &= (0x_1,\dots,0x_n)\\
            &= (0,\dots,0)\\
            &= \bm{0}
        \end{align*}
        as desired.
    \end{proof}
\end{exercise}

\begin{definition}\label{dfn:18.6}
    Let $\x\in\R^n$. The \textbf{norm} of $\x$ is defined as
    \begin{equation*}
        \norm{\x} = \sqrt{x_1^2+\cdots+x_n^2}
    \end{equation*}
\end{definition}

\begin{definition}\label{dfn:18.7}
    We call $\norm{\y-\x}$ the \textbf{distance} between $\x$ and $\y$.
\end{definition}

\begin{remark}\label{rmk:18.8}
    If $n=1$, the norm coincides with the definition of the absolute value in $\R$.
\end{remark}

\begin{lemma}\label{lem:18.9}
    \leavevmode
    \begin{enumerate}[label=\textup{(}\alph*\textup{)},ref={\thelemma\alph*}]
        \item \label{lem:18.9a}If $x,y\in\R$, then $xy\leq\frac{x^2+y^2}{2}$.
        \begin{proof}
            Let $x,y$ be arbitrary elements of $\R$. Then by Lemma \ref{lem:7.26}, $0\leq(x-y)^2$. Therefore, we have that
            \begin{align*}
                xy &= \frac{2xy+0}{2}\\
                &\leq \frac{2xy+(x-y)^2}{2}\\
                &= \frac{2xy+x^2-2xy+y^2}{2}\\
                &= \frac{x^2+y^2}{2}
            \end{align*}
            as desired.
        \end{proof}
        \item \label{lem:18.9b}If $\x,\y\in\R^n$, then $|x_1y_1+\cdots+x_ny_n|\leq\norm{\x}\cdot\norm{\y}$.
        \begin{proof}
            Suppose first that $\norm{\x}=\norm{\y}=1$. Then by Definition \ref{dfn:18.6}, $\norm{\x}=1=\sqrt{x_1^2+\cdots+x_n^2}$, from which it follows that $1=x_1^2+\cdots+x_n^2$. Therefore, we have that
            \begin{align*}
                |x_1y_1+\cdots+x_ny_n| &\leq |x_1y_1|+\cdots+|x_ny_n|\tag*{Lemma \ref{lem:8.8}}\\
                &= |x_1||y_1|+\cdots+|x_n||y_n|\\
                &\leq \frac{|x_1|^2+|y_1|^2}{2}+\cdots+\frac{|x_n|^2+|y_n|^2}{2}\tag*{Lemma \ref{lem:18.9a}}\\
                &= \frac{x_1^2+y_1^2}{2}+\cdots+\frac{x_n^2+y_n^2}{2}\\
                &= \frac{(x_1^2+\cdots+x_n^2)+(y_1^2+\cdots+y_n^2)}{2}\\
                &= \frac{1+1}{2}\\
                &= 1
            \end{align*}
            as desired.\par
            Now let $\x,\y$ be arbitrary elements of $\R^n$. Consider the vectors $\mathbf{u}_\x,\mathbf{u}_\y$ defined by $\mathbf{u}_\x=\frac{\x}{\norm{\x}}$ and $\mathbf{u}_\y=\frac{\y}{\norm{\y}}$. By the proof of the first case, we have that
            \begin{align*}
                |x_1y_1+\cdots+x_ny_n| &= \norm{\x}\cdot\norm{\y}\cdot\left| \frac{x_1y_1}{\norm{\x}\cdot\norm{\y}}+\cdots+\frac{x_ny_n}{\norm{\x}\cdot\norm{\y}} \right|\\
                &= \norm{\x}\cdot\norm{\y}\cdot|u_{\x_1}u_{\y_1}+\cdots+u_{\x_n}u_{\y_n}|\\
                &\leq \norm{\x}\cdot\norm{\y}\cdot 1\\
                &= \norm{\x}\cdot\norm{\y}
            \end{align*}
            as desired.
        \end{proof}
    \end{enumerate}
\end{lemma}

\begin{theorem}\label{trm:18.10}
    If $\x,\y\in\R^n$ and $\lambda\in\R$, then
    \begin{enumerate}[label=\textup{(}\alph*\textup{)},ref={\thetheorem\alph*}]
        \item \label{trm:18.10a}$\norm{\x}\geq 0$. Moreover, $\norm{\x}=0$ if and only if $\x=\bm{0}$.
        \begin{proof}
            Let $\x$ be an arbitrary element of $\R^n$.\par\smallskip
            We first prove that $\norm{\x}\geq 0$. By Lemma \ref{lem:7.26}, $x_i^2\geq 0$ for all $i\in[n]$. Thus, by Definition \ref{dfn:7.21}, $x_1^2+\cdots+x_n^2\geq 0$. Therefore, we have by Definition \ref{dfn:18.6} that $\norm{\x}=\sqrt{x_1^2+\cdots+x_n^2}\geq 0$, as desired.\par\smallskip
            We now prove that $\norm{\x}=0$ if and only if $\x=\bm{0}$. Suppose first that $\norm{\x}=0$. Then by Definition \ref{dfn:18.6} and Script \ref{sct:7}, $x_1^2+\cdots+x_n^2=0$. Now suppose for the sake of contradiction that $\x\neq\bm{0}$. Then there exists an $x_i$ such that $x_i\neq 0$. Thus, by Lemma \ref{lem:7.26}, $x_i^2>0$. Additionally, $x_j^2\geq 0$ for all $j\in[n]$. Thus, we have that $0<x_i^2\leq x_1^2+\cdots+x_n^2$. But by Definition \ref{dfn:3.1}, this implies that $x_1^2+\cdots+x_n^2\neq 0$, a contradiction.\par
            Now suppose that $\x=\bm{0}$. Then by Definition \ref{dfn:18.6}, $\norm{\x}=\sqrt{0^2+\cdots+0^2}=0$, as desired.
        \end{proof}
        \item \label{trm:18.10b}$\norm{\lambda\x}=|\lambda|\cdot\norm{\x}$.
        \begin{proof}
            Let $\lambda$ be an arbitrary element of $\R$, and let $\x$ be an arbitrary element of $\R^n$. Then we have that
            \begin{align*}
                \norm{\lambda\x} &= \sqrt{(\lambda x_1)^2+\cdots+(\lambda x_n)^2}\tag*{Definition \ref{dfn:18.6}}\\
                &= |\lambda|\cdot\sqrt{x_1^2+\cdots+x_n^2}\\
                &= |\lambda|\cdot\norm{\x}\tag*{Definition \ref{dfn:18.6}}
            \end{align*}
            as desired.
        \end{proof}
        \item \label{trm:18.10c}$\norm{\x+\y}\leq\norm{\x}+\norm{\y}$.
        \begin{proof}
            Let $\x,\y$ be arbitrary elements of $\R^n$. Then we have that
            \begin{align*}
                \norm{\x+\y} &= \sqrt{(x_1+y_1)^2+\cdots+(x_n+y_n)^2}\tag*{Definition \ref{dfn:18.6}}\\
                &= \sqrt{(x_1^2+\cdots+x_n^2)+(2x_1y_1+\cdots+2x_ny_n)+(y_1^2+\cdots+y_n^2)}\\
                &\leq \sqrt{\norm{\x}^2+2\norm{\x}\norm{\y}+\norm{\y}^2}\tag*{Lemma \ref{lem:18.9}}\\
                &= \sqrt{(\norm{\x}+\norm{\y})^2}\\
                &= \norm{\x}+\norm{\y}
            \end{align*}
            as desired.
        \end{proof}
    \end{enumerate}
\end{theorem}

\begin{corollary}\label{cly:18.11}
    If $\x,\y,\z\in\R^n$ and $\lambda\in\R$, then
    \begin{enumerate}[label=\textup{(}\alph*\textup{)}]
        \item $\norm{\x-\z}\leq\norm{\x-\y}+\norm{\y-\z}$.
        \item $|\norm{\x}-\norm{\y}|\leq\norm{\x-\y}$.
    \end{enumerate}
    \begin{proof}
        The proofs are symmetric to those of Lemma \ref{lem:8.8}.
    \end{proof}
\end{corollary}

\marginnote{7/10:}The next goal is to "topologize" $\R^n$. To discuss topology on $\R^n$, we first need to introduce notions for $\R^n$ that are analogous to open and closed intervals for $\R$.

\begin{remark}\label{rmk:18.12}
    For $\x=(x_1,\dots,x_n)\in\R^n$ and $\y=(y_1,\dots,y_m)\in\R^m$, we identify $(\x,\y)\in\R^n\times\R^m$ with $(x_1,\dots,x_n,y_1,\dots,y_m)\in\R^{n+m}$. So if $A\subset\R^n$ and $B\subset\R^m$, we can consider $A\times B$ to be a subset of $\R^{n+m}$.\par
    If also $C\in\R^k$, then $(A\times B)\times C$ and $A\times(B\times C)$ correspond to the same subset of $\R^{n+m+k}$ under this identification; we write $A\times B\times C$ for this set.
\end{remark}

\begin{definition}\label{dfn:18.13}
    An \textbf{open rectangle} in $\R^n$ is a set of the form $(a_1,b_1)\times\cdots\times(a_n,b_n)$, a product of open intervals. Similarly, a \textbf{closed rectangle} in $\R^n$ is a set of the form $[a_1,b_1]\times\cdots\times[a_n,b_n]$. We allow the possibility that $a_j=b_j$ (where $[a_j,a_j]=\{a_j\}$). If there is at least one $j$ with $a_j=b_j$, then we say that the rectangle is \textbf{degenerate}; otherwise, we say that the rectangle is \textbf{non-degenerate}.
\end{definition}

\begin{definition}\label{dfn:18.14}
    A subset $U\subset\R^n$ is \textbf{open} if for all $\x\in U$, there exists an open rectangle $R$ such that $\x\in R\subset U$. A subset $C\in\R^n$ is \textbf{closed} if its compliment is open.
\end{definition}

\begin{exercise}\label{exr:18.15}
    Decide whether each of the following is an open set in $\R^2$.
    \begin{enumerate}[label={\textup{(}\alph*\textup{)}}]
        \item $\{(x_1,x_2)\mid x_1,x_2\in\R,\ x_1>0,\ x_2>0\}$.
        \begin{proof}
            To prove that $U=\{(x_1,x_2)\mid x_1,x_2\in\R,\ x_1>0,\ x_2>0\}$ is open, Definition \ref{dfn:18.14} tells us that it will suffice to show that for all $\x\in U$, there exists an open rectangle $R$ such that $\x\in R\subset U$. Let $\x$ be an arbitrary element of $U$. Then by the definition of $U$, $0<x_1$ and $0<x_2$. It follows by Theorem \ref{trm:5.2} and Corollary \ref{cly:6.12}, there exist $a_1,b_1,a_2,b_2$ such that $0<a_1<x_1<b_1$ and $0<a_2<x_2<b_2$. Thus, by Equations \ref{eqn:8.1}, $x_1\in(a_1,b_1)$ and $x_2\in(a_2,b_2)$. Consequently, if we let $R=(a_1,b_1)\times(a_2,b_2)$, Definition \ref{dfn:18.13} guarantees that $R$ is an open rectangle. Additionally, Definition \ref{dfn:1.15} asserts that $(x_1,x_2)=\x\in R$, as desired. Additionally, if $\y$ is any vector in $R$, then by the definition of $R$, $0<a_1<y_1$ and $0<a_2<y_2$. Thus, by transitivity, $\y\in U$. Therefore, by Definition \ref{dfn:1.3}, $R\subset U$, as desired.
        \end{proof}
        \item $\{(x,0)\mid x\in\R\}$.
        \begin{proof}
            To prove that $U=\{(x,0)\mid x\in\R\}$ is not open, Definition \ref{dfn:18.14} tells us that it will suffice to find an $\x\in U$ such that for all open rectangles $R$ containing $\x$, $R\not\subset U$. Let $\x=(0,0)$, and let $R$ be an arbitrary open rectangle containing $\x$. By Definitions \ref{dfn:18.13} and \ref{dfn:1.15} along with Equations \ref{eqn:8.1}, $a_1<0<b_1$ and $a_2<0<b_2$. Thus, by consecutive applications of Theorem \ref{trm:5.2}, there exist points $y_1,y_2\in\R$ such that $a_1<y_1<0$ and $a_2<y_2<0$. It follows that $\y=(y_1,y_2)\in R$. However, since $y_2\neq 0$ by Definition \ref{dfn:3.1}, $\y\notin U$. Therefore, by Definition \ref{dfn:1.3}, $R\not\subset U$, as desired.
        \end{proof}
    \end{enumerate}
\end{exercise}

\begin{exercise}\label{exr:18.16}
    Show that if $R_1,\dots,R_m$ are open rectangles containing $\x\in\R^n$, then $R=R_1\cap\cdots\cap R_m$ is an open rectangle containing $\x\in\R^n$. If $R=(a_1,b_1)\times\cdots\times(a_n,b_n)$, derive formulas for $a_i$ and $b_i$ in terms of the corresponding quantities for $R_1,\dots,R_m$.
    \begin{proof}
        Let $R_i=(r_{ij},s_{ij})_{j=1}^n$ for all $i\in[m]$. To prove that $R=\bigcap_{i=1}^mR_i$ is an open rectangle containing $\x$, Definitions \ref{dfn:18.13} and \ref{dfn:1.15} tell us that it will suffice to show that $R$ is the Cartesian product of open intervals, each containing its respective $x_j$. Since $\x\in R_i$ for all $i\in[m]$, we have by Definition \ref{dfn:1.15} that $x_j\in(r_{ij},s_{ij})$ for all $i\in[m],\ j\in[n]$. Thus, by Corollary \ref{cly:3.19}, $\bigcap_{i=1}^m(r_{ij},s_{ij})$ is a region (hence an open interval by Corollary \ref{cly:4.11} and Lemma \ref{lem:8.3}) containing $x_j$ for all $j\in[n]$. Therefore, since $R=\bigcap_{i=1}^mR_i=\prod_{i=1}^n(\bigcap_{i=1}^m(r_{ij},s_{ij}))$ by Script \ref{sct:1}, we have that $R$ is the Cartesian product of open intervals, each containing its respective $x_j$, as desired.\par
        Let $a_j=\max_{i=1}^m(r_{ij})$ and let $b_j=\min_{i=1}^m(s_{ij})$ for all $j\in[n]$. To prove that $R=(a_j,b_j)_{j=1}^n$, Definition \ref{dfn:1.2} tells us that it will suffice to show that every $\x\in R$ is an element of $(a_j,b_j)_{j=1}^n$ and vice versa. Suppose first that $\x$ is an arbitrary element of $R$. Then by Definition \ref{dfn:1.6}, $\x\in R_i$ for all $i\in[m]$. It follows by Definition \ref{dfn:1.15} that $x_j\in(r_{ij},s_{ij})$ for all $i\in[m],\ j\in[n]$, including the $j,j'$ for which $r_{ij}$ is at its maximum and $s_{ij'}$ is at its minimum. In other words, $x_j\in(a_j,b_j)$ for all $j\in[n]$. Therefore, by Definition \ref{dfn:1.15}, $\x\in(a_j,b_j)_{j=1}^n$, as desired. The proof is symmetric in the other direction.
    \end{proof}
\end{exercise}

\begin{definition}\label{dfn:18.17}
    The \textbf{open ball} (in $\R^n$ with center $\mathbf{p}$ and radius $r>0$) is defined as
    \begin{equation*}
        B(\mathbf{p},r) = \{\x\in\R^n\mid\norm{\x-\mathbf{p}}<r\}
    \end{equation*}
    The \textbf{closed ball} (in $\R^n$ with center $\mathbf{p}$ and radius $r>0$) is defined as
    \begin{equation*}
        \overline{B}(\mathbf{p},r) = \{\x\in\R^n\mid\norm{\x-\mathbf{p}}\leq r\}
    \end{equation*}
\end{definition}

\begin{remark}\label{rmk:18.18}
    In $\R^1$, an open rectangle is also an open ball, and vice versa.
\end{remark}

The following results illustrate how open rectangles and open balls in $\R^n$ are "compatible" with each other.

\begin{lemma}\label{lem:18.19}
    Fix $\x\in\R$.
    \begin{enumerate}[label={\textup{(}\alph*\textup{)}},ref={\thelemma\alph*}]
        \item \label{lem:18.19a}If $R$ is an open rectangle containing $\x$, then there exists $r>0$ such that $B(\x,r)\subset R$.
        \begin{proof}
            Since $\x\in R$, Definitions \ref{dfn:18.13} and \ref{dfn:1.15} tell us that that $x_i\in(a_i,b_i)$ for all $i\in[n]$. Additionally, we know by Corollary \ref{cly:4.11} and Lemma \ref{lem:8.3} that each $(a_i,b_i)$ is an open interval. Combining the last two results, we have by Lemma \ref{lem:8.10} that for each $i\in[n]$, there exists $\delta_i>0$ such that $(x_i-\delta_i,x_i+\delta_i)\subset(a_i,b_i)$. Let $r=\min\{\delta_i\}_{i=1}^n$.\par
            To prove that $B(\x,r)\subset R$, Definition \ref{dfn:1.3} tells us that it will suffice to show that every $\y\in B(\x,r)$ is an element of $R$. Let $\y$ be an arbitrary element of $B(\x,r)$. Then by Definition \ref{dfn:18.17}, $\norm{\y-\x}<r$. It follows that
            \begin{align*}
                |y_i-x_i| &= \sqrt{(y_i-x_i)^2}\\
                &\leq \sqrt{(y_1-x_1)^2+\cdots+(y_n-x_n)^2}\tag*{Lemma \ref{lem:7.26}}\\
                &= \norm{\y-\x}\tag*{Definition \ref{dfn:18.6}}\\
                &< r
            \end{align*}
            for all $i\in[n]$. Thus, by the definition of $r$, $|y_i-x_i|\leq\delta_i$ for all $i\in[n]$. Consequently, by Exercise \ref{exr:8.9} and Definition \ref{dfn:1.3}, $y_i\in(a_i,b_i)$ for all $i\in[n]$. Therefore, by Definitions \ref{dfn:1.15} and \ref{dfn:18.13}, $\y\in R$, as desired.
        \end{proof}
        \item \label{lem:18.19b}If $B$ is an open ball containing $\x$, then there exists an open rectangle $R$ such that $\x\in R\subset B$.
        \begin{lemma*}
            If $\x\in\R^n$, then $\norm{\x}\leq\sum_{i=1}^n|x_i|$.
            \begin{proof}
                By Definition \ref{dfn:18.2}, we can decompose $\x$ into the sum of $n$ unit vectors $\mathbf{u_i}$ (where $\mathbf{u_i}$ points one unit in the $i^\text{th}$ direction), each scaled by $x_i$; symbolically, let $\x=\sum_{i=1}^nx_i\mathbf{u_i}$. Therefore,
                \begin{align*}
                    \norm{\x} &= \norm{\sum_{i=1}^nx_i\mathbf{u_i}}\\
                    &= \sum_{i=1}^n\norm{x_i\mathbf{u_i}}\tag*{Theorem \ref{trm:18.10c}}\\
                    &= \sum_{i=1}^n|x_i|\cdot\norm{\mathbf{u_i}}\tag*{Theorem \ref{trm:18.10b}}\\
                    &= \sum_{i=1}^n|x_i|\cdot\sqrt{1^2}\tag*{Definition \ref{dfn:18.6}}\\
                    &= \sum_{i=1}^n|x_i|
                \end{align*}
                as desired.
            \end{proof}
        \end{lemma*}
        \begin{proof}[Proof of Lemma \ref{lem:18.19b}]
            Suppose $\x\in B(\y,r)$. Then by Definition \ref{dfn:18.17}, $\norm{\x-\y}<r$. Thus, we can define $r'=r-\norm{\x-\y}$ such that $r'>0$. With this term defined, we can let $R=(x_i-\frac{r'}{n},x_i+\frac{r'}{n})_{i=1}^n$.\par
            To prove that $\x\in R$, Definition \ref{dfn:18.13} tells us that it will suffice to show that $x_i\in(x_i-\frac{r'}{n},x_i+\frac{r'}{n})$ for all $i\in[n]$. But since $|x_i-x_i|=0<\frac{r'}{n}$ for all $i\in[n]$, Exercise \ref{exr:8.9} asserts that this is true.\par
            To prove that $R\subset B$, Definition \ref{dfn:1.3} tells us that it will suffice to show that every $\z\in R$ is an element of $B$. Let $\z$ be an arbitrary element of $R$. Then by Definition \ref{dfn:18.13}, $z_i\in(x_i-\frac{r'}{n},x_i+\frac{r'}{n})$ for all $i\in[n]$. It follows by Exercise \ref{exr:8.9} that $|z_i-x_i|<\frac{r'}{n}$ for all $i\in[n]$. Consequently,
            \begin{align*}
                \norm{\z-\y} &\leq \norm{\z-\x}+\norm{\x-\y}\tag*{Corollary \ref{cly:18.11}}\\
                &\leq \sum_{i=1}^n|z_i-x_i|+\norm{\x-\y}\tag*{Lemma}\\
                &< \sum_{i=1}^n\frac{r'}{n}+\norm{\x-\y}\\
                &= r'+\norm{\x-\y}\\
                &= r-\norm{\x-\y}+\norm{\x-\y}\\
                &= r
            \end{align*}
            Therefore, by Definition \ref{dfn:18.17}, $\z\in B$, as desired.
        \end{proof}
    \end{enumerate}
\end{lemma}

\begin{corollary}\label{cly:18.20}
    A set $U\subset\R^n$ is open if and only if for every $\x\in U$, there exists $r>0$ such that $B(\x,r)\subset U$.
    \begin{proof}
        Suppose first that $U\subset\R^n$ is open. Let $\x$ be an arbitrary element of $U$. By Definition \ref{dfn:18.14}, there exists an open rectangle $R$ such that $\x\in R\subset U$. Therefore, by Lemma \ref{lem:18.19}, there exists $r>0$ such that $B(\x,r)\subset R\subset U$, as desired.\par
        Now suppose that for all $\x\in U$, there exists $r>0$ such that $B(\x,r)\subset U$. To prove that $U$ is open, Definition \ref{dfn:18.14} tells us that it will suffice to show that for all $\x\in U$, there exists an open rectangle $R$ such that $\x\in R\subset U$. Let $\x$ be an arbitrary element of $U$. Then there exists $r>0$ such that $B(\x,r)\subset U$. Therefore, by Lemma \ref{lem:18.19}, there exists an open rectangle $R$ such that $\x\in R\subset B\subset U$, as desired.
    \end{proof}
\end{corollary}

\begin{corollary}\label{cly:18.21}\marginnote{\emph{7/14:}}
    Open balls are open and closed balls are closed.
    \begin{proof}
        We will take this one claim at a time.\par\smallskip
        Let $B(\x,r)$ be an arbitrary open ball. To prove that $B$ is open, Definition \ref{dfn:18.14} tells us that it will suffice to show that for all $\y\in B$, there exists an open rectangle $R$ such that $\y\in R\subset B$. But by Lemma \ref{lem:18.19}, this is true.\par
        Let $\overline{B}(\x,r)$ be an arbitrary closed ball. To prove that $\overline{B}$ is closed, Definition \ref{dfn:18.14} tells us that it will suffice to show that $\R^n\setminus\overline{B}$ is open. To do this, Definition \ref{dfn:18.14} tells us again that it will suffice to verify that for all $\y\in\R^n\setminus\overline{B}$, there exists an open rectangle $R$ such that $\y\in R\subset\R^n\setminus\overline{B}$. Let $\y$ be an arbitrary element of $\R^n\setminus\overline{B}$. Then by Definition \ref{dfn:18.17}, $\norm{\y-\x}>r$. Thus, $\norm{\y-\x}-r>0$, so we may define $r'=\norm{\y-\x}-r$. Now consider $B(\y,r')$. By Lemma \ref{lem:18.19}, there exists an open rectangle $R$ such that $\y\in R\subset B$. Consequently, by Script \ref{sct:1}, the only thing left to do to verify that $R\subset\R^n\setminus\overline{B}$ is to show that $B\cap\overline{B}=\emptyset$. As such, suppose for the sake of contradiction that $B\cap\overline{B}\neq\emptyset$. Then there exists $\z\in\R^n$ such that $\z\in B$ and $z\in\overline{B}$. It follows by consecutive applications of Definition \ref{dfn:18.17} that $\norm{\z-\y}<r'$ and $\norm{\z-\x}\leq r$. But then we have that
        \begin{align*}
            \norm{\x-\y} &\leq \norm{\x-\z}+\norm{\z-\y}\tag*{Corollary \ref{cly:18.11}}\\
            &< r'+r\\
            &= \norm{\y-\x}-r+r\\
            &= \norm{\x-\y}
        \end{align*}
        a contradiction, as desired.
    \end{proof}
\end{corollary}

\begin{proposition}\label{prp:18.22}
    Let $U\subset\R^n$. The following are equivalent:
    \begin{enumerate}[label={\textup{(}\alph*\textup{)}}]
        \item $U$ is open.
        \item $U$ is a (possibly empty) union of open balls.
        \item $U$ is a (possibly empty) union of open rectangles.
    \end{enumerate}
    \begin{proof}
        As in Theorem \ref{trm:11.5}, to prove that statements a-c are equivalent, it will suffice to verify that $a\Rightarrow b$, $b\Rightarrow c$, and $c\Rightarrow a$. Let's begin.\par\smallskip
        First, suppose that $U$ is open. Then by Corollary \ref{cly:18.20}, for every $\x\in U$, there exists $r>0$ such that $B_\x(\x,r)\subset U$. Therefore, $U=\bigcup_{\x\in U}B_\x$, as desired.\par
        Second, suppose that $U$ is a union of open balls. Then for every open ball $B(\x,r)$ comprising $U$, Lemma \ref{lem:18.19} asserts that for every $\y\in B$, there exists an open rectangle $R_\y$ such that $\y\in R_\y\subset B$. Therefore, $U=\bigcup_{\y\in U}R_\y$, as desired.\par
        Third, suppose that $U$ is a union of open rectangles. Then for every $\x\in U$, there exists an open rectangle $R$ such that $\x\in R\subset U$. Therefore, by Definition \ref{dfn:18.14}, $U$ is open, as desired.
    \end{proof}
\end{proposition}

\begin{remark}\label{rmk:18.23}
    If $X\subset\R^n$, then $X$ is also a topolotical space with the \textbf{subspace topology}. That is, $A\subset X$ is \textbf{open} (in $X$) if there exists an open set $U\subset\R^n$ such that $X\cap U=A$. (See Script \ref{sct:8}.)
\end{remark}

We now discuss functions between Euclidean spaces.

\begin{definition}\label{dfn:18.24}
    Let $A\subset\R^n$ and let $f:A\to\R$. Define the \textbf{graph} of $f$ by
    \begin{equation*}
        \graph(f) = \{(x_1,\dots,x_n,f(x_1,\dots,x_n))\in\R^{n+1}\mid(x_1,\dots,x_n)\in A\}
    \end{equation*}
\end{definition}

\begin{exercise}\label{exr:18.25}
    For each of the following functions, describe the graph as a subset of $\R^3$.
    \begin{enumerate}[label={(\alph*)}]
        \item $f:\R^2\to\R$ given by $f(x,y)=2$ for all $(x,y)\in\R^2$.
        \begin{proof}[Description]
            For this function, we have $\graph(f)=\{(x,y,2)\in\R^3\mid(x,y)\in\R^2\}$. This makes the graph equal to the set of all points in $\R^3$ with $z=2$, which will be a planar, constant, infinite subspace of $\R^3$.
        \end{proof}
        \item $f:\R^2\to\R$ given by $f(x,y)=x+y+1$ for all $(x,y)\in\R^2$.
        \begin{proof}[Description]
            For this function, we have $\graph(f)=\{(x,y,x+y+1)\in\R^3\mid(x,y)\in\R^2\}$. Thus, the graph will be a planar, sloped, infinite subspace of $\R^3$ with gradient pointing in the $\mathbf{\hat{\imath}}+\mathbf{\hat{\jmath}}$ direction.
        \end{proof}
        \item $f:\R^2\to\R$ given by $f(x,y)=x^2+y^2$ for all $(x,y)\in\R^2$.
        \begin{proof}[Description]
            For this function, we have $\graph(f)=\{(x,y,x^2+y^2)\in\R^3\mid(x,y)\in\R^2\}$. Thus, the graph will be the paraboloid centered at the origin.
        \end{proof}
    \end{enumerate}
\end{exercise}

In Script \ref{sct:9}, we gave a definition of continuity that we can generalize to this case:

\begin{definition}\label{dfn:18.26}
    Let $X,Y$ be topological spaces. A function $f:X\to Y$ is \textbf{continuous} if for every open set $U\subset Y$, the preimage $f^{-1}(U)$ is open in $X$.\par
    The function $f:X\to Y$ is \textbf{continuous} at $x\in X$ if for every open set $U\subset Y$ containing $f(x)$, the preimage $f^{-1}(U)$ is open in $X$.
\end{definition}

\begin{theorem}\label{trm:18.27}\leavevmode
    \begin{enumerate}[label={\textup{(}\alph*\textup{)}}]
        \item A function $f:X\to Y$ is continuous if and only if it is continuous at every $x\in X$.
        \item A function $f:X\to Y$ is continuous if and only if $f^{-1}(B)$ is closed in $X$ whenever $B$ is closed in $Y$.
    \end{enumerate}
    \begin{proof}
        The proofs are symmetric to those of Theorem \ref{trm:9.10} and Proposition \ref{prp:9.5}, respectively.
    \end{proof}
\end{theorem}




\end{document}