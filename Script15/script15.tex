\documentclass[../main.tex]{subfiles}

\pagestyle{main}
\renewcommand{\chaptermark}[1]{\markboth{\chaptername\ \thechapter}{}}
\setcounter{chapter}{14}

\begin{document}




\chapter{Sequences}\label{sct:15}
\section{Journal}
\begin{definition}\label{dfn:15.1}\marginnote{5/6:}
    A \textbf{sequence} (of real numbers) is a function $a:\N\to\R$.\par
    By setting $a_n=a(n)$, we can think of a sequence as a list $a_1,a_2,a_3,\dots$ of real numbers. We use the notation $(a_n)_{n=1}^\infty$ for such a sequence, or if there is no possibility of confusion, we sometimes abbreviate this and write simply $(a_n)$. More generally, we also use the term sequence to refer to the function defined on $\{n\in\N\cup\{0\}\mid n\geq n_0\}$ for any fixed $n_0\in\N\cup\{0\}$. We write $(a_n)_{n=n_0}^\infty$ for such a sequence.
\end{definition}

\begin{definition}\label{dfn:15.2}
    We say that a sequence $(a_n)$ \textbf{converges} to a point $p\in\R$ if for every open interval $I$ containing $p$, there exists $N\in\N$ such that if $n\geq N$, then $a_n\in I$. If a sequence converges to some point, we say it is \textbf{convergent}. If $(a_n)$ does not converge to any point, we say that the sequence \textbf{diverges} or is \textbf{divergent}.
\end{definition}

\begin{exercise}\label{exr:15.3}
    Show that a sequence $(a_n)$ converges to $p$ if and only if any region containing $p$ contains all but finitely many terms of the sequence.
    \begin{proof}
        Suppose first that $(a_n)$ converges to $p$. Let $R$ be an arbitrary region containing $p$. By Corollary \ref{cly:4.11} and Lemma \ref{lem:8.3}, $R$ is an open interval. Thus, by Definition \ref{dfn:15.2}, there exists $N\in\N$ such that if $n\geq N$, then $a_n\in R$. To prove that $R$ contains all but finitely many terms of the sequence, it will suffice to show that the set $A=\{a_n\mid a_n\notin R\}$ is finite. Since $a_n\in R$ for all $n\geq N$, it follows that $a_n\in R$ only if $n<N$. Thus, by Script \ref{sct:1}, $A\subset\{a_n\mid 0\leq n<N\}$. Since the latter set is clearly finite, it follows by Script \ref{sct:1} that $A$ is finite.\par
        Now suppose that any region containing $p$ contains all but finitely many terms $(a_n)$. To prove that $(a_n)$ converges to $p$, Definition \ref{dfn:15.2} tells us that it will suffice to show that for every open interval $I$ containing $p$, there exists $N\in\N$ such that if $n\geq N$, then $a_n\in I$. Let $I$ be an arbitrary open interval containing $p$. Then by Theorem \ref{trm:4.10}, there exists a region $R$ containing $p$ such that $R\subset I$. It follows by the hypothesis that $A=\{a_n\mid a_n\notin R\}$ is finite. We divide into two cases ($|A|=0$ and $|A|\in\N$). Suppose first that $|A|=0$. Choose $N=n_0$. It follows that if $n\geq N$, then $a_n\notin A$, so $a_n\in R$, so $a_n\in I$, as desired. Now suppose that $|A|\in\N$. By Definition \ref{dfn:1.18}, $a^{-1}(A)\subset\N$. Consequently, by Lemma \ref{lem:3.4}, $a^{-1}(A)$ has a last point $N-1$. Choose $N=(N-1)+1$. It follows that if $n\geq N$, then $n\notin a^{-1}(A)$, so $a_n\notin A$, so $a_n\in R$, so $a_n\in I$, as desired.
    \end{proof}
\end{exercise}

\begin{theorem}\label{trm:15.4}
    Suppose that $(a_n)$ converges to both $p$ and to $p'$. Then $p=p'$.
    \begin{proof}
        Suppose for the sake of contradiction that $p\neq p'$. Then by Theorem \ref{trm:3.22}, there exist disjoint regions $R,R'$ containing $p,p'$, respectively. Additionally, by consecutive applications of Corollary \ref{cly:4.11} and Lemma \ref{lem:8.3}, $R,R'$ are open intervals. Combining these last two results, we have by consecutive applications of Definition \ref{dfn:15.2} that there exist $N,N'\in\N$ such that if $n\geq N$, then $a_n\in R$ and if $n\geq N'$, then $a_n\in R'$. Let $M=\max(N,N')$. It follows that $M\geq N$ and $M\geq N'$. Thus, by the above, $a_M\in R$ and $a_M\in R'$. But this implies by Definition \ref{dfn:1.6} that $a_M\in R\cap R'$. Therefore, by Definition \ref{dfn:1.9}, $R$ and $R'$ are not disjoint, a contradiction.
    \end{proof}
\end{theorem}

\begin{definition}\label{dfn:15.5}
    If a sequence $(a_n)$ converges to $p\in\R$, we call $p$ the \textbf{limit} of $(a_n)$ and write
    \begin{equation*}
        \lim_{n\to\infty}a_n = p
    \end{equation*}
\end{definition}

\begin{exercise}\label{exr:15.6}
    Which of the following sequences $(a_n)$ converge? Which diverge? For each that converges, what is the limit? Give proofs for your answers.
    \begin{enumerate}[label={(\alph*)}]
        \item $a_n=5$.
        \begin{proof}
            To prove that this sequence converges with limit $\lim_{n\to\infty}a_n=5$, Definition \ref{dfn:15.5} tells us that it will suffice to show that $(a_n)$ converges to 5. To do so, Definition \ref{dfn:15.2} tells us that it will suffice to verify that for every open interval $I$ containing 5, there exists $N\in\N$ such that if $n\geq N$, then $a_n\in I$. Let's begin.\par
            Let $I$ be an arbitrary open interval containing 5. Choose $N=1$. Let $n$ be an arbitrary natural number such that $n\geq N$. It follows by the definition of the sequence that $a_n=5\in I$, as desired.
        \end{proof}
        \item $a_n=n$.
        \begin{proof}
            To prove that this sequence diverges, Definition \ref{dfn:15.2} tells us that it will suffice to show that for every point $p\in\R$, there exists an open interval $I$ containing $p$ such that for all $N\in\N$, there exists an $n\geq N$ such that $a_n\notin I$. Let's begin.\par
            Let $p$ be an arbitrary element of $\R$. Choose $I=(p-1,p+1)$. Clearly $p\in I$. Let $N$ be an arbitrary natural number. By Corollary \ref{cly:6.12}, there exists a natural number $N'$ such that $p+1<N'$. Choose $M=\max(N,N')$. Thus, $M\geq N$. Additionally, it follows by the definition of the sequence that $a_M=M$. But this implies that $a_M\geq N'>p+1$, i.e., $a_M\notin I$ by Equations \ref{eqn:8.1}.
        \end{proof}
        \item $a_n=\frac{1}{n}$.
        \begin{proof}
            To prove that this sequence converges with limit $\lim_{n\to\infty}a_n=0$, Definitions \ref{dfn:15.5} and \ref{dfn:15.2} tell us that it will suffice to show that for every open interval $I$ containing 0, there exists $N\in\N$ such that if $n\geq N$, then $a_n\in I$. Let's begin.\par
            Let $I$ be an arbitrary interval containing 0. By Lemma \ref{lem:8.10}, there exists a region $(a,b)$ containing 0 such that $(a,b)\subset I$. By Corollary \ref{cly:6.12}, there exists a natural number $N$ such that $\frac{1}{b}<N$. Choose this $N$ to be our $N$. Now let $n$ be an arbitrary natural number such that $n\geq N$. It follows that $\frac{1}{b}<n$. Thus, since $0<b$ and $0<n$, we have by consecutive applications of Lemma \ref{lem:7.24} that $0<\frac{1}{n}<b$. Consequently, since we also know that $a<0$ and $a_n=\frac{1}{n}$, we have by transitivity and substitution that $a<a_n<b$. It follows by Equations \ref{eqn:8.1} that $a_n\in(a,b)$. Therefore, by Definition \ref{dfn:1.3}, $a_n\in I$, as desired.
        \end{proof}
        \item $a_n=(-1)^n$.
        \begin{proof}
            To prove that this sequence diverges, Definition \ref{dfn:15.2} tells us that it will suffice to show that for every point $p\in\R$, there exists an open interval $I$ containing $p$ such that for all $N\in\N$, there exists an $n\geq N$ such that $a_n\notin I$. Let's begin.\par
            Let $p$ be an arbitrary element of $R$. Choose $I=(p-1,p+1)$. Clearly $p\in I$. Let $N$ be an arbitrary natural number. By Script \ref{sct:0}, either $N$ is even and $N+1$ is odd or vice versa. Thus, let $N$ be even (the case where $N$ is odd is symmetric). It follows that $N\geq N$ yields $a_N=(-1)^{2m}=((-1)^2)^m=1^m=1$ and that $N+1\geq N$ yields $a_{N+1}=(-1)^{2m+1}=1(-1)^1=-1$. Now suppose for the sake of contradiction that $a_N\in I$ and $a_{N+1}\in I$. Since $a_N=1\in I$, we have by Equations \ref{eqn:8.1} that $p-1<1<p+1$. It follows by Definition \ref{dfn:7.21} that $p-3<-1<p-1$. But $-1<p-1$ implies by Equations \ref{eqn:8.1} that $a_{N+1}=-1\notin I$, a contradiction. Therefore, $N+1\geq N$ is a number such that $a_{N+1}\notin I$, as desired.
        \end{proof}
    \end{enumerate}
\end{exercise}

\begin{theorem}\label{trm:15.7}\marginnote{\emph{5/11:}}
    A sequence $(a_n)$ converges to $p\in\R$ if and only if for all $\epsilon>0$, there is some $N\in\N$ such that for all $n\geq N$, we have $|a_n-p|<\epsilon$.
    \begin{proof}
        Suppose first that $(a_n)$ converges to $p$. Let $\epsilon>0$ be arbitrary. Consider the $p$-containing region $R=(p-\epsilon,p+\epsilon)$. By Corollary \ref{cly:4.11} and \ref{lem:8.3}, $R$ is an open interval. Thus, by Definition \ref{dfn:15.2}, there exists $N\in\N$ such that if $n\geq N$, then $a_n\in R$. Choose this $N$ to be our $N$. Let $n$ be an arbitrary natural number such that $n\geq N$. Then $a_n\in(p-\epsilon,p+\epsilon)$. Therefore, by Exercise \ref{exr:8.9}, $|a_n-p|<\epsilon$, as desired.\par
        Now suppose that for all $\epsilon>0$, there is some $N\in\N$ such that for all $n\geq N$, we have $|a_n-p|<\epsilon$. To prove that $(a_n)$ converges to $p$, Definition \ref{dfn:15.2} tells us that it will suffice to show that for every open interval $I$ containing $p$, there exists $N\in\N$ such that if $n\geq N$, then $a_n\in I$. Let $I$ be an arbitrary open interval that satisfies $p\in I$. It follows by Lemma \ref{lem:8.10} that there exists a number $\epsilon>0$ such that $(p-\epsilon,p+\epsilon)\subset I$. With respect to this $\epsilon$, we have by hypothesis that there is some $N\in\N$ such that for all $n\geq N$, we have $|a_n-p|<\epsilon$. Choose this $N$ to be our $N$. Let $n$ be an arbitrary natural number such that $n\geq N$. Then $|a_n-p|<\epsilon$. Consequently, by Exercise \ref{exr:8.9}, $a_n\in(p-\epsilon,p+\epsilon)$. Therefore, by Definition \ref{dfn:1.3}, $a_n\in I$, as desired.
    \end{proof}
\end{theorem}

\begin{exercise}\label{exr:15.8}\leavevmode
    \begin{enumerate}[label={(\alph*)},ref={\theexercise\alph*}]
        \item \label{exr:15.8a}Prove that $\lim_{n\to\infty}\frac{(-1)^n}{n}=0$.
        \begin{proof}
            To prove that $\lim_{n\to\infty}\frac{(-1)^n}{n}=0$, Definition \ref{dfn:15.5} and Theorem \ref{trm:15.7} tell us that it will suffice to show that for all $\epsilon>0$, there is some $N\in\N$ such that for all $n\geq N$, we have $|a_n-0|=|a_n|<\epsilon$. Let $\epsilon>0$ be arbitrary. By Corollary \ref{cly:6.12}, there exists a natural number $N$ such that $\frac{1}{\epsilon}<N$. Choose this $N$ to be our $N$. Let $n$ be an arbitrary natural number such that $n\geq N$. It follows by transitivity that $\frac{1}{\epsilon}<n$. Thus, since $0<n$ and $0<\epsilon$, we have by consecutive applications of Lemma \ref{lem:7.24} that $0<\frac{1}{n}<\epsilon$. Additionally, since $(-1)^n=1$ or $(-1)^n=-1$ for all $n\in\N$ by Script \ref{sct:0}, we have by Definition \ref{dfn:8.4} that $|\frac{(-1)^n}{n}|=|\frac{1}{n}|=\frac{1}{n}$. Consequently, we know that $|\frac{(-1)^n}{n}|<\epsilon$. But since $a_n=\frac{(-1)^n}{n}$, we have that $|a_n|<\epsilon$, as desired.
        \end{proof}
        \item \label{exr:15.8b}Let $x\in\R$ with $|x|<1$. Prove that $\lim_{n\to\infty}x^n=0$.
        \begin{lemma*}
            If $|y|>1$ and $n$ is a natural number, then $|y|^n\geq n(|y|-1)+1$.
            \begin{proof}
                Define $1+x=|y|$. It follows by Definition \ref{dfn:7.21} that $x>0>-1$, which can be weakened to $x\geq -1$. Additionally, since $n$ is a natural number, $n\geq 1$ by Script \ref{sct:0}. Thus, since $x\geq -1$ and $n\geq 1$, we have by Additional Exercise \ref{axr:12.3b} that $(1+x)^n\geq 1+nx$. Substituting, we have $|y|^n\geq n(|y|-1)+1$, as desired.
            \end{proof}
        \end{lemma*}
        \begin{proof}[Proof of Exercise \ref{exr:15.8b}]
            To prove that $\lim_{n\to\infty}x^n=0$, Definition \ref{dfn:15.5} and Theorem \ref{trm:15.7} tell us that it will suffice to show that for all $\epsilon>0$, there is some $N\in\N$ such that for all $n\geq N$, we have $|a_n-0|=|a_n|<\epsilon$. Let $\epsilon>0$ be arbitrary. By Corollary \ref{cly:6.12}, there exists a natural number $N$ such that $\frac{1}{\epsilon(|\frac{1}{x}|-1)}<N$. Let $n$ be an arbitrary natural number such that $n\geq N$. It follows by Script \ref{sct:7}, the lemma, and the fact that $\frac{1}{|x|}>1$ (since $1>|x|$) that
            \begingroup
            \allowdisplaybreaks
            \begin{align*}
                |x^n| &= |x^n|\cdot\frac{1}{\epsilon\left( \left| \frac{1}{x} \right|-1 \right)}\cdot\epsilon\left( \left| \tfrac{1}{x} \right|-1 \right)\\
                &< |x^n|\cdot n\cdot\epsilon\left( \left| \tfrac{1}{x} \right|-1 \right)\\
                &< \epsilon\cdot|x^n|\cdot n\left( \left| \tfrac{1}{x} \right|-1 \right)+1\\
                &\leq \epsilon\cdot|x^n|\cdot\left| \tfrac{1}{x} \right|^n\\
                &= \epsilon\cdot|x^n|\cdot\frac{1}{|x^n|}\\
                &= \epsilon
            \end{align*}
            \endgroup
            as desired.
        \end{proof}
    \end{enumerate}
\end{exercise}

\begin{theorem}\label{trm:15.9}
    If $\lim_{n\to\infty}a_n=a$ and $\lim_{n\to\infty}b_n=b$ both exist, then\footnote{Note that these proofs are entirely symmetric to those in Theorem \ref{trm:11.9}.}
    \begin{enumerate}[label={\textup{(}\alph*\textup{)}}]
        \item $\lim_{n\to\infty}(a_n+b_n)=\lim_{n\to\infty}a_n+\lim_{n\to\infty}b_n$.
        \begin{proof}
            Let $p=\lim_{n\to\infty}a_n$ and let $q=\lim_{n\to\infty}b_n$. To prove that $\lim_{n\to\infty}(a_n+b_n)$ exists and equals $\lim_{n\to\infty}a_n+\lim_{n\to\infty}b_n$, Definition \ref{dfn:15.5} and Theorem \ref{trm:15.7} tell us that it will suffice to show that for all $\epsilon>0$, there is some $N\in\N$ such that for all $n\geq N$, we have $|a_n+b_n-(p+q)|<\epsilon$. Let $\epsilon>0$ be arbitrary. It follows by consecutive applications of Definition \ref{dfn:15.5} and Theorem \ref{trm:15.7} that there exist natural numbers $N_a,N_b$ such that for all $n\geq N_a$, we have $|a_n-p|<\frac{\epsilon}{2}$ and for all $n\geq N_b$, we have $|b_n-q|<\frac{\epsilon}{2}$. Now choose $N=\max(N_a,N_b)$. Let $n$ be an arbitrary natural number such that $n\geq N$. It follows that $n\geq N\geq N_a$, so we know that $|a_n-p|<\frac{\epsilon}{2}$. Similarly, $|b_n-q|<\frac{\epsilon}{2}$. Therefore, we have that
            \begin{align*}
                |a_n+b_n-(p+q)| &\leq |a_n-p|+|b_n-q|\tag*{Lemma \ref{lem:8.8}}\\
                &< \frac{\epsilon}{2}+\frac{\epsilon}{2}\\
                &= \epsilon
            \end{align*}
        \end{proof}
        \item $\lim_{n\to\infty}(a_n\cdot b_n)=\left( \lim_{n\to\infty}a_n \right)\cdot\left( \lim_{n\to\infty}b_n \right)$.
        \begin{proof}
            Let $p=\lim_{n\to\infty}a_n$ and let $q=\lim_{n\to\infty}b_n$. To prove that $\lim_{n\to\infty}(a_n\cdot b_n)$ exists and equals $\left( \lim_{n\to\infty}a_n \right)\cdot\left( \lim_{n\to\infty}b_n \right)$, Definition \ref{dfn:15.5} and Theorem \ref{trm:15.7} tell us that it will suffice too show that for all $\epsilon>0$, there is some $N\in\N$ such that for all $n\geq N$, we have $|a_n\cdot b_n-p\cdot q|<\epsilon$. Let $\epsilon>0$ be arbitrary. It follows by consecutive applications of Definition \ref{dfn:15.5} and Theorem \ref{trm:15.7} that there exist natural numbers $N_a,N_b$ such that for all $n\geq N_a$, we have $|a_n-p|<\min(\frac{\epsilon}{2(|q|+1)},1)$ and for all $n\geq N_b$, we have $|b_n-q|<\frac{\epsilon}{2(|p|+1)}$. Now choose $N=\max(N_a,N_b)$. Let $n$ be an arbitrary natural number such that $n\geq N$. It follows that $n\geq N\geq N_a$, so we know that $|a_n-p|<\min(\frac{\epsilon}{2(|q|+1)},1)$. Similarly, $|b_n-q|<\frac{\epsilon}{2(|p|+1)}$. As a last note before we launch into the main inequality, observe that $|a_n|-|p|\leq |a_n-p|<\min(\frac{\epsilon}{2(|p|+1)},1)\leq 1$, i.e., that $|a_n|<1+|p|$. Therefore, we have that
            \begin{align*}
                |a_n\cdot b_n-p\cdot q| &= |a_n(b_n-q)+q(a_n-p)|\\
                &\leq |a_n|\cdot|b_n-q|+|q|\cdot|a_n-p|\tag*{Lemma \ref{lem:8.8}}\\
                &< (1+|p|)\cdot\frac{\epsilon}{2(|p|+1)}+|q|\cdot\frac{\epsilon}{2(|q|+1)}\\
                &= \frac{\epsilon}{2}\cdot\frac{1+|p|}{1+|p|}+\frac{\epsilon}{2}\cdot\frac{|q|}{|q|+1}\\
                &< \frac{\epsilon}{2}+\frac{\epsilon}{2}\\
                &= \epsilon
            \end{align*}
        \end{proof}
    \end{enumerate}
    Moreover, if $\lim_{n\to\infty}b_n\neq 0$, then
    \begin{enumerate}[resume,label={\textup{(}\alph*\textup{)}}]
        \item $\lim_{n\to\infty}\frac{a_n}{b_n}=\frac{\lim_{n\to\infty}a_n}{\lim_{n\to\infty}b_n}$.
        \begin{proof}
            Let $p=\lim_{n\to\infty}a_n$ and let $q=\lim_{n\to\infty}b_n$. To prove that $\lim_{n\to\infty}\frac{a_n}{b_n}$ exists and equals $\frac{\lim_{n\to\infty}a_n}{\lim_{n\to\infty}b_n}$, Definition \ref{dfn:15.5} and Theorem \ref{trm:15.7} tell us that it will suffice to show that for all $\epsilon>0$, there is some $N\in\N$ such that for all $n\geq N$, we have $|\frac{a_n}{b_n}-\frac{p}{q}|<\epsilon$. Let $\epsilon>0$ be arbitrary. It follows by consecutive applications of Definition \ref{dfn:15.5} and Theorem \ref{trm:15.7} that there exist natural numbers $N_a,N_b$ such that for all $n\geq N_a$, we have $|a_n-p|<\frac{\epsilon|q|}{4}$ and for all $n\geq N_b$, we have $|b_n-q|<\min(\frac{|q|}{2},\frac{\epsilon|q|^2}{4|p|})$. Now choose $N=\max(N_a,N_b)$. Let $n$ be an arbitrary natural number such that $n\geq N$. It follows that $n\geq N\geq N_a$, so we know that $|a_n-p|<\frac{\epsilon|q|}{4}$. Similarly, $|b_n-q|<\min(\frac{|q|}{2},\frac{\epsilon|q|^2}{4|p|})$.\par
            Before we get into the body of the proof, we need a preliminary result: it follows from the fact that $|b_n-q|<\frac{|q|}{2}$ that
            \begin{align*}
                |q| &= 2|q|-|q|\\
                &= 2(|q|-|b_n|+|b_n|)-|q|\\
                &\leq 2(|q-b_n|+|b_n|)-|q|\\
                &= 2(|b_n-q|+|b_n|)-|q|\\
                &< 2\left( \frac{|q|}{2}+|b_n| \right)-|q|\\
                &= |q|+2|b_n|-|q|\\
                &= 2|b_n|
            \end{align*}
            With this result, we are ready to introduce the main inequality:
            \begin{align*}
                \left| \frac{a_n}{b_n}-\frac{p}{q} \right| &= \left| \frac{a_nq-b_np}{b_nq} \right|\\
                &= \frac{|q(a_n-p)+p(q-b_n)|}{|b_n|\cdot|q|}\\
                &\leq \frac{|q|\cdot|a_n-p|+|p|\cdot|q-b_n|}{|b_n|\cdot|q|}\\
                &< \frac{|q|\cdot\frac{\epsilon|q|}{4}+|p|\cdot\frac{\epsilon|q|^2}{4|p|}}{|b_n|\cdot|q|}\\
                &= \frac{\epsilon|q|}{2|b_n|}\\
                &< \frac{\epsilon|q|}{|q|}\\
                &= \epsilon
            \end{align*}
        \end{proof}
    \end{enumerate}
\end{theorem}




\end{document}