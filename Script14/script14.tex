\documentclass[../main.tex]{subfiles}

\pagestyle{main}
\renewcommand{\chaptermark}[1]{\markboth{\chaptername\ \thechapter}{}}
\setcounter{chapter}{13}

\begin{document}




\chapter{Integrals and Derivatives}\label{sct:14}
\section{Journal}
\begin{theorem}\marginnote{\emph{5/4:}}
    Suppose that $f$ is integrable on $[a,b]$. Define $F:[a,b]\to\R$ by
    \begin{equation*}
        F(x) = \int_a^xf
    \end{equation*}
    If $f$ is continuous at $p\in(a,b)$, then $F$ is differentiable at $p$ and
    \begin{equation*}
        F'(p) = f(p)
    \end{equation*}
    If $f$ is continuous at $a$, then $F_+'(a)$ exists and equals $f(a)$. Similarly, if $f$ is continuous at $b$, $F_-'(b)$ exists and equals $f(b)$.
    \begin{proof}
        To prove that $F$ is differentiable at $p$ and $F'(p)=f(p)$, Definition \ref{dfn:12.1} tells us that it will suffice to show that $\lim_{h\to 0^+}\frac{F(p+h)-F(p)}{h}=\lim_{h\to 0^-}\frac{F(p+h)-F(p)}{h}=f(p)$. We will tackle the right-handed limit first. To do so, Definition \ref{dfn:11.1} tells us that it will suffice to verify that for every $\epsilon>0$, there exists a $\delta>0$ such that if $(p+h)\in[a,b]$ and $0<h<\delta$, then $|\frac{F(p+h)-F(p)}{h}-f(p)|<\epsilon$. Let $\epsilon>0$ be arbitrary. Since $f$ is continuous at $p$, Theorem \ref{trm:11.5} asserts that there exists a $\delta>0$ such that if $x\in[a,b]$ and $|x-p|<\delta$, then $|f(x)-f(p)|<\frac{\epsilon}{2}$. Choose this $\delta$ to be our $\delta$. Let $h$ be an arbitrary number satisfying $(p+h)\in[a,b]$ and $0<h<\delta$. Therefore,
        \begingroup
        \allowdisplaybreaks
        \begin{align*}
            \left| \frac{F(p+h)-F(p)}{h}-f(p) \right| &= \left| \frac{\int_a^{p+h}f-\int_a^pf}{h}-f(p) \right|\\
            &= \left| \frac{\int_p^{p+h}f}{h}-f(p) \right|\tag*{Theorem \ref{trm:13.23}}\\
            &= \left| \frac{\int_p^{p+h}f-hf(p)}{h} \right|\\
            &= \left| \frac{\int_p^{p+h}f-f(p)((p+h)-p)}{h} \right|\\
            &= \left| \frac{\int_p^{p+h}f-\int_p^{p+h}f(p)\dd{x}}{h} \right|\tag*{Exercise \ref{exr:13.17}}\\
            &= \left| \frac{1}{h}\int_p^{p+h}(f(x)-f(p))\dd{x} \right|\tag*{Theorem \ref{trm:13.24}}\\
            &\leq \left| \frac{1}{h} \right|\int_p^{p+h}|f(x)-f(p)|\dd{x}\tag*{Theorem \ref{trm:13.26}}\\
            &\leq \left| \frac{1}{h} \right|\frac{\epsilon}{2}((p+h)-p)\tag*{Theorem \ref{trm:13.27}}\\
            &= \frac{\epsilon}{2}\\
            &< \epsilon
        \end{align*}
        \endgroup
        The proof is symmetric for the left-handed limit. These proofs can also be applied to the endpoints.
    \end{proof}
\end{theorem}

\begin{remark}\label{rmk:14.2}
    Thus, we have that if $f$ is continuous on $[a,b]$, $F$ is differentiable on $[a,b]$ and $F'(p)=f(p)$ for all $p\in[a,b]$ (where at the endpoints, we understand that the derivative should be interpreted as the one-sided derivative).
\end{remark}

\begin{lemma}\label{lem:14.3}
    Suppose that $f:[a,b]\to\R$ is integrable and that $\Omega$ is a number satisfying $L(f,P)\leq\Omega\leq U(f,P)$ for all partitions $P$ of $[a,b]$. Then
    \begin{equation*}
        \int_a^bf = \Omega
    \end{equation*}
    \begin{proof}
        Suppose for the sake of contradiction that $\int_a^bf\neq\Omega$. We divide into two cases ($\int_a^bf<\Omega$ and $\int_a^bf>\Omega$). If $\int_a^bf<\Omega$, then by Definition \ref{dfn:13.16}, $U(f)=\int_a^bf<\Omega$. It follows by Definition \ref{dfn:13.14} and \ref{lem:5.11} that there exists an object $U(f,P)\in\{U(f,P)\mid P\text{ is a partition of }[a,b]\}$ such that $U(f)\leq U(f,P)<\Omega$. But this contradicts the hypothesis that $U(f,P)\geq\Omega$ for all partitions $P$ of $[a,b]$. The argument is symmetric in the other case.
    \end{proof}
\end{lemma}




\end{document}