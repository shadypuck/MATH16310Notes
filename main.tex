\documentclass{report}

\usepackage[margin=1in]{geometry}
\usepackage{fancyhdr}
\usepackage{csquotes}
\usepackage{xr}
\usepackage{marginnote}
\usepackage{scrextend}
\usepackage[bottom]{footmisc}
\usepackage{enumitem}
\usepackage{amsmath,amssymb,amsthm}
\usepackage{bm,scalerel}
\usepackage{physics}
\usepackage[hidelinks]{hyperref}

\fancypagestyle{main}{
    \fancyhf{}
    \fancyfoot[R]{Labalme\ \thepage}
    \fancyhead[R]{MATH\ 16210}
    \fancyhead[L]{\leftmark}
}
\fancypagestyle{plain}{
    \fancyhead{}
    \renewcommand{\headrulewidth}{0pt}
}

\MakeOuterQuote{"}

\externaldocument{main}
\externaldocument{../../../Misc./Sequence Combined Notes/Honors Calculus IBL/combined}

\reversemarginpar

\deffootnotemark{\textsuperscript{\textup{[}\thefootnotemark\textup{]}}}
\deffootnote[2.1em]{0em}{0em}{\textsuperscript{\thefootnote}}

\setitemize[3]{label=\scriptsize$\blacksquare$}

\newtheorem{theorem}{Theorem}[chapter]
\newtheorem{lemma}[theorem]{Lemma}
\newtheorem{corollary}[theorem]{Corollary}
\newtheorem*{lemma*}{Lemma}
\theoremstyle{definition}
\newtheorem{definition}[theorem]{Definition}
\newtheorem{exercise}[theorem]{Exercise}
\newtheorem{remark}[theorem]{Remark}

\renewcommand{\chaptername}{Script}

\newcommand{\N}{\mathbb{N}}
\newcommand{\Z}{\mathbb{Z}}
\newcommand{\Q}{\mathbb{Q}}
\newcommand{\R}{\mathbb{R}}

\newcommand{\x}{\mathbf{x}}
\newcommand{\y}{\mathbf{y}}
\newcommand{\z}{\mathbf{z}}

\newlength\bshft
\bshft=.08pt\relax
\newcommand{\fakeboldb}[1]{
    \ThisStyle{\ooalign{$\SavedStyle#1$\cr%
    \kern-\bshft$\SavedStyle#1$\cr%
    \kern\bshft$\SavedStyle#1$}}
}
\newlength\cshft
\cshft=.38pt\relax
\newcommand{\fakeboldc}[1]{
    \ThisStyle{\ooalign{$\SavedStyle#1$\cr%
    \kern-\cshft$\SavedStyle#1$\cr%
    \kern\cshft$\SavedStyle#1$}}
}

\usepackage{subfiles}

\title{MATH 16310 (Honors Calculus III IBL) Notes}
\author{Steven Labalme}

\begin{document}




\maketitle



\pagenumbering{roman}
\tableofcontents
\newpage



\pagenumbering{arabic}
\pagestyle{main}
\renewcommand{\chaptermark}[1]{\markboth{\chaptername\ \thechapter}{}}
\setcounter{chapter}{11}
\subfile{Script12/script12.tex}


\section{Discussion}
\begin{itemize}
    \item \marginnote{3/30:}We can also do Exercise \ref{exr:12.6} with left- and right-handed limits, as defined in Additional Exercise 11.2.
    \item We can also do Exercise \ref{exr:12.7} by induction.
    \item \marginnote{4/1:}We can also do Theorem \ref{trm:12.13} by noting that of the left- and right-hand derivatives, one will be $\leq 0$ and the other $\geq 0$, but since they must be equal, they must equal 0.
    \item Include more rigorous restriction bits for Corollary \ref{cly:12.14} as a lemma?
    \item \marginnote{4/6:}Modify Theorem \ref{trm:12.13} with Lemma \ref{lem:11.8}.
    \item Redo Corollary \ref{cly:12.17}c as a direct proof?
    \item We don't have to be too rigorous with the restriction of $f$ in Corollary \ref{cly:12.17}. In fact, we need not even mention it.
    \item Include a bit more of the basic algebra proving that $h(a)=h(b)=0$ for Corollary \ref{cly:12.20}.
    \item Potentially for Theorem \ref{trm:12.21}, we can use $\varphi(y)$ but modified with $f^{-1}$ to prove the iffy limit transition.
    \begin{itemize}
        \item Potentially we can apply the chain rule for this proof?
        \item Could we use the fact that $f$ is continuous to prove that as $x\to p$, $f(x)\to f(p)$?
    \end{itemize}
\end{itemize}



\subfile{Script13/script13.tex}


\section{Discussion}
\begin{itemize}
    \item \marginnote{4/8:}The key to $\epsilon$-$\delta$ proofs is to find a way to get $|y-x|$ into the $|f(y)-f(x)|$ expression and then deal with the others.
    \item \marginnote{4/13:}Note that we can also prove Exercise \ref{exr:13.7} with the following procedure:
    \begin{itemize}
        \item Lemma: If $f$ is uniformly continuous on two intervals $I,J$ whose union $I\cup J$ is also an interval, then $f$ is uniformly continuous on $I\cup J$.
        \item Establish that $f(x)=\sqrt{x}$ is uniformly continuous on $[0,1]$ using Theorem \ref{trm:13.6}.
        \begin{itemize}
            \item Note that the continuity of $f$ on $[0,1]$ follows from the fact that $f$ is differentiable on $(0,1)\subset\R^+$ (Exercise \ref{exr:12.22}) by Theorems \ref{trm:12.5} and \ref{trm:9.10}.
            \item Note that the compactness of $[0,1]$ follows from Theorem \ref{trm:10.14}.
        \end{itemize}
        \item Recall that $f$ is uniformly continuous on $[1,\infty)$ from Exercise \ref{exr:13.3}.
        \item Apply the lemma.
    \end{itemize}
    \item \marginnote{4/15:}We can just say that the supremum of a singleton set is the element in that set (no proof required).
    \item \marginnote{4/22:}Theorem \ref{trm:13.24} was originally presented with a Theorem \ref{trm:13.18} $\epsilon$ proof preceding the Lemma \ref{lem:13.20} $\Omega$ proof. However, this is unnecessary.
    \begin{itemize}
        \item Lemma (a) can also be proven by contradiction.
    \end{itemize}
    \item \marginnote{5/4:}For Theorem \ref{trm:13.28}, I should more rigorously establish the existence of $s$, perhaps with Theorem \ref{trm:13.26}.
\end{itemize}



\subfile{Script14/script14.tex}


\section{Discussion}
\begin{itemize}
    \item We can't use Exercise \ref{exr:10.21} because it requires that $f$ be continuous on the whole interval $[p-\delta,p+\delta]$.
    \begin{itemize}
        \item For my proof: If all of the values on the interval are within $\frac{\epsilon}{2}$, then the whole thing is within $\epsilon$?
    \end{itemize}
\end{itemize}




\end{document}