\documentclass{report}

\usepackage[margin=1in]{geometry}
\usepackage{fancyhdr}
\usepackage{xr}
\usepackage{marginnote}
\usepackage[hidelinks]{hyperref}

\fancypagestyle{main}{
    \fancyhf{}
    \fancyfoot[R]{Labalme\ \thepage}
    \fancyhead[R]{MATH\ 16210}
    \fancyhead[L]{\leftmark}
}
\fancypagestyle{plain}{
    \fancyhead{}
    \renewcommand{\headrulewidth}{0pt}
}

\externaldocument{main}
\externaldocument{../../../Misc./Sequence Combined Notes/Honors Calculus IBL/combined}

\reversemarginpar

\renewcommand{\chaptername}{Script}

\usepackage{subfiles}

\title{MATH 16310 (Honors Calculus III IBL) Notes}
\author{Steven Labalme}

\begin{document}




\maketitle



\pagenumbering{roman}
\tableofcontents
\newpage



\pagenumbering{arabic}
\pagestyle{main}
\renewcommand{\chaptermark}[1]{\markboth{\chaptername\ \thechapter}{}}
\setcounter{chapter}{11}
\subfile{Script12/script12.tex}


\section{Discussion}
\begin{itemize}
    \item \marginnote{3/30:}We can also do Exercise \ref{exr:12.6} with left- and right-handed limits, as defined in Additional Exercise 11.2.
    \item We can also do Exercise \ref{exr:12.7} by induction.
    \item \marginnote{4/1:}We can also do Theorem \ref{trm:12.13} by noting that of the left- and right-hand derivatives, one will be $\leq 0$ and the other $\geq 0$, but since they must be equal, they must equal 0.
    \item Include more rigorous restriction bits for Corollary \ref{cly:12.14} as a lemma?
    \item \marginnote{4/6:}Modify Theorem \ref{trm:12.13} with Lemma \ref{lem:11.8}.
    \item Redo Corollary \ref{cly:12.17}c as a direct proof?
    \item We don't have to be too rigorous with the restriction of $f$ in Corollary \ref{cly:12.17}. In fact, we need not even mention it.
    \item Include a bit more of the basic algebra proving that $h(a)=h(b)=0$ for Corollary \ref{cly:12.20}.
    \item Potentially for Theorem \ref{trm:12.21}, we can use $\varphi(y)$ but modified with $f^{-1}$ to prove the iffy limit transition.
    \begin{itemize}
        \item Potentially we can apply the chain rule for this proof?
        \item Could we use the fact that $f$ is continuous to prove that as $x\to p$, $f(x)\to f(p)$?
    \end{itemize}
\end{itemize}



\subfile{Script13/script13.tex}


\section{Discussion}
\begin{itemize}
    \item The key to $\epsilon$-$\delta$ proofs is to find a way to get $|y-x|$ into the $|f(y)-f(x)|$ expression and then deal with the others.
\end{itemize}




\end{document}