\documentclass[../main.tex]{subfiles}

\pagestyle{main}
\renewcommand{\chaptermark}[1]{\markboth{\chaptername\ \thechapter}{}}
\setcounter{chapter}{14}

\begin{document}




\chapter{Sequences}\label{sct:15}
\section{Journal}
\begin{definition}\label{dfn:15.1}\marginnote{5/6:}
    A \textbf{sequence} (of real numbers) is a function $a:\N\to\R$.\par
    By setting $a_n=a(n)$, we can think of a sequence as a list $a_1,a_2,a_3,\dots$ of real numbers. We use the notation $(a_n)_{n=1}^\infty$ for such a sequence, or if there is no possibility of confusion, we sometimes abbreviate this and write simply $(a_n)$. More generally, we also use the term sequence to refer to the function defined on $\{n\in\N\cup\{0\}\mid n\geq n_0\}$ for any fixed $n_0\in\N\cup\{0\}$. We write $(a_n)_{n=n_0}^\infty$ for such a sequence.
\end{definition}

\begin{definition}\label{dfn:15.2}
    We say that a sequence $(a_n)$ \textbf{converges} to a point $p\in\R$ if for every open interval $I$ containing $p$, there exists $N\in\N$ such that if $n\geq N$, then $a_n\in I$. If a sequence converges to some point, we say it is \textbf{convergent}. If $(a_n)$ does not converge to any point, we say that the sequence \textbf{diverges} or is \textbf{divergent}.
\end{definition}

\begin{exercise}\label{exr:15.3}
    Show that a sequence $(a_n)$ converges to $p$ if and only if any region containing $p$ contains all but finitely many terms of the sequence.
    \begin{proof}
        Suppose first that $(a_n)$ converges to $p$. Let $R$ be an arbitrary region containing $p$. By Corollary \ref{cly:4.11} and Lemma \ref{lem:8.3}, $R$ is an open interval. Thus, by Definition \ref{dfn:15.2}, there exists $N\in\N$ such that if $n\geq N$, then $a_n\in R$. To prove that $R$ contains all but finitely many terms of the sequence, it will suffice to show that the set $A=\{a_n\mid a_n\notin R\}$ is finite. Since $a_n\in R$ for all $n\geq N$, it follows that $a_n\in R$ only if $n<N$. Thus, by Script \ref{sct:1}, $A\subset\{a_n\mid 0\leq n<N\}$. Since the latter set is clearly finite, it follows by Script \ref{sct:1} that $A$ is finite.\par
        Now suppose that any region containing $p$ contains all but finitely many terms $(a_n)$. To prove that $(a_n)$ converges to $p$, Definition \ref{dfn:15.2} tells us that it will suffice to show that for every open interval $I$ containing $p$, there exists $N\in\N$ such that if $n\geq N$, then $a_n\in I$. Let $I$ be an arbitrary open interval containing $p$. Then by Theorem \ref{trm:4.10}, there exists a region $R$ containing $p$ such that $R\subset I$. It follows by the hypothesis that $A=\{a_n\mid a_n\notin R\}$ is finite. We divide into two cases ($|A|=0$ and $|A|\in\N$). Suppose first that $|A|=0$. Choose $N=n_0$. It follows that if $n\geq N$, then $a_n\notin A$, so $a_n\in R$, so $a_n\in I$, as desired. Now suppose that $|A|\in\N$. By Definition \ref{dfn:1.18}, $a^{-1}(A)\subset\N$. Consequently, by Lemma \ref{lem:3.4}, $a^{-1}(A)$ has a last point $N-1$. Choose $N=(N-1)+1$. It follows that if $n\geq N$, then $n\notin a^{-1}(A)$, so $a_n\notin A$, so $a_n\in R$, so $a_n\in I$, as desired.
    \end{proof}
\end{exercise}

\begin{theorem}\label{trm:15.4}
    Suppose that $(a_n)$ converges to both $p$ and to $p'$. Then $p=p'$.
    \begin{proof}
        Suppose for the sake of contradiction that $p\neq p'$. Then by Theorem \ref{trm:3.22}, there exist disjoint regions $R,R'$ containing $p,p'$, respectively. Additionally, by consecutive applications of Corollary \ref{cly:4.11} and Lemma \ref{lem:8.3}, $R,R'$ are open intervals. Combining these last two results, we have by consecutive applications of Definition \ref{dfn:15.2} that there exist $N,N'\in\N$ such that if $n\geq N$, then $a_n\in R$ and if $n\geq N'$, then $a_n\in R'$. Let $M=\max(N,N')$. It follows that $M\geq N$ and $M\geq N'$. Thus, by the above, $a_M\in R$ and $a_M\in R'$. But this implies by Definition \ref{dfn:1.6} that $a_M\in R\cap R'$. Therefore, by Definition \ref{dfn:1.9}, $R$ and $R'$ are not disjoint, a contradiction.
    \end{proof}
\end{theorem}

\begin{definition}\label{dfn:15.5}
    If a sequence $(a_n)$ converges to $p\in\R$, we call $p$ the \textbf{limit} of $(a_n)$ and write
    \begin{equation*}
        \lim_{n\to\infty}a_n = p
    \end{equation*}
\end{definition}

\begin{exercise}\label{exr:15.6}
    Which of the following sequences $(a_n)$ converge? Which diverge? For each that converges, what is the limit? Give proofs for your answers.
    \begin{enumerate}[label={(\alph*)},ref={\theexercise\alph*}]
        \item \label{exr:15.6a}$a_n=5$.
        \begin{proof}
            To prove that this sequence converges with limit $\lim_{n\to\infty}a_n=5$, Definition \ref{dfn:15.5} tells us that it will suffice to show that $(a_n)$ converges to 5. To do so, Definition \ref{dfn:15.2} tells us that it will suffice to verify that for every open interval $I$ containing 5, there exists $N\in\N$ such that if $n\geq N$, then $a_n\in I$. Let's begin.\par
            Let $I$ be an arbitrary open interval containing 5. Choose $N=1$. Let $n$ be an arbitrary natural number such that $n\geq N$. It follows by the definition of the sequence that $a_n=5\in I$, as desired.
        \end{proof}
        \item \label{exr:15.6b}$a_n=n$.
        \begin{proof}
            To prove that this sequence diverges, Definition \ref{dfn:15.2} tells us that it will suffice to show that for every point $p\in\R$, there exists an open interval $I$ containing $p$ such that for all $N\in\N$, there exists an $n\geq N$ such that $a_n\notin I$. Let's begin.\par
            Let $p$ be an arbitrary element of $\R$. Choose $I=(p-1,p+1)$. Clearly $p\in I$. Let $N$ be an arbitrary natural number. By Corollary \ref{cly:6.12}, there exists a natural number $N'$ such that $p+1<N'$. Choose $M=\max(N,N')$. Thus, $M\geq N$. Additionally, it follows by the definition of the sequence that $a_M=M$. But this implies that $a_M\geq N'>p+1$, i.e., $a_M\notin I$ by Equations \ref{eqn:8.1}.
        \end{proof}
        \item \label{exr:15.6c}$a_n=\frac{1}{n}$.
        \begin{proof}
            To prove that this sequence converges with limit $\lim_{n\to\infty}a_n=0$, Definitions \ref{dfn:15.5} and \ref{dfn:15.2} tell us that it will suffice to show that for every open interval $I$ containing 0, there exists $N\in\N$ such that if $n\geq N$, then $a_n\in I$. Let's begin.\par
            Let $I$ be an arbitrary interval containing 0. By Lemma \ref{lem:8.10}, there exists a region $(a,b)$ containing 0 such that $(a,b)\subset I$. By Corollary \ref{cly:6.12}, there exists a natural number $N$ such that $\frac{1}{b}<N$. Choose this $N$ to be our $N$. Now let $n$ be an arbitrary natural number such that $n\geq N$. It follows that $\frac{1}{b}<n$. Thus, since $0<b$ and $0<n$, we have by consecutive applications of Lemma \ref{lem:7.24} that $0<\frac{1}{n}<b$. Consequently, since we also know that $a<0$ and $a_n=\frac{1}{n}$, we have by transitivity and substitution that $a<a_n<b$. It follows by Equations \ref{eqn:8.1} that $a_n\in(a,b)$. Therefore, by Definition \ref{dfn:1.3}, $a_n\in I$, as desired.
        \end{proof}
        \item \label{exr:15.6d}$a_n=(-1)^n$.
        \begin{proof}
            To prove that this sequence diverges, Definition \ref{dfn:15.2} tells us that it will suffice to show that for every point $p\in\R$, there exists an open interval $I$ containing $p$ such that for all $N\in\N$, there exists an $n\geq N$ such that $a_n\notin I$. Let's begin.\par
            Let $p$ be an arbitrary element of $R$. Choose $I=(p-1,p+1)$. Clearly $p\in I$. Let $N$ be an arbitrary natural number. By Script \ref{sct:0}, either $N$ is even and $N+1$ is odd or vice versa. Thus, let $N$ be even (the case where $N$ is odd is symmetric). It follows that $N\geq N$ yields $a_N=(-1)^{2m}=((-1)^2)^m=1^m=1$ and that $N+1\geq N$ yields $a_{N+1}=(-1)^{2m+1}=1(-1)^1=-1$. Now suppose for the sake of contradiction that $a_N\in I$ and $a_{N+1}\in I$. Since $a_N=1\in I$, we have by Equations \ref{eqn:8.1} that $p-1<1<p+1$. It follows by Definition \ref{dfn:7.21} that $p-3<-1<p-1$. But $-1<p-1$ implies by Equations \ref{eqn:8.1} that $a_{N+1}=-1\notin I$, a contradiction. Therefore, $N+1\geq N$ is a number such that $a_{N+1}\notin I$, as desired.
        \end{proof}
    \end{enumerate}
\end{exercise}

\begin{theorem}\label{trm:15.7}\marginnote{\emph{5/11:}}
    A sequence $(a_n)$ converges to $p\in\R$ if and only if for all $\epsilon>0$, there is some $N\in\N$ such that for all $n\geq N$, we have $|a_n-p|<\epsilon$.
    \begin{proof}
        Suppose first that $(a_n)$ converges to $p$. Let $\epsilon>0$ be arbitrary. Consider the $p$-containing region $R=(p-\epsilon,p+\epsilon)$. By Corollary \ref{cly:4.11} and \ref{lem:8.3}, $R$ is an open interval. Thus, by Definition \ref{dfn:15.2}, there exists $N\in\N$ such that if $n\geq N$, then $a_n\in R$. Choose this $N$ to be our $N$. Let $n$ be an arbitrary natural number such that $n\geq N$. Then $a_n\in(p-\epsilon,p+\epsilon)$. Therefore, by Exercise \ref{exr:8.9}, $|a_n-p|<\epsilon$, as desired.\par
        Now suppose that for all $\epsilon>0$, there is some $N\in\N$ such that for all $n\geq N$, we have $|a_n-p|<\epsilon$. To prove that $(a_n)$ converges to $p$, Definition \ref{dfn:15.2} tells us that it will suffice to show that for every open interval $I$ containing $p$, there exists $N\in\N$ such that if $n\geq N$, then $a_n\in I$. Let $I$ be an arbitrary open interval that satisfies $p\in I$. It follows by Lemma \ref{lem:8.10} that there exists a number $\epsilon>0$ such that $(p-\epsilon,p+\epsilon)\subset I$. With respect to this $\epsilon$, we have by hypothesis that there is some $N\in\N$ such that for all $n\geq N$, we have $|a_n-p|<\epsilon$. Choose this $N$ to be our $N$. Let $n$ be an arbitrary natural number such that $n\geq N$. Then $|a_n-p|<\epsilon$. Consequently, by Exercise \ref{exr:8.9}, $a_n\in(p-\epsilon,p+\epsilon)$. Therefore, by Definition \ref{dfn:1.3}, $a_n\in I$, as desired.
    \end{proof}
\end{theorem}

\begin{exercise}\label{exr:15.8}\leavevmode
    \begin{enumerate}[label={(\alph*)},ref={\theexercise\alph*}]
        \item \label{exr:15.8a}Prove that $\lim_{n\to\infty}\frac{(-1)^n}{n}=0$.
        \begin{proof}
            To prove that $\lim_{n\to\infty}\frac{(-1)^n}{n}=0$, Definition \ref{dfn:15.5} and Theorem \ref{trm:15.7} tell us that it will suffice to show that for all $\epsilon>0$, there is some $N\in\N$ such that for all $n\geq N$, we have $|a_n-0|=|a_n|<\epsilon$. Let $\epsilon>0$ be arbitrary. By Corollary \ref{cly:6.12}, there exists a natural number $N$ such that $\frac{1}{\epsilon}<N$. Choose this $N$ to be our $N$. Let $n$ be an arbitrary natural number such that $n\geq N$. It follows by transitivity that $\frac{1}{\epsilon}<n$. Thus, since $0<n$ and $0<\epsilon$, we have by consecutive applications of Lemma \ref{lem:7.24} that $0<\frac{1}{n}<\epsilon$. Additionally, since $(-1)^n=1$ or $(-1)^n=-1$ for all $n\in\N$ by Script \ref{sct:0}, we have by Definition \ref{dfn:8.4} that $|\frac{(-1)^n}{n}|=|\frac{1}{n}|=\frac{1}{n}$. Consequently, we know that $|\frac{(-1)^n}{n}|<\epsilon$. But since $a_n=\frac{(-1)^n}{n}$, we have that $|a_n|<\epsilon$, as desired.
        \end{proof}
        \item \label{exr:15.8b}Let $x\in\R$ with $|x|<1$. Prove that $\lim_{n\to\infty}x^n=0$.
        \begin{lemma*}
            If $|y|>1$ and $n$ is a natural number, then $|y|^n\geq n(|y|-1)+1$.
            \begin{proof}
                Define $1+x=|y|$. It follows by Definition \ref{dfn:7.21} that $x>0>-1$, which can be weakened to $x\geq -1$. Additionally, since $n$ is a natural number, $n\geq 1$ by Script \ref{sct:0}. Thus, since $x\geq -1$ and $n\geq 1$, we have by Additional Exercise \ref{axr:12.3b} that $(1+x)^n\geq 1+nx$. Substituting, we have $|y|^n\geq n(|y|-1)+1$, as desired.
            \end{proof}
        \end{lemma*}
        \begin{proof}[Proof of Exercise \ref{exr:15.8b}]
            To prove that $\lim_{n\to\infty}x^n=0$, Definition \ref{dfn:15.5} and Theorem \ref{trm:15.7} tell us that it will suffice to show that for all $\epsilon>0$, there is some $N\in\N$ such that for all $n\geq N$, we have $|a_n-0|=|a_n|<\epsilon$. Let $\epsilon>0$ be arbitrary. We divide into two cases ($x=0$ and $x\neq 0$). If $x=0$, then choose $N=1$. Let $n$ be an arbitrary natural number such that $n\geq N$. Since $0^n=0$ by Script \ref{sct:7}, we have $|a_n|=|0^n|=0<\epsilon$, as desired. On the other hand, if $x\neq 0$, then we continue. By Corollary \ref{cly:6.12}, there exists a natural number $N$ such that $\frac{1}{\epsilon(|\frac{1}{x}|-1)}<N$. Choose this $N$ to be our $N$. Let $n$ be an arbitrary natural number such that $n\geq N$. Therefore,
            \begingroup
            \allowdisplaybreaks
            \begin{align*}
                |x^n| &= |x^n|\cdot\frac{1}{\epsilon\left( \left| \frac{1}{x} \right|-1 \right)}\cdot\epsilon\left( \left| \frac{1}{x} \right|-1 \right)\\
                &< |x^n|\cdot N\cdot\epsilon\left( \left| \frac{1}{x} \right|-1 \right)\\
                &\leq |x^n|\cdot n\cdot\epsilon\left( \left| \frac{1}{x} \right|-1 \right)\\
                &< \epsilon\cdot|x^n|\cdot n\left( \left| \frac{1}{x} \right|-1 \right)+1\\
                &\leq \epsilon\cdot|x^n|\cdot\left| \frac{1}{x} \right|^n\tag*{The Lemma}\\
                &= \epsilon\cdot|x^n|\cdot\frac{1}{|x^n|}\\
                &= \epsilon
            \end{align*}
            \endgroup
            as desired.
        \end{proof}
    \end{enumerate}
\end{exercise}

\begin{theorem}\label{trm:15.9}
    If $\lim_{n\to\infty}a_n=a$ and $\lim_{n\to\infty}b_n=b$ both exist, then
    \begin{enumerate}[label={\textup{(}\alph*\textup{)}},ref={\thetheorem\alph*}]
        \item \label{trm:15.9a}$\lim_{n\to\infty}(a_n+b_n)=\lim_{n\to\infty}a_n+\lim_{n\to\infty}b_n$.
        \begin{proof}
            Let $a=\lim_{n\to\infty}a_n$ and let $b=\lim_{n\to\infty}b_n$. To prove that $\lim_{n\to\infty}(a_n+b_n)$ exists and equals $\lim_{n\to\infty}a_n+\lim_{n\to\infty}b_n$, Definition \ref{dfn:15.5} and Theorem \ref{trm:15.7} tell us that it will suffice to show that for all $\epsilon>0$, there is some $N\in\N$ such that for all $n\geq N$, we have $|(a_n+b_n)-(a+b)|<\epsilon$. Let $\epsilon>0$ be arbitrary. It follows by consecutive applications of Definition \ref{dfn:15.5} and Theorem \ref{trm:15.7} that there exist natural numbers $N_a,N_b$ such that for all $n\geq N_a$, we have $|a_n-a|<\frac{\epsilon}{2}$ and for all $n\geq N_b$, we have $|b_n-b|<\frac{\epsilon}{2}$. Now choose $N=\max(N_a,N_b)$. Let $n$ be an arbitrary natural number such that $n\geq N$. It follows that $n\geq N\geq N_a$, so we know that $|a_n-a|<\frac{\epsilon}{2}$. Similarly, $|b_n-b|<\frac{\epsilon}{2}$. Therefore, we have that
            \begin{align*}
                |(a_n+b_n)-(a+b)| &\leq |a_n-a|+|b_n-b|\tag*{Lemma \ref{lem:8.8}}\\
                &< \frac{\epsilon}{2}+\frac{\epsilon}{2}\\
                &= \epsilon
            \end{align*}
        \end{proof}
        \item \label{trm:15.9b}$\lim_{n\to\infty}(a_n\cdot b_n)=\left( \lim_{n\to\infty}a_n \right)\cdot\left( \lim_{n\to\infty}b_n \right)$.
        \begin{proof}
            Let $a=\lim_{n\to\infty}a_n$ and let $b=\lim_{n\to\infty}b_n$. To prove that $\lim_{n\to\infty}(a_n\cdot b_n)$ exists and equals $\left( \lim_{n\to\infty}a_n \right)\cdot\left( \lim_{n\to\infty}b_n \right)$, Definition \ref{dfn:15.5} and Theorem \ref{trm:15.7} tell us that it will suffice too show that for all $\epsilon>0$, there is some $N\in\N$ such that for all $n\geq N$, we have $|a_n\cdot b_n-a\cdot b|<\epsilon$. Let $\epsilon>0$ be arbitrary. By Theorem \ref{trm:15.13}\footnote{The proof of Theorem \ref{trm:15.13} does not depend on any following results, so its use here is not circular logic.}, $(a_n)$ is bounded. Thus, by Definition \ref{dfn:15.12}, $\{a_n\mid n\in\N\}$ is bounded. Consequently, by the proof of Exercise \ref{exr:13.9}, there exists a number $M_a$ such that $|a_n|<M_a$ for all $n\in\N$. Now define $M=\max(M_a,b)$. Using this $M$ (which by definition is positive since it's greater than $|a_n|$, which is at least 0) as well as our previously defined arbitrary $\epsilon$, we have by consecutive applications of Definition \ref{dfn:15.5} and Theorem \ref{trm:15.7} that there exist natural numbers $N_a,N_b$ such that for all $n\geq N_a$, we have $|a_n-a|<\frac{\epsilon}{2M}$ and for all $n\geq N_b$, we have $|b_n-b|<\frac{\epsilon}{2M}$. Now choose $N=\max(N_a,N_b)$. Let $n$ be an arbitrary natural number such that $n\geq N$. It follows that $n\geq N\geq N_a$, so we know that $|a_n-a|<\frac{\epsilon}{2M}$. Similarly, $|b_n-b|<\frac{\epsilon}{2M}$. Therefore,
            \begin{align*}
                |a_nb_n-ab| &= |a_n(b_n-b)+b(a_n-a)|\\
                &\leq |a_n|\cdot|b_n-b|+|b|\cdot|a_n-a|\tag*{Lemma \ref{lem:8.8}}\\
                &\leq |M|\cdot|b_n-b|+|M|\cdot|a_n-a|\\
                &< M\cdot\frac{\epsilon}{2M}+M\cdot\frac{\epsilon}{2M}\\
                &= \frac{\epsilon}{2}+\frac{\epsilon}{2}\\
                &= \epsilon
            \end{align*}
        \end{proof}
    \end{enumerate}
    Moreover, if $\lim_{n\to\infty}b_n\neq 0$, then
    \begin{enumerate}[resume,label={\textup{(}\alph*\textup{)}},ref={\thetheorem\alph*}]
        \item \label{trm:15.9c}$\lim_{n\to\infty}\frac{a_n}{b_n}=\frac{\lim_{n\to\infty}a_n}{\lim_{n\to\infty}b_n}$.
        \begin{lemma*}
            Let $\lim_{n\to\infty}b_n=b\neq 0$. Then there exists $m\in\R^+$ such that $m\leq|b|$ and $N\in\N$ such that if $n\geq N$, then $m\leq|b_n|$.
            \begin{proof}
                Since $b\neq 0$, it follows from Definition \ref{dfn:8.4} that $0<|b|$. Thus, by Theorem \ref{trm:5.2}, there exists a point $m\in\R$ such that $0<m<|b|$. It follows from the fact that $0<m$ that $m\in\R^+$, and from the fact that $m<|b|$ that $m\leq|b|$, as desired.\par\smallskip
                As to the other part of the proof, we divide into two cases ($b>0$ and $b<0$).\par
                Suppose first that $b>0$. By Exercise \ref{exr:6.5}, Axiom \ref{axm:3.3}, and Definition \ref{dfn:3.3}, there exists a number $y$ such that $b<y$. Now consider the region $(m,y)$. Since $m<|b|=b<y$, Equations \ref{eqn:8.1} assert that $b\in(m,y)$. Additionally, by Corollary \ref{cly:4.11} and Lemma \ref{lem:8.3}, $(m,y)$ is an open interval. Thus, by Definitions \ref{dfn:15.5} and \ref{dfn:15.2}, there exists $N\in\N$ such that if $n\geq N$, then $b_n\in(m,y)$. Choose this $N$ to be our $N$. Let $n$ be an arbitrary natural number such that $n\geq N$. Then $b_n\in(m,y)$. It follows by Equations \ref{eqn:8.1} that $m<b_n<y$, which can be weakened to $m\leq b_n$. Since $0<m\leq b_n$, Definition \ref{dfn:8.4} asserts that $m\leq|b_n|$, as desired.\par
                Now suppose that $b<0$. By Exercise \ref{exr:6.5}, Axiom \ref{axm:3.3}, and Definition \ref{dfn:3.3}, there exists a number $x$ such that $x<b$. Now consider the region $(x,-m)$. Since $m<|b|=-b$, we have by Lemma \ref{lem:7.24} that $b<-m$. This combined with the fact that $x<b$ implies by Equations \ref{eqn:8.1} that $b\in(x,-m)$. Additionally, by Corollary \ref{cly:4.11} and Lemma \ref{lem:8.3}, $(x,-m)$ is an open interval. Thus, by Definitions \ref{dfn:15.5} and \ref{dfn:15.2}, there exists $N\in\N$ such that if $n\geq N$, then $b_n\in(x,-m)$. Choose this $N$ to be our $N$. Let $n$ be an arbitrary natural number such that $n\geq N$. Then $b_n\in(x,-m)$. It follows by Equations \ref{eqn:8.1} that $x<b_n<-m$, which can be weakened to $b_n\leq -m$. Consequently, by Lemma \ref{lem:7.24}, $m\leq -b_n$. Since $0<m\leq -b_n$, Definition \ref{dfn:8.4} asserts that $m\leq|b_n|$, as desired.
            \end{proof}
        \end{lemma*}
        \begin{proof}[Proof of Theorem \ref{trm:15.9c}]
            Let $a=\lim_{n\to\infty}a_n$ and let $b=\lim_{n\to\infty}b_n$. To prove that $\lim_{n\to\infty}\frac{a_n}{b_n}$ exists and equals $\frac{\lim_{n\to\infty}a_n}{\lim_{n\to\infty}b_n}$, Definition \ref{dfn:15.5} and Theorem \ref{trm:15.7} tell us that it will suffice to show that for all $\epsilon>0$, there is some $N\in\N$ such that for all $n\geq N$, we have $|\frac{a_n}{b_n}-\frac{a}{b}|<\epsilon$. Let $\epsilon>0$ be arbitrary. Choose $M=\max(|a|,|b|)$. Additionally, by the lemma, choose $m\in\R,N'\in\N$ such that $m\leq|b|$ and if $n\geq N'$, then $m\leq|b_n|$. Using this $M$ and $m$ (which, again, by definition are both positive and nonzero) as well as our previously defined arbitrary $\epsilon$, we have by consecutive applications of Definition \ref{dfn:15.5} and Theorem \ref{trm:15.7} that there exist natural numbers $N_a,N_b$ such that for all $n\geq N_a$, we have $|a_n-a|<\frac{\epsilon m^2}{M}$ and for all $n\geq N_b$, we have $|b_n-b|<\frac{\epsilon m^2}{M}$. Now choose $N=\max(N',N_a,N_b)$. Let $n$ be an arbitrary natural number such that $n\geq N$. It follows that $n\geq N\geq N_a$, so we know that $|a_n-a|<\frac{\epsilon m^2}{M}$. Additionally, since $n\geq N'$, $m\leq|b|$ and $m\leq|b_n|$. Similarly, $|b_n-b|<\frac{\epsilon m^2}{M}$. Therefore,
            \begin{align*}
                \left| \frac{a_n}{b_n}-\frac{a}{b} \right| &= \left| \frac{a_nb-b_na}{b_nb} \right|\\
                &= \frac{|b(a_n-a)+a(b-b_n)|}{|b_n|\cdot|b|}\\
                &\leq \frac{|b|\cdot|a_n-a|+|a|\cdot|b-b_n|}{|b_n|\cdot|b|}\tag*{Lemma \ref{lem:8.8}}\\
                &\leq \frac{M\cdot|a_n-a|+M\cdot|b_n-b|}{m\cdot m}\\
                &= \frac{M}{m^2}\cdot|a_n-a|+\frac{M}{m^2}\cdot|b_n-b|\\
                &< \frac{\epsilon}{2}+\frac{\epsilon}{2}\\
                &= \epsilon
            \end{align*}
        \end{proof}
    \end{enumerate}
\end{theorem}

\begin{exercise}\label{exr:15.10}\marginnote{5/13:}
    Which of the following sequences $(a_n)$ converge? Which diverge? For each that converges, what is the limit? Give proofs for your answers.
    \begin{enumerate}[label={(\alph*)},ref={\theexercise\alph*}]
        \item \label{exr:15.10a}$a_n=(-1)^n\cdot n$.
        \begin{proof}
            Suppose for the sake of contradiction that $\lim_{n\to\infty}a_n$ converges. Then since $(b_n)$ defined by $b_n=\frac{1}{n}$ converges by Exercise \ref{exr:15.6c}, we have by Theorem \ref{trm:15.9} that
            \begin{align*}
                \left( \lim_{n\to\infty}(-1)^n\cdot n \right)\cdot\left( \lim_{n\to\infty}\frac{1}{n} \right) &= \lim_{n\to\infty}(-1)^n\cdot n\cdot\frac{1}{n}\\
                &= \lim_{n\to\infty}(-1)^n
            \end{align*}
            But by Exercise \ref{exr:15.6d}, $\lim_{n\to\infty}(-1)^n$ diverges, a contradiction.
        \end{proof}
        \item \label{exr:15.10b}$a_n=\frac{1}{n^2+1}(2+\frac{1}{n})$.
        \begin{proof}
            To prove that $\lim_{n\to\infty}\frac{1}{n^2+1}(2+\frac{1}{n})=0$, we will first confirm that $\lim_{n\to\infty}\frac{1}{n^2+1}=0$ and $\lim_{n\to\infty}2=2$. These results can be tied together with the fact that $\lim_{n\to\infty}\frac{1}{n}=0$ (by Exercise \ref{exr:15.6c}) to prove the desired result with Theorem \ref{trm:15.9}. Let's begin.\par
            To confirm that $\lim_{n\to\infty}\frac{1}{n^2+1}=0$, Theorem \ref{trm:15.7} tells us that it will suffice to demonstrate that for all $\epsilon>0$, there is some $N\in\N$ such that for all $n\geq N$, we have $|\frac{1}{n^2+1}-0|=|\frac{1}{n^2+1}|<\epsilon$. Let $\epsilon>0$ be arbitrary. Since $\lim_{n\to\infty}\frac{1}{n}=0$ by Exercise \ref{exr:15.6c}, we have by Theorem \ref{trm:15.7} that there exists an $N\in\N$ such that for all $n\geq N$, we have $|\frac{1}{n}|<\epsilon$. Choose this $N$ to be our $N$. Let $n$ be an arbitrary natural number such that $n\geq N$. Then since $n\leq n^2<n^2+1$ by Script \ref{sct:7}, we have that
            \begin{align*}
                \left| \frac{1}{n^2+1} \right| &= \frac{1}{n^2+1}\tag*{Definition \ref{dfn:8.4}}\\
                &< \frac{1}{n^2}\\
                &\leq \frac{1}{n}\\
                &= \left| \frac{1}{n} \right|\tag*{Definition \ref{dfn:8.4}}\\
                &< \epsilon
            \end{align*}
            as desired.\par
            The proof that $\lim_{n\to\infty}2=2$ is symmetric to that of Exercise \ref{exr:15.6a}.\par
            Having established that $\lim_{n\to\infty}\frac{1}{n^2+1}=0$, $\lim_{n\to\infty}2=2$, and $\lim_{n\to\infty}\frac{1}{n}=0$, we have by consecutive applications of Theorem \ref{trm:15.9} that
            \begin{align*}
                0 &= 0\cdot(2+0)\\
                &= \left( \lim_{n\to\infty}\frac{1}{n^2+1} \right)\cdot\left( \lim_{n\to\infty}2+\lim_{n\to\infty}\frac{1}{n} \right)\\
                &= \lim_{n\to\infty}\frac{1}{n^2+1}\left( 2+\frac{1}{n} \right)
            \end{align*}
            as desired.
        \end{proof}
        \item \label{exr:15.10c}$a_n=\frac{5n+1}{2n+3}$.
        \begin{proof}
            The proof that $\lim_{n\to\infty}3=3$ is symmetric to that of Exercise \ref{exr:15.6a}. Additionally, by Exercise \ref{exr:15.6a}, the proof of Exercise \ref{exr:15.10b}, and Exercise \ref{exr:15.6c}, we know that $\lim_{n\to\infty}5=5$, $\lim_{n\to\infty}2=2$, and $\lim_{n\to\infty}\frac{1}{n}=0$, respectively. Therefore, by consecutive applications of Theorem \ref{trm:15.9}, we have that
            \begin{align*}
                \frac{5}{2} &= \frac{5+0}{2+3\cdot 0}\\
                &= \frac{\lim_{n\to\infty}5+\lim_{n\to\infty}\frac{1}{n}}{\lim_{n\to\infty}2+\left( \lim_{n\to\infty}3 \right)\cdot\left( \lim_{n\to\infty}\frac{1}{n} \right)}\\
                &= \lim_{n\to\infty}\frac{5+\frac{1}{n}}{2+3\cdot\frac{1}{n}}\\
                &= \lim_{n\to\infty}\frac{5n+1}{2n+3}
            \end{align*}
            as desired.
        \end{proof}
        \item \label{exr:15.10d}$a_n=\frac{(-1)^n+1}{n}$.
        \begin{proof}
            By Exercises \ref{exr:15.8a} and \ref{exr:15.6c}, we know that $\lim_{n\to\infty}\frac{(-1)^n}{n}=0$ and $\lim_{n\to\infty}\frac{1}{n}=0$, respectively. Therefore, by Theorem \ref{trm:15.9},
            \begin{align*}
                0 &= 0+0\\
                &= \lim_{n\to\infty}\frac{(-1)^n}{n}+\lim_{n\to\infty}\frac{1}{n}\\
                &= \lim_{n\to\infty}\left( \frac{(-1)^n}{n}+\frac{1}{n} \right)\\
                &= \lim_{n\to\infty}\frac{(-1)^n+1}{n}
            \end{align*}
            as desired.
        \end{proof}
    \end{enumerate}
\end{exercise}

We've used the word "limit" in two contexts now: The limit of a point in a set, and the limit of a sequence. The definitions of these two terms may seem similar. Is there a formal connection? Theorem \ref{trm:15.11} alludes to an answer.

\begin{theorem}\label{trm:15.11}
    Let $A\subset\R$. Then $p\in\overline{A}$ if and only if there exists a sequence $(a_n)$, with each $a_n\in A$, that converges to $p$.
    \begin{proof}
        % Suppose for the sake of contradiction that $p\notin\overline{A}$. Then by Definition \ref{dfn:4.4}, $p\notin A$ and $p\notin LP(A)$. Consider an arbitrary region $R$ containing $p$. Since $\lim_{n\to\infty}a_n=p$, there exists $N\in\N$ such that for all $n\geq N$, $a_n\in R$. It follows that $R\cap A\neq\emptyset$. Additionally, $A=A\setminus\{p\}$ since $p\notin A$. Thus, $R\cap(A\setminus\{p\})\neq\emptyset$. Therefore, $p\in LP(A)$, a contradiction.

        Suppose first that $p\in\overline{A}$. Then by Definitions \ref{dfn:4.4} and \ref{dfn:1.5}, $p\in A$ or $p\in LP(A)$. We now divide into two cases. If $p\in A$, then define $(a_n)$ by $a_n=p$ for all $n\in\N$. Clearly, each $a_n\in A$ since $p\in A$, and $\lim_{n\to\infty}a_n=p$ by a proof symmetric to that of Exercise \ref{exr:15.6a}, as desired. If $p\in LP(A)$, then define $R_n=(p-\frac{1}{n},p+\frac{1}{n})$ for all $n\in\N$. Since $p\in LP(A)$, we have by Definition \ref{dfn:3.13} that $R_n\cap(A\setminus\{p\})\neq\emptyset$ for all $n\in\N$. It follows by the axiom of choice that we can choose a point $a_n$ in $R_n\cap(A\setminus\{p\})$ for all $n\in\N$. Thus, by Definitions \ref{dfn:1.6} and \ref{dfn:1.11}, each $a_n\in A$ (as desired) and $a_n\in R_n$ for all $n\in\N$. We now seek to prove that $(a_n)$ converges to $p$; to do so, Theorem \ref{trm:15.7} tells us that it will suffice to show that for all $\epsilon>0$, there exists an $N\in\N$ such that for all $n\geq N$, $|a_n-p|<\epsilon$. Let $\epsilon>0$ be arbitrary. By Corollary \ref{cly:6.12}, there exists an $N\in\N$ such that $\frac{1}{N}<\epsilon$. Choose this $N$ to be our $N$. Let $n$ be an arbitrary natural number such that $n\geq N$. Then by consecutive applications of Lemma \ref{lem:7.24}, $\frac{1}{n}\leq\frac{1}{N}$. Consequently, by Script \ref{sct:1} as well as the definitions of $R_n$ and $R_N$, $R_n\subset R_N$. It follows by Definition \ref{dfn:1.3} since $a_n\in R_n$ that $a_n\in R_N$. Therefore, by Exercise \ref{exr:8.9}, $|a_n-p|<\frac{1}{N}<\epsilon$, as desired.\par
        Now suppose that there exists a sequence $(a_n)$ with each $a_n\in A$ that converges to $p$. We divide into two cases ($p\in A$ and $p\notin A$). If $p\in A$, then by Definitions \ref{dfn:1.5} and \ref{dfn:4.4}, $p\in\overline{A}$, as desired. If $p\notin A$, then to prove that $p\in\overline{A}$, Definitions \ref{dfn:4.4} and \ref{dfn:1.5} tell us that we must show that $p\in LP(A)$. To do so, Definition \ref{dfn:3.13} tells us that it will suffice to verify that for all regions $R$ containing $p$, $R\cap(A\setminus\{p\})\neq\emptyset$. Let $R$ be an arbitrary region $R$ with $p\in R$. By Corollary \ref{cly:4.11} and \ref{lem:8.3}, $R$ is an open interval. Thus, by Definition \ref{dfn:15.2}, there exists an $N\in\N$ such that if $n\geq N$, then $a_n\in R$. It follows that $a_N\in R$. Additionally, by hypothesis, $a_N\in A$. These results combined with the fact that $A=A\setminus\{p\}$ (since $p\notin A$) imply by Definition \ref{dfn:1.6} that $a_N\in R\cap(A\setminus\{p\})$. Therefore, by Definition \ref{dfn:1.8}, $R\cap(A\setminus\{p\})\neq\emptyset$, as desired.
    \end{proof}
\end{theorem}

\begin{definition}\label{dfn:15.12}
    A sequence $(a_n)$ is \textbf{bounded} if its image $\{a_n\mid n\in\N\}$ is bounded.
\end{definition}

\begin{theorem}\label{trm:15.13}
    Every convergent sequence is bounded.
    \begin{proof}
        Let $(a_n)$ be a sequence that converges to $p$. To prove that $(a_n)$ is bounded, Definitions \ref{dfn:15.12} and \ref{dfn:5.6} tell us that it will suffice to find numbers $l,u$ such that $l\leq a_n\leq u$ for all $a_n$. Let $(x,y)$ be a region that contains $p$. By Corollary \ref{cly:4.11} and Lemma \ref{lem:8.3}, $(x,y)$ is an open interval. Thus, by Exercise \ref{exr:15.3}, we have that $(x,y)$ contains all but finitely many terms of the sequence, i.e., $\{a_n\mid a_n\notin(x,y)\}$ is finite. We divide into two cases ($\{a_n\mid a_n\notin(x,y)\}=\emptyset$ and $\{a_n\mid a_n\notin(x,y)\neq\emptyset$). If $\{a_n\mid a_n\notin(x,y)\}=\emptyset$, then $a_n\in(x,y)$ for all $a_n$. It follows by Equations \ref{eqn:8.1} that $x<a_n<y$ for all $a_n$. If we now choose $l=x$ and $u=y$, we can weaken the previous statement to $l=x\leq a_n\leq y=u$, as desired. On the other hand, if $\{a_n\mid a_n\notin(x,y)\}\neq\emptyset$, then by Lemma \ref{lem:3.4}, $\{a_n\mid a_n\notin(x,y)\}$ has a first and a last point. It follows by Exercise \ref{exr:5.9} that $\{a_n\mid a_n\notin(x,y)\}$ is bounded by $\inf\{a_n\mid a_n\notin(x,y)\}$ and $\sup\{a_n\mid a_n\notin(x,y)\}$. Choose $l=\min(x,\inf\{a_n\mid a_n\notin(x,y)\})$ and $u=\max(y,\sup\{a_n\mid a_n\notin(x,y)\})$. Let $a_n$ be an arbitrary term in the sequence. We divide into two subcases ($a_n\in(x,y)$ and $a_n\notin(x,y)$). If $a_n\in(x,y)$, then $l\leq x<a_n<y\leq u$, as desired. On the other hand, if $a_n\notin(x,y)$, then $l\leq\inf\{a_n\mid a_n\notin(x,y)\}\leq a_n\leq\sup\{a_n\mid a_n\notin(x,y)\}\leq u$, as desired.
    \end{proof}
\end{theorem}

The converse is not true, but there are important partial converses. For the first, Theorem \ref{trm:15.14}, we recall Definition \ref{dfn:8.16} along with Definition \ref{dfn:15.1}, which say that $(a_n)$ is an increasing sequence if $a_n\leq a_{n+1}$ for all $n\in\N$, and $(a_n)$ is decreasing if $a_n\geq a_{n+1}$ for all $n\in\N$. The definitions for strictly increasing/strictly decreasing are similar but with strict inequalities.

\begin{theorem}\label{trm:15.14}
    Every bounded increasing sequence converges to the supremum of its image. Every bounded decreasing sequence converges to the infimum of its image.
    \begin{proof}
        We will only address the first part of the theorem; the proof of the second part is symmetric.\par
        Let $(a_n)$ be a bounded increasing sequence and let $p=\sup\{a_n\mid n\in\N\}$. To prove that $(a_n)$ converges to $p$, Theorem \ref{trm:15.7} tells us  that it will suffice to show that for all $\epsilon>0$, there is some $N\in\N$ such that for all $n\geq N$, we have $|a_n-p|<\epsilon$. Let $\epsilon>0$ be arbitrary. By Lemma \ref{lem:5.11}, there exists $a_N\in\{a_n\mid n\in\N\}$ such that $p-\epsilon<a_N\leq p$. Choose $N$ to be the natural number that generates $a_N$. Let $n$ be an arbitrary natural number such that $n\geq N$. Then since $a_N\leq a_{N+1}\leq\cdots\leq a_{n-1}\leq a_n$, we have by transitivity that $a_N\leq a_n$. Additionally, since $a_n\in\{a_n\mid n\in\N\}$, we have by Definitions \ref{dfn:5.7} and \ref{dfn:5.6} that $a_n\leq p$. Thus, since $p-\epsilon<a_N\leq a_n\leq p<p+\epsilon$, we have by Equations \ref{eqn:8.1} that $a_n\in(p-\epsilon,p+\epsilon)$. Therefore, by Exercise \ref{exr:8.9}, $|a_n-p|<\epsilon$, as desired.
    \end{proof}
\end{theorem}

To discuss the second partial converse, Theorem \ref{trm:15.18}, we need another definition.

\begin{definition}\label{dfn:15.15}
    Let $(a_n)$ be a sequence. A \textbf{subsequence} of $(a_n)$ is a sequence $b:\N\to\R$ defined by the composition $b=a\circ i$, where $i:\N\to\N$ is a strictly increasing function. If $(a_n)$ has a subsequence with limit $p$, we call $p$ a \textbf{subsequential limit} of $(a_n)$.
\end{definition}

We can write $b_k=a(i(k))=a_{i(k)}=a_{i_k}$, so that $(b_k)$ is the sequence $b_1,b_2,b_3,\dots$, which is equal to the sequence $a_{i_1},a_{i_2},a_{i_3},\dots$, where $i_1<i_2<i_3<\cdots$.

\begin{theorem}\label{trm:15.16}
    If $(a_n)$ converges to $p$, then so do all of its subsequences.
    \begin{proof}
        Let $(b_n)$ be an arbitrary subsequence of $(a_n)$. To prove that $(b_n)$ converges to $p$, Theorem \ref{trm:15.7} tells us that it will suffice to show that for all $\epsilon>0$, there exists an $N\in\N$ such that for all $n\geq N$, we have $|b_n-p|<\epsilon$. Let $\epsilon>0$ be arbitrary. Since $(a_n)$ converges to $p$, Theorem \ref{trm:15.7} implies that there exists an $N\in\N$ such that for all $n\geq N$, we have $|a_n-p|<\epsilon$. Choose this $N$ to be our $N$. Let $n$ be an arbitrary natural number such that $n\geq N$. By Definition \ref{dfn:15.15} and Script \ref{sct:1}, $i(n)\geq n$. Therefore, we have by the above that $|b_n-p|=|a_{i_n}-p|<\epsilon$, as desired.
    \end{proof}
\end{theorem}




\end{document}