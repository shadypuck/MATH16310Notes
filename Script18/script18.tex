\documentclass[../main.tex]{subfiles}

\pagestyle{main}
\renewcommand{\chaptermark}[1]{\markboth{\chaptername\ \thechapter}{}}
\setcounter{chapter}{17}

\begin{document}




\chapter[The Euclidean Space \texorpdfstring{$\protect\fakeboldb{\R}^{\bm{n}}$}{TEXT}]{The Euclidean Space \texorpdfstring{$\protect\fakeboldc{\R}^{\bm{n}}$}{TEXT}}\label{sct:18}
\begin{definition}\label{dfn:18.1}\marginnote{7/7:}
    The \textbf{Euclidean $\bm{n}$-space} $\R^n$ is the $n$-fold Cartesian product of $\R$. Symbolically,
    \begin{equation*}
        \R^n = \{(x_1,\dots,x_n)\mid x_1,\dots,x_n\in\R\}
    \end{equation*}
    is the set of $n$-tuples of real numbers. We often write
    \begin{equation*}
        \x = (x_1,\dots,x_n)
    \end{equation*}
    to denote an element, which is also referred to as a \textbf{vector}, in $\R^n$ and
    \begin{equation*}
        \bm{0} = (0,\dots,0)
    \end{equation*}
\end{definition}

\begin{definition}\label{dfn:18.2}
    Let $\x=(x_1,\dots,x_n),\y=(y_1,\dots,y_n)\in\R^n$ and $\lambda\in\R$. We define the following operations.
    \begin{enumerate}[label={(\alph*)}]
        \item (Addition) $\x+\y=(x_1+y_1,\dots,x_n+y_n)$.
        \item (Scalar Multiplication) $\lambda\x=(\lambda x_1,\dots,\lambda x_n)$.
    \end{enumerate}
\end{definition}

\begin{exercise}\label{exr:18.3}
    Prove that the addition on $\R^n$ satisfies FA1-FA4 (see Definition \ref{dfn:7.8}). Moreover, prove that
    \begin{enumerate}[label={VS\arabic*.}]
        \item (Associativity of Scalar Multiplication) If $\lambda,\mu\in\R$ and $\x\in\R^n$, then $(\lambda\mu)\x=\lambda(\mu\x)$.
        \item (Distributivity of Scalars) If $\lambda,\mu\in\R$ and $\x\in\R^n$, then $(\lambda+\mu)\x=\lambda\x+\mu\x$.
        \item (Distributivity of Vectors) If $\lambda\in\R$ and $\x,\y\in\R^n$, then $\lambda(\x+\y)=\lambda\x+\lambda\y$.
        \item (Scalar Multiplicative Identity) If $\x\in\R^n$, then $1\x=\x$.
    \end{enumerate}
    These eight properties together are called the \textbf{vector space axioms}.
    \begin{proof}
        To prove that $\R^n$ obeys FA1 from Definition \ref{dfn:7.8}, it will suffice to show that for all $\x,\y\in\R^n$, $\x+\y=\y+\x$. Let $\x,\y$ be arbitrary elements of $\R^n$. Then by Definition \ref{dfn:18.2},
        \begin{align*}
            \x+\y &= (x_1+y_1,\dots,x_n+y_n)\\
            &= (y_1+x_1,\dots,y_n+x_n)\\
            &= \y+\x
        \end{align*}
        as desired.\par
        To prove that $\R^n$ obeys FA2 from Definition \ref{dfn:7.8}, it will suffice to show that for all $\x,\y,\z\in\R^n$, $(\x+\y)+\z=\x+(\y+\z)$. Let $\x,\y,\z$ be arbitrary elements of $\R^n$. Then by Definition \ref{dfn:18.2},
        \begin{align*}
            (\x+\y)+\z &= (x_1+y_1,\dots,x_n+y_n)+\z\\
            &= ((x_1+y_1)+z_1,\dots,(x_n+y_n)+z_n)\\
            &= (x_1+(y_1+z_1),\dots,x_n+(y_n+z_n))\\
            &= \x+(y_1+z_1,\dots,y_n+z_n)\\
            &= \x+(\y+\z)
        \end{align*}
        as desired.\par
        To prove that $\R^n$ obeys FA3 from Definition \ref{dfn:7.8}, it will suffice to find an element $0\in\R^n$ such that $\x+0=0+\x=\x$ for all $\x\in\R^n$. Choose $\bm{0}$ to be our $0$. Let $\x$ be an arbitrary element of $\R^n$. Then by Definition \ref{dfn:18.2},
        \begin{align*}
            \x+\bm{0} &= (x_1+0,\dots,x_n+0)\\
            &= (x_1,\dots,x_n)\\
            &= \x\\
            &= (0+x_1,\dots,0+x_n)\\
            &= \bm{0}+\x
        \end{align*}
        as desired.\par
        To prove that $\R^n$ obeys FA4 from Definition \ref{dfn:7.8}, it will suffice to show that for all $\x\in\R^n$, there exists $\y\in\R^n$ such that $\x+\y=\y+\x=0$. Let $\x$ be an arbitrary element of $\R^n$. Choose $\y=(-x_1,\dots,-x_n)$. Then by Definition \ref{dfn:18.2},
        \begin{align*}
            \x+\y &= (x_1+(-x_1),\dots,x_n+(-x_n))\\
            &= (0,\dots,0)\\
            &= \bm{0}\\
            &= ((-x_1)+x_1,\dots,(-x_n)+x_n)\\
            &= \y+\x
        \end{align*}
        as desired.\par
        To prove that $\R^n$ obeys VS1, it will suffice to show that for all $\lambda,\mu\in\R$ and $\x\in\R^n$, we have $(\lambda\mu)\x=\lambda(\mu\x)$. Let $\lambda,\mu$ be arbitrary elements of $\R$, and let $\x$ be an arbitrary element of $\R^n$. Then by Definition \ref{dfn:18.2},
        \begin{align*}
            (\lambda\mu)\x &= ((\lambda\mu)x_1,\dots,(\lambda\mu)x_n)\\
            &= (\lambda(\mu x_1),\dots,\lambda(\mu x_n))\\
            &= \lambda(\mu\x)
        \end{align*}
        as desired.\par
        To prove that $\R^n$ obeys VS2, it will suffice to show that for all $\lambda,\mu\in\R$ and $\x\in\R^n$, we have $(\lambda+\mu)\x=\lambda\x+\mu\x$. Let $\lambda,\mu$ be arbitrary elements of $\R$, and let $\x$ be an arbitrary element of $\R^n$. Then by Definition \ref{dfn:18.2},
        \begin{align*}
            (\lambda+\mu)\x &= ((\lambda+\mu)x_1,\dots,(\lambda+\mu)x_n)\\
            &= (\lambda x_1+\mu x_1,\dots,\lambda x_n+\mu x_n)\\
            &= (\lambda x_1,\dots,\lambda x_n)+(\mu x_1,\dots,\mu x_n)\\
            &= \lambda\x+\mu\x
        \end{align*}
        as desired.\par
        To prove that $\R^n$ obeys VS3, it will suffice to show that for all $\lambda\in\R$ and $\x,\y\in\R^n$, we have $\lambda(\x+\y)=\lambda\x+\lambda\y$. Let $\lambda$ be an arbitrary element of $\R$, and let $\x,\y$ be arbitrary elements of $\R^n$. Then by Definition \ref{dfn:18.2},
        \begin{align*}
            \lambda(\x+\y) &= \lambda(x_1+y_1,\dots,x_n+y_n)\\
            &= (\lambda(x_1+y_1),\dots,\lambda(x_n+y_n))\\
            &= (\lambda x_1+\lambda y_1,\dots,\lambda x_n+\lambda y_n)\\
            &= (\lambda x_1,\dots,\lambda x_n)+(\lambda y_1,\dots,\lambda y_2)\\
            &= \lambda\x+\lambda\y
        \end{align*}
        as desired.\par
        To prove that $\R^n$ obeys VS4, it will suffice to show that for all $\x\in\R^n$, we have $1\x=\x$. Let $\x$ be an arbitrary element of $\R^n$. Then by Definition \ref{dfn:18.2},
        \begin{align*}
            1\x &= (1x_1,\dots,1x_n)\\
            &= (x_1,\dots,x_n)\\
            &= \x
        \end{align*}
        as desired.
    \end{proof}
\end{exercise}

\begin{remark}\label{rmk:18.4}
    Since $\R^n$ with the two operations defined as above satisfies these eight axioms, we call $\R^n$ a \textbf{vector space}.
\end{remark}

\begin{exercise}\label{exr:18.5}
    Prove that if $\x\in\R^n$, then $0\x=\bm{0}$.
    \begin{proof}
        By Definition \ref{dfn:18.2}, we have that
        \begin{align*}
            0\x &= (0x_1,\dots,0x_n)\\
            &= (0,\dots,0)\\
            &= \bm{0}
        \end{align*}
        as desired.
    \end{proof}
\end{exercise}

\begin{definition}\label{dfn:18.6}
    Let $\x\in\R^n$. The \textbf{norm} of $\x$ is defined as
    \begin{equation*}
        \norm{\x} = \sqrt{x_1^2+\cdots+x_n^2}
    \end{equation*}
\end{definition}

\begin{definition}\label{dfn:18.7}
    We call $\norm{\y-\x}$ the \textbf{distance} between $\x$ and $\y$.
\end{definition}

\begin{remark}\label{rmk:18.8}
    If $n=1$, the norm coincides with the definition of the absolute value in $\R$.
\end{remark}

\begin{lemma}\label{lem:18.9}
    \leavevmode
    \begin{enumerate}[label=\textup{(}\alph*\textup{)}]
        \item If $x,y\in\R$, then $xy\leq\frac{x^2+y^2}{2}$.
        \begin{proof}
            Let $x,y$ be arbitrary elements of $\R$. Then by Lemma \ref{lem:7.26}, $0\leq(x-y)^2$. Therefore, we have that
            \begin{align*}
                xy &= \frac{2xy+0}{2}\\
                &\leq \frac{2xy+(x-y)^2}{2}\\
                &= \frac{2xy+x^2-2xy+y^2}{2}\\
                &= \frac{x^2+y^2}{2}
            \end{align*}
            as desired.
        \end{proof}
        \item If $\x,\y\in\R^n$, then $|x_1y_1+\cdots+x_ny_n|\leq\norm{\x}\cdot\norm{\y}$.
        \begin{proof}
            Suppose first that $\norm{\x}=\norm{\y}=1$. Then by Definition \ref{dfn:18.6}, $\norm{\x}=1=\sqrt{x_1^2+\cdots+x_n^2}$, from which it follows that $1=x_1^2+\cdots+x_n^2$. Therefore, we have that
            \begin{align*}
                |x_1y_1+\cdots+x_ny_n| &\leq |x_1y_1|+\cdots+|x_ny_n|\tag*{Lemma \ref{lem:8.8}}\\
                &\leq \frac{x_1^2+y_1^2}{2}+\cdots+\frac{x_n^2+y_n^2}{2}\\
                &= \frac{(x_1^2+\cdots+x_n^2)+(y_1^2+\cdots+y_n^2)}{2}\\
                &= \frac{1+1}{2}\\
                &= 1
            \end{align*}
            as desired.\par
            Now let $\x,\y$ be arbitrary elements of $\R^n$. Consider the vectors $\mathbf{u}_\x,\mathbf{u}_\y$ defined by $\mathbf{u}_\x=\frac{\x}{\norm{\x}}$ and $\mathbf{u}_\y=\frac{\y}{\norm{\y}}$. By the proof of the first case, we have that
            \begin{align*}
                |x_1y_1+\cdots+x_ny_n| &= \norm{\x}\cdot\norm{\y}\cdot\left| \frac{x_1y_1}{\norm{\x}\cdot\norm{\y}}+\cdots+\frac{x_ny_n}{\norm{\x}\cdot\norm{\y}} \right|\\
                &= \norm{\x}\cdot\norm{\y}\cdot|u_{\x_1}u_{\y_1}+\cdots+u_{\x_n}u_{\y_n}|\\
                &\leq \norm{\x}\cdot\norm{\y}\cdot 1\\
                &= \norm{\x}\cdot\norm{\y}
            \end{align*}
            as desired.
        \end{proof}
    \end{enumerate}
\end{lemma}

\begin{theorem}\label{trm:18.10}
    If $\x,\y\in\R^n$ and $\lambda\in\R$, then
    \begin{enumerate}[label=\textup{(}\alph*\textup{)}]
        \item $\norm{\x}\geq 0$. Moreover, $\norm{\x}=0$ if and only if $\x=\bm{0}$.
        \begin{proof}
            Let $\x$ be an arbitrary element of $\R^n$.\par\smallskip
            We first prove that $\norm{\x}\geq 0$. By Lemma \ref{lem:7.26}, $x_i^2\geq 0$ for all $i\in[n]$. Thus, by Definition \ref{dfn:7.21}, $x_1^2+\cdots+x_n^2\geq 0$. Therefore, we have by Definition \ref{dfn:18.6} that $\norm{\x}=\sqrt{x_1^2+\cdots+x_n^2}\geq 0$, as desired.\par\smallskip
            We now prove that $\norm{\x}=0$ if and only if $\x=\bm{0}$. Suppose first that $\norm{\x}=0$. Then by Definition \ref{dfn:18.6} and Script \ref{sct:7}, $x_1^2+\cdots+x_n^2=0$. Now suppose for the sake of contradiction that $\x\neq\bm{0}$. Then there exists an $x_i$ such that $x_i\neq 0$. Thus, by Lemma \ref{lem:7.26}, $x_i^2>0$. Additionally, $x_j^2\geq 0$ for all $j\in[n]$. Thus, we have that $0<x_i^2\leq x_1^2+\cdots+x_n^2$. But by Definition \ref{dfn:3.1}, this implies that $x_1^2+\cdots+x_n^2\neq 0$, a contradiction.\par
            Now suppose that $\x=\bm{0}$. Then by Definition \ref{dfn:18.6}, $\norm{\x}=\sqrt{0^2+\cdots+0^2}=0$, as desired.
        \end{proof}
        \item $\norm{\lambda\x}=|\lambda|\cdot\norm{\x}$.
        \begin{proof}
            Let $\lambda$ be an arbitrary element of $\R$, and let $\x$ be an arbitrary element of $\R^n$. Then we have that
            \begin{align*}
                \norm{\lambda\x} &= \sqrt{(\lambda x_1)^2+\cdots+(\lambda x_n)^2}\tag*{Definition \ref{dfn:18.6}}\\
                &= |\lambda|\cdot\sqrt{x_1^2+\cdots+x_n^2}\\
                &= |\lambda|\cdot\norm{\x}\tag*{Definition \ref{dfn:18.6}}
            \end{align*}
            as desired.
        \end{proof}
        \item $\norm{\x+\y}\leq\norm{\x}+\norm{\y}$.
        \begin{proof}
            Let $\x,\y$ be arbitrary elements of $\R^n$. Then we have that
            \begin{align*}
                \norm{\x+\y} &= \sqrt{(x_1+y_1)^2+\cdots+(x_n+y_n)^2}\tag*{Definition \ref{dfn:18.6}}\\
                &= \sqrt{(x_1^2+\cdots+x_n^2)+(2x_1y_1+\cdots+2x_ny_n)+(y_1^2+\cdots+y_n^2)}\\
                &\leq \sqrt{\norm{\x}^2+2\norm{\x}\norm{\y}+\norm{y}^2}\tag*{Lemma \ref{lem:18.9}}\\
                &= \sqrt{(\norm{\x}+\norm{\y})^2}\\
                &= \norm{\x}+\norm{\y}
            \end{align*}
            as desired.
        \end{proof}
    \end{enumerate}
\end{theorem}

\begin{corollary}\label{cly:18.11}
    If $\x,\y,\z\in\R^n$ and $\lambda\in\R$, then
    \begin{enumerate}[label=\textup{(}\alph*\textup{)}]
        \item $\norm{\x-\z}\leq\norm{\x-\y}+\norm{\y-\z}$.
        \item $|\norm{\x}-\norm{\y}|\leq\norm{x-y}$.
    \end{enumerate}
    \begin{proof}
        The proofs are symmetric to those of Lemma \ref{lem:8.8}.
    \end{proof}
\end{corollary}




\end{document}