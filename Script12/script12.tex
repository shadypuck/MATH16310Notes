\documentclass[../main.tex]{subfiles}

\pagestyle{main}
\renewcommand{\chaptermark}[1]{\markboth{\chaptername\ \thechapter}{}}
\setcounter{chapter}{11}

\begin{document}




\chapter{Derivatives}\label{sct:12}
\section{Journal}
\marginnote{3/30:}Throughout this sheet, we let $f:A\to\R$ be a real valued function with domain $A\subset\R$. We also now assume the domain $A\subset\R$ is open.
\begin{definition}\label{dfn:12.1}
    The \textbf{derivative} of $f$ at a point $a\in A$ is the number $f'(a)$ defined by the limit
    \begin{equation*}
        f'(a) = \lim_{h\to 0}\frac{f(a+h)-f(a)}{h}
    \end{equation*}
    provided that the limit on the right-hand side exists. If $f'(a)$ exists, we say that $f$ is \textbf{differentiable} (at $a$). If $f$ is differentiable at all points of its domain, we say that $f$ is \textbf{differentiable}. In this case, the values $f'(a)$ define a new function $f':A\to\R$ called the \textbf{derivative} (of $f$).
\end{definition}

\begin{remark}\label{rmk:12.2}
    If $A$ is not open, the limit in Definition \ref{dfn:12.1} may not exist. For example, if $f:[a,b]\to\R$, then we cannot define the derivative at the endpoints. For any $c$ in the domain of $f$, we define the \textbf{right-hand derivative} $f'_+(c)$ and the \textbf{left-hand derivative} $f'_-(c)$ by
    \begin{align*}
        f'_+(c) &= \lim_{h\to 0^+}\frac{f(c+h)-f(c)}{h}&
        f'_-(c) &= \lim_{h\to 0^-}\frac{f(c+h)-f(c)}{h}
    \end{align*}
    We say that $f$ is \textbf{differentiable} (on $[a,b]$) if $f$ is differentiable on $(a,b)$ and $f'_+(a)$ and $f'_-(b)$ exist.
\end{remark}

\begin{lemma}\label{lem:12.3}
    Let $a\in\R$. Then
    \begin{equation*}
        \lim_{x\to a}f(x) = \lim_{h\to 0}f(a+h)
    \end{equation*}
    assuming that one of the two limits exists. (So if the limit on the left exists, so does the one on the right, and they are equal. Similarly, if the limit on the right exists, then so does the one on the left, and they are equal.)
    \begin{proof}
        Suppose first that $\lim_{x\to a}f(x)$ exists, and let it be equal to $L$. To prove that $\lim_{h\to 0}f(a+h)$ exists and that it equals $\lim_{x\to a}f(x)$, Definition \ref{dfn:11.1} tells us that it will suffice to show that for every $\epsilon>0$, there exists a $\delta>0$ such that if $(h+a)\in A$ and $0<|h-0|=|h|<\delta$, then $|f(a+h)-L|<\epsilon$. Let $\epsilon>0$ be arbitrary. Since $\lim_{x\to a}f(x)$ exists, Definition \ref{dfn:11.1} implies that there exists a $\delta>0$ such that if $x\in A$ and $0<|x-a|<\delta$, then $|f(x)-L|<\epsilon$. We will choose this $\delta$ to be our $\delta$. Now let $h$ be an arbitrary number satisfying both $(h+a)\in A$ and $0<|h|<\delta$; we seek to show that $|f(a+h)-L|<\epsilon$. Since $(h+a)\in A$, $h+a=x$ for some $x\in A$. It follows that $h=x-a$, meaning that $x$ is an object that is both an element of $A$ and that satisfies $0<|h|=|x-a|<\delta$, so we know that $|f(a+h)-L|=|f(x)-L|<\epsilon$, as desired.\par
        The proof is symmetric in the other direction.
    \end{proof}
\end{lemma}

\begin{theorem}\label{trm:12.4}
    Let $a\in\R$. Then $f$ is differentiable at $a$ if and only if $\lim_{x\to a}\frac{f(x)-f(a)}{x-a}$ exists. Moreover, if $f$ is differentiable at $a$, then the derivative of $f$ at $a$ is given by the limit
    \begin{equation*}
        f'(a) = \lim_{x\to a}\frac{f(x)-f(a)}{x-a}
    \end{equation*}
    \begin{proof}
        Suppose first that $f$ is differentiable at $a$. Then by Definition \ref{dfn:12.1}, $f'(a)$ exists. It follows that
        \begin{align*}
            f'(a) &= \lim_{h\to 0}\frac{f(a+h)-f(a)}{h}\tag*{Definition \ref{dfn:12.1}}\\
            &= \lim_{h\to 0}\frac{f(a+h)-f(a)}{(a+h)-a}\\
            &= \lim_{x\to a}\frac{f(x)-f(a)}{x-a}\tag*{Lemma \ref{lem:12.3}}
        \end{align*}
        Note that the substitution in the last step follows from using $\tilde{f}(x)=\frac{f(x)-f(a)}{x-a}$ as the "$f(x)$" function in Lemma \ref{lem:12.3}.\par
        The proof is symmetric in the reverse direction.
    \end{proof}
\end{theorem}

\begin{theorem}\label{trm:12.5}
    If $f$ is differentiable at $a$, then $f$ is continuous at $a$.
    \begin{proof}
        To prove that $f$ is continuous at $a$, Theorem \ref{trm:11.5} tells us that it will suffice to show that $\lim_{x\to a}f(x)=f(a)$. By Definition \ref{dfn:12.1}, the hypothesis implies that $f'(a)$ exists. Thus, by Theorem \ref{trm:12.4}, we know that $\lim_{x\to a}\frac{f(x)-f(a)}{x-a}$ exists. Additionally, by Exercise \ref{exr:11.6}, $g(x)=x-a$ is continuous at $a$. It follows by Theorem \ref{trm:11.5} that either $a\notin LP(A)$ or $\lim_{x\to a}g(x)$ exists (and equals $g(a)$). However, since $A$ is open by hypothesis and $a\in A$, Theorem \ref{trm:4.10} implies that there exists a region $R$ such that $a\in R$ and $R\subset A$. But $a\in R$ implies that $a\in LP(R)$ by Corollary \ref{cly:5.5}, and this combined with the fact that $R\subset A$ implies by Theorem \ref{trm:3.14} that $a\in LP(A)$\footnote{I will not go through this or similar derivations again, although they may be technically necessary. Indeed, assume moving on that statements analogous to $a\in LP(A)$ hold true.}. Thus, we have that $\lim_{x\to a}g(x)=g(a)=a-a=0$. Combining the last few results, we have
        \begin{align*}
            0 &= \left( \lim_{x\to a}\frac{f(x)-f(a)}{x-a} \right)\cdot 0\\
            &= \left( \lim_{x\to a}\frac{f(x)-f(a)}{x-a} \right)\left( \lim_{x\to a}(x-a) \right)\\
            &= \lim_{x\to a}\left( \frac{f(x)-f(a)}{x-a}\cdot(x-a) \right)\tag*{Theorem \ref{trm:11.9}}\\
            &= \lim_{x\to a}(f(x)-f(a))
        \end{align*}
        If we now consider $f(a)$ to be a constant function (i.e., $\lim_{x\to a}f(a)=f(a)$ by Exercise \ref{exr:11.6}, Theorem \ref{trm:11.5}, and the above result that $a\in LP(A)$), it follows from the above that
        \begin{align*}
            \lim_{x\to a}(f(x)-f(a))+\lim_{x\to a}f(a) &= 0+f(a)\\
            \lim_{x\to a}(f(x)-f(a)+f(a)) &= f(a)\tag*{Theorem \ref{trm:11.9}}\\
            \lim_{x\to a}f(x) &= f(a)
        \end{align*}
        as desired.
    \end{proof}
\end{theorem}

\begin{exercise}\label{exr:12.6}
    Show that the converse of Theorem \ref{trm:12.5} is not true.
    \begin{proof}
        The converse of Theorem \ref{trm:12.5} asserts that "if $f$ is continuous at $a$, then $f$ is differentiable at $a$." To falsify this statement, we will use the absolute value function $|x|$ as a counterexample. Let's begin.\par
        By Exercise \ref{exr:11.7}, $|x|$ is continuous. It follows by Theorem \ref{trm:9.10} that $|x|$ is continuous at 0. However, we can show that $|x|$ is not differentiable at 0.\par
        To do this, Definition \ref{dfn:12.1} and Theorem \ref{trm:12.4} tell us that it will suffice to verify that $\lim_{x\to 0}\frac{|x|-|0|}{x-0}=\lim_{x\to 0}\frac{|x|}{x}$ does not exist. Suppose for the sake of contradiction that $\lim_{x\to 0}\frac{|x|}{x}=L$. Then by Definition \ref{dfn:11.1}, for $\epsilon=1>0$, there exists a $\delta>0$ such that if $0<|x-0|=|x|<\delta$, then $|\frac{|x|}{x}-L|<1$. However, we can show that no such $\delta$ exists. Let $\delta>0$ be arbitrary. By Theorem \ref{trm:5.2}, there exists a number $x\in\R$ such that $0<x<\delta$. It follows by Definition \ref{dfn:8.4} and Exercise \ref{exr:8.5} that $0<|x|=|-x|<\delta$. Since both $x$ and $-x$ are in the appropriate range, we know that
        \begin{align*}
            \left| \frac{|x|}{x}-L \right| &= \left| \frac{x}{x}-L \right|&
                \left| \frac{|-x|}{-x}-L \right| &= \left| \frac{x}{-x}-L \right|\tag*{Definition \ref{dfn:8.4}}\\
            &= |1-L|&
                &= |-1-L|\tag*{Script \ref{sct:7}}\\
            &= |L-1|&
                &= |L+1|\tag*{Exercise \ref{exr:8.5}}\\
            &< 1&
                &< 1
        \end{align*}
        By consecutive applications of the lemma from Exercise \ref{exr:8.9}, it follows that
        \begin{align*}
            -1 &< L-1 < 1&
                -1 &< L+1 < 1\\
            0 &< L < 2&
                -2 &< L < 0
        \end{align*}
        But this implies that $L<0$ and $L>0$, a contradiction.
    \end{proof}
\end{exercise}

\begin{exercise}\label{exr:12.7}
    Show that for all $n\in\N$,
    \begin{equation*}
        x^n-a^n = (x-a)\left( x^{n-1}+ax^{n-2}+a^2x^{n-3}+\cdots+a^{n-2}x+a^{n-1} \right)
    \end{equation*}
    or equivalently,
    \begin{equation*}
        x^n-a^n = (x-a)\left( \sum_{i=0}^{n-1}x^{n-1-i}a^i \right)
    \end{equation*}
    \begin{proof}
        By simple algebra (see Script \ref{sct:7}), we have
        \begin{align*}
            (x-a)\left( \sum_{i=0}^{n-1}x^{n-1-i}a^i \right) &= \sum_{i=0}^{n-1}(x-a)x^{n-1-i}a^i\\
            &= \sum_{i=0}^{n-1}\left( x^{n-i}a^i-x^{n-1-i}a^{i+1} \right)\\
            &= x^n+\sum_{i=1}^{n-1}x^{n-i}a^i-\sum_{i=0}^{n-2}x^{n-1-i}a^{i+1}-a^n\\
            &= x^n+\sum_{i=1}^{n-1}x^{n-i}a^i-\sum_{i=0+1}^{n-2+1}x^{n-1-(i-1)}a^{(i-1)+1}-a^n\\
            &= x^n+\sum_{i=1}^{n-1}x^{n-i}a^i-\sum_{i=1}^{n-1}x^{n-i}a^i-a^n\\
            &= x^n-a^n
        \end{align*}
        as desired.
    \end{proof}
\end{exercise}

\begin{exercise}\label{exr:12.8}\leavevmode
    \begin{enumerate}[label={(\alph*)}]
        \item Let $n\in\N$. Suppose $f:\R\to\R$ is given by $f(x)=x^n$. Use Exercise \ref{exr:12.7} to prove that $f'(a)=na^{n-1}$ for all $a\in\R$.
        \begin{proof}
            Let $a$ be an arbitrary element of $\R$. To prove that $f'(a)=na^{n-1}$, Theorem \ref{trm:12.4} tell us that it will suffice to show that $\lim_{x\to a}\frac{f(x)-f(a)}{x-a}=na^{n-1}$. By Corollary \ref{cly:11.12}, the polynomial $\sum_{i=0}^{n-1}x^{n-1-i}a^i$ is continuous. Thus, by Theorem \ref{trm:9.10}, it is continuous at $a$. It follows by Theorem \ref{trm:11.5} that $\lim_{x\to a}\sum_{i=0}^{n-1}x^{n-1-i}a^i=\sum_{i=0}^{n-1}a^{n-1-i}a^i$.\par
            Additionally, we can demonstrate that $\lim_{x\to a}\frac{x-a}{x-a}=1$. To verify that $\lim_{h\to 0}\frac{h}{h}=1$, Definition \ref{dfn:11.1} tells us that it will suffice to confirm that for all $\epsilon>0$, there exists a $\delta>0$ such that if $h\in\R$ and $0<|h-0|=|h|<\delta$, then $|\frac{h}{h}-1|<\epsilon$. Let $\epsilon>0$ be arbitrary. Choose $\delta=1$, and let $h$ be an arbitrary element of $\R$ satisfying $0<|h|<\delta$. It follows by Script \ref{sct:7} that $|\frac{h}{h}-1|=|1-1|=0<\epsilon$, as desired. Since $\lim_{h\to 0}\frac{h}{h}=1$, we know by Script \ref{sct:7} that $\lim_{h\to 0}\frac{(a+h)-a}{(a+h)-a}=1$. Thus, by Lemma \ref{lem:12.3}, $\lim_{x\to a}\frac{x-a}{x-a}=1$, as desired.\par
            It follows from the above two results that
            \begin{align*}
                na^{n-1} &= \underbrace{a^{n-1}+\cdots+a^{n-1}}_{n\text{ times}}\\
                &= \sum_{i=0}^{n-1}a^{n-1}\\
                &= \sum_{i=1}^{n-1}a^{n-1-i}a^i\\
                &= \lim_{x\to a}\sum_{i=0}^{n-1}x^{n-1-i}a^i\\
                &= 1\cdot\left( \lim_{x\to a}\sum_{i=0}^{n-1}x^{n-1-i}a^i \right)\\
                &= \left( \lim_{x\to a}\frac{x-a}{x-a} \right)\left( \lim_{x\to a}\sum_{i=0}^{n-1}x^{n-1-i}a^i \right)\\
                &= \lim_{x\to a}\frac{x-a}{x-a}\cdot\sum_{i=0}^{n-1}x^{n-1-i}a^i\tag*{Theorem \ref{trm:11.9}}\\
                &= \lim_{x\to a}\frac{x^n-a^n}{x-a}\tag*{Exercise \ref{exr:12.7}}\\
                &= \lim_{x\to a}\frac{f(x)-f(a)}{x-a}\\
                &= f'(a)\tag*{Theorem \ref{trm:12.4}}
            \end{align*}
            as desired.
        \end{proof}
        \item Let $k\in\R$. Prove that if $f:\R\to\R$ is given by $f(x)=k$, then $f'(a)=0$ for all $a\in\R$.
        \begin{proof}
            Let $a$ be an arbitrary element of $\R$. To prove that $f'(a)=0$, Definition \ref{dfn:12.1} tells us that it will suffice to show that $\lim_{h\to 0}\frac{f(a+h)-f(a)}{h}=0$. By Exercise \ref{exr:11.6}, the function $g:\R\to\R$ defined by $g(x)=0$ is continuous at every $x\in\R$, including 0. It follows by Theorem \ref{trm:11.5} that $\lim_{h\to 0}g(h)=g(0)=0$. Therefore,
            \begin{align*}
                0 &= \lim_{h\to 0}g(h)\\
                &= \lim_{h\to 0}0\\
                &= \lim_{h\to 0}\frac{0}{h}\\
                &= \lim_{h\to 0}\frac{k-k}{h}\\
                &= \lim_{h\to 0}\frac{f(a+h)-f(a)}{h}\\
                &= f'(a)\tag*{Definition \ref{dfn:12.1}}
            \end{align*}
            as desired.
        \end{proof}
    \end{enumerate}
\end{exercise}

\begin{exercise}\label{exr:12.9}
    Suppose that $f:A\to\R$ and $g:A\to\R$ are differentiable at $a\in A$.
    \begin{enumerate}[label={(\alph*)}]
        \item Prove that $f+g$ is differentiable at $a$ and compute $(f+g)'(a)$ in terms of $f'(a)$ and $g'(a)$.
        \begin{proof}
            To prove that $f+g$ is differentiable at $a$, Definition \ref{dfn:12.1} tells us that it will suffice to show $\lim_{h\to 0}\frac{(f+g)(a+h)-(f+g)(a)}{h}$ exists. Since $f,g$ are differentiable at $a$, we know by Definition \ref{dfn:12.1} that $\lim_{h\to 0}\frac{f(a+h)-f(a)}{h}$ and $\lim_{h\to 0}\frac{g(a+h)-g(a)}{h}$ exist. Thus, by Theorem \ref{trm:11.9} the limit of their sum exists and equals
            \begin{align*}
                \lim_{h\to 0}\left( \frac{f(a+h)-f(a)}{h}+\frac{g(a+h)-g(a)}{h} \right) &= \lim_{h\to 0}\frac{f(a+h)+g(a+h)-f(a)-g(a)}{h}\\
                &= \lim_{h\to 0}\frac{(f+g)(a+h)-(f+g)(a)}{h}
            \end{align*}
            as desired. Having established that $\lim_{h\to 0}\frac{(f+g)(a+h)-(f+g)(a)}{h}$ exists, $(f+g)'(a)$ can be computed in terms of $f'(a)$ and $g'(a)$ with the following algebra.
            \begin{align*}
                (f+g)'(a) &= \lim_{h\to 0}\frac{(f+g)(a+h)-(f+g)(a)}{h}\tag*{Definition \ref{dfn:12.1}}\\
                &= \lim_{h\to 0}\left( \frac{f(a+h)-f(a)}{h}+\frac{g(a+h)-g(a)}{h} \right)\\
                &= \lim_{h\to 0}\frac{f(a+h)-f(a)}{h}+\lim_{h\to 0}\frac{g(a+h)-g(a)}{h}\tag*{Theorem \ref{trm:11.9}}\\
                &= f'(a)+g'(a)\tag*{Definition \ref{dfn:12.1}}
            \end{align*}
        \end{proof}
        \item Prove that $fg$ is differentiable and compute $(fg)'(a)$ in terms of $f(a)$, $g(a)$, $f'(a)$, and $g'(a)$.
        \begin{proof}
            To prove that $fg$ is differentiable at $a$, Definition \ref{dfn:12.1} tells us that it will suffice to show $\lim_{h\to 0}\frac{(fg)(a+h)-(fg)(a)}{h}$ exists. Since $f,g$ are differentiable at $a$, we know by Definition \ref{dfn:12.1} that $\lim_{h\to 0}\frac{f(a+h)-f(a)}{h}$ and $\lim_{h\to 0}\frac{g(a+h)-g(a)}{h}$ exist. For the same reason, we know by Theorem \ref{trm:12.5} that $g$ is continuous, i.e., continuous at $a$ by Theorem \ref{trm:9.10}. Consequently, by Theorem \ref{trm:11.5}, $\lim_{x\to a}g(x)$ exists (and equals $g(a)$). Note that the preceding limit is equal to $\lim_{h\to 0}g(a+h)$ by Lemma \ref{lem:12.3}. Lastly, we have by Exercise \ref{exr:11.6} that the constant function $f(a)$ is continuous at 0. Consequently, by Theorem \ref{trm:11.5}, $\lim_{h\to 0}f(a)$ exists (and equals $f(a)$). Combining all of these results, consecutive applications of Theorem \ref{trm:11.9} assert that the limits
            \begin{align*}
                \lim_{h\to 0}g(a+h)\cdot\frac{f(a+h)-f(a)}{h}&&
                \lim_{h\to 0}f(a)\cdot\frac{g(a+h)-g(a)}{h}
            \end{align*}
            exist. Furthermore, it asserts that the limit of their sum exists and equals
            \begin{align*}
                \lim_{h\to 0}\left( g(a+h)\cdot\frac{f(a+h)-f(a)}{h}+f(a)\cdot\frac{g(a+h)-g(a)}{h} \right)\\
                &\hspace{-5cm}= \lim_{h\to 0}\frac{g(a+h)(f(a+h)-f(a))+f(a)(g(a+h)-g(a))}{h}\\
                &\hspace{-5cm}= \lim_{h\to 0}\frac{f(a+h)g(a+h)-f(a)g(a+h)+f(a)g(a+h)-f(a)g(a)}{h}\\
                &\hspace{-5cm}= \lim_{h\to 0}\frac{f(a+h)g(a+h)-f(a)g(a)}{h}\\
                &\hspace{-5cm}= \lim_{h\to 0}\frac{(fg)(a+h)-(fg)(a)}{h}
            \end{align*}
            as desired. Having established that $\lim_{h\to 0}\frac{(fg)(a+h)-(fg)(a)}{h}$ exists, $(fg)'(a)$ can be computed in terms of $f(a)$, $g(a)$, $f'(a)$, and $g'(a)$ with the following algebra.
            \begin{align*}
                (fg)'(a) &= \lim_{h\to 0}\frac{(fg)(a+h)-(fg)(a)}{h}\tag*{Definition \ref{dfn:12.1}}\\
                &= \lim_{h\to 0}\left( g(a+h)\cdot\frac{f(a+h)-f(a)}{h}+f(a)\cdot\frac{g(a+h)-g(a)}{h} \right)\\
                &= \lim_{h\to 0}g(a+h)\cdot\lim_{h\to 0}\frac{f(a+h)-f(a)}{h}+\lim_{h\to 0}f(a)\cdot\lim_{h\to 0}\frac{g(a+h)-g(a)}{h}\tag*{Theorem \ref{trm:11.9}}\\
                &= g(a)f'(a)+f(a)g'(a)
            \end{align*}
        \end{proof}
        \item Prove that $\frac{1}{g}$ is differentiable at $a$ (under an appropriate assumption) and compute $(\frac{1}{g})'(a)$ in terms of $g'(a)$ and $g(a)$. What assumption do you need to make?
        \begin{proof}
            Assume that $g(a)\neq 0$.\par
            To prove that $\frac{1}{g}$ is differentiable at $a$, Definition \ref{dfn:12.1} tells us that it will suffice to show that the limit $\lim_{h\to 0}\frac{(1/g)(a+h)-(1/g)(a)}{h}$ exists. Since $g$ is differentiable at $a$, we know by Definition \ref{dfn:12.1} that $\lim_{h\to 0}\frac{g(a+h)-g(a)}{h}$ exists. For the same reason, we know by Theorem \ref{trm:12.5} that $g$ is continuous, i.e., continuous at $a$ by Theorem \ref{trm:9.10}. Consequently, by Theorem \ref{trm:11.5}, $\lim_{x\to a}g(x)$ exists (and equals $g(a)$). It follows by Lemma \ref{lem:12.3} that the preceding limit is equal to $\lim_{h\to 0}g(a+h)$. Thus, since it is also equal to $g(a)\neq 0$, we have by Theorem \ref{trm:11.9} that $\lim_{h\to 0}\frac{1}{g}(a+h)$ exists (and equals $\frac{1}{g(a)}$). Lastly, we have by Exercise \ref{exr:11.6} that the constant function $-\frac{1}{g(a)}$ is continuous at 0. Consequently, by Theorem \ref{trm:11.5}, $\lim_{h\to 0}-\frac{1}{g(a)}$ exists (and equals $-\frac{1}{g(a)}$). Combining this with the previous result, Theorem \ref{trm:11.9} asserts that the limit $\lim_{h\to 0}-\frac{1}{g(a+h)g(a)}$ exists (and equals $-\frac{1}{g(a)^2}$). Furthermore, it asserts that the limit of its product with $\lim_{h\to 0}\frac{g(a+h)-g(a)}{h}$ exists and equals
            \begin{align*}
                \lim_{h\to 0}-\frac{1}{g(a+h)g(a)}\cdot\frac{g(a+h)-g(a)}{h} &= \lim_{h\to 0}\frac{\frac{g(a)-g(a+h)}{g(a+h)g(a)}}{h}\\
                &= \lim_{h\to 0}\frac{\frac{1}{g(a+h)}-\frac{1}{g(a)}}{h}\\
                &= \lim_{h\to 0}\frac{\frac{1}{g}(a+h)-\frac{1}{g}(a)}{h}
            \end{align*}
            as desired. Having established that $\lim_{h\to 0}\frac{(1/g)(a+h)-(1/g)(a)}{h}$ exists, $(\frac{1}{g})'(a)$ can be computed in terms of $g(a)$ and $g'(a)$ with the following algebra.
            \begin{align*}
                \left( \frac{1}{g} \right)'(a) &= \lim_{h\to 0}\frac{\frac{1}{g}(a+h)-\frac{1}{g}(a)}{h}\tag*{Definition \ref{dfn:12.1}}\\
                &= \lim_{h\to 0}-\frac{1}{g(a+h)g(a)}\cdot\frac{g(a+h)-g(a)}{h}\\
                &= \lim_{h\to 0}-\frac{1}{g(a+h)g(a)}\cdot\lim_{h\to 0}\frac{g(a+h)-g(a)}{h}\tag*{Theorem \ref{trm:11.9}}\\
                &= -\frac{g'(a)}{g(a)^2}
            \end{align*}
        \end{proof}
        \item Prove that $\frac{f}{g}$ is differentiable at $a$ (under an appropriate assumption) and compute $(\frac{f}{g})'(a)$ in terms of $f(a)$, $g(a)$, $f'(a)$, and $g'(a)$. What assumption do you need to make?
        \begin{proof}
            Assume that $g(a)\neq 0$.\par
            It follows by part (c) that $\frac{1}{g}$ is differentiable at $a$, and then by part (b) that $f\cdot\frac{1}{g}=\frac{f}{g}$ is differentiable at $a$.\par
            Having established that $(\frac{f}{g})'(a)$ exists, it can be computed in terms of $f(a)$, $g(a)$, $f'(a)$, and $g'(a)$ with the following algebra.
            \begin{align*}
                \left( \frac{f}{g} \right)'(a) &= \left( f\cdot\frac{1}{g} \right)'(a)\\
                &= f(a)\left( \frac{1}{g} \right)'(a)+f'(a)\left( \frac{1}{g} \right)(a)\\
                &= f(a)\cdot -\frac{g'(a)}{g(a)^2}+\frac{f'(a)g(a)}{g(a)^2}\\
                &= \frac{f'(a)g(a)-f(a)g'(a)}{g(a)^2}
            \end{align*}
        \end{proof}
    \end{enumerate}
\end{exercise}

\marginnote{4/1:}One of the most important results concerning the differentiation of functions is the rule for the derivative of a composition of functions. Let $f:B\to\R,g:A\to\R$ be functions such that $g(A)\subset B$. The composition $(f\circ g)(x)=f(g(x))$ is defined for all $x\in A$.

\begin{theorem}\label{trm:12.10}
    Let $a\in A$, $g:A\to\R$, and $f:I\to\R$ where $I$ is an interval containing $g(A)$. Suppose that $g$ is differentiable at $a$ and $f$ is differentiable at $g(a)$. Then $f\circ g$ is differentiable $a$ and
    \begin{equation*}
        (f\circ g)'(a) = f'(g(a))\cdot g'(a)
    \end{equation*}
    % \begin{lemma*}
    %     % Continuity of $\varphi$ as a lemma?

    %     Suppose that for all regions $R$ containing $a$, there exists a point $x\in R$ such that $g(x)\neq g(a)$. Then there exists a region $R$ containing $a$ such that $g(x)\neq g(a)$ for all $x\in R\setminus\{a\}$.
    %     \begin{proof}
    %         We divide into two cases ($g'(a)=0$ and $g'(a)\neq 0$).\par
    %         Suppose first that $g'(a)=0$. Then...
            
    %         % Using MVT: There are points $x$ arbitrarily close to 0 with $g'(x)\neq 0$. Around all of these points, there exists a region where $g(x)\neq g(a)$. The union of all of these regions is the desired region.

    %         % Theorem \ref{trm:12.5}: $g$ is continuous at $a$. Theorem \ref{trm:11.5}: $\lim_{x\to a}g(x)=g(a)$. We can opt to work with $S\subset A$ such that $a\in S$. There are points arbitrarily close to $a$ with $g(x)\neq g(a)$. At each of these points, 
            
    %         % \par
    %         Now suppose that $g'(a)\neq 0$. Then we divide into two subcases ($g'(a)>0$ and $g'(a)<0$). If $g'(a)>0$, then by Theorem \ref{trm:12.4}, $\lim_{x\to a}\frac{g(x)-g(a)}{x-a}>0$. Thus, by Lemma \ref{lem:11.8}, there exists a region $S$ with $a\in S$ such that $\frac{g(x)-g(a)}{x-a}>0$ for all $x\in S\cap A$. Additionally, since $A$ is open, Theorem \ref{trm:4.10} implies that there exists a region $T$ with $a\in T$ such that $T\subset A$. It follows by Theorem \ref{trm:3.18} that $S\cap T=R$ where $R$ is a region containing $a$. Choose this $R$ to be our $R$. By definition, $a\in R$. To prove that $g(x)\neq g(a)$ for all $x\in R\setminus\{a\}$, let $x$ be an arbitrary element of $R\setminus\{a\}$. Then by Definition \ref{dfn:1.11}, $x\in R$ and $x\neq a$. It follows from the former result by the above and Definition \ref{dfn:1.6} that $x\in S$ and $x\in T$. Since $x\in T$ and $T\subset A$, Definition \ref{dfn:1.3} implies that $x\in A$. This combined with the fact that $x\in S$ implies by Definition \ref{dfn:1.6} that $x\in S\cap A$. Thus, by the above, $\frac{g(x)-g(a)}{x-a}>0$. Therefore, by Script \ref{sct:7}, $g(x)-g(a)\neq 0$, i.e., $g(x)\neq g(a)$, as desired.
    %     \end{proof}
    % \end{lemma*}
    \begin{proof}
        % We divide into two cases (there exists a region $R$ with $a\in R$ such that $g(x)=g(a)$ for all $x\in R$; and for all regions $R$ with $a\in R$, there exists $x\in R$ such that $g(x)\neq g(a)$). If there exists a region $R$ with $a\in R$ such that $g(x)=g(a)$ for all $x\in R$, then $(f\circ g)(x)=(f\circ g)(a)$ for all $x\in R$. It follows by consecutive applications of Exercise \ref{exr:12.8} that $f\circ g$ is differentiable at $a$, $(f\circ g)'(a)=0$, and $g'(a)=0$. Therefore, we have that
        % \begin{align*}
        %     (f\circ g)'(a) &= 0\\
        %     &= f'(g(a))\cdot 0\\
        %     &= f'(g(a))\cdot g'(a)
        % \end{align*}
        % as desired.\par

        % We divide into two cases (there exists a region $(p,a)$ or $(a,q)$ such that $g(x)=g(a)$ for all $x$ in the region; and for all regions $R$ with $a\in\R$, there exist $x,y\in R$ such that $x<a<y$ and $g(x)\neq g(a)\neq g(y)$). Suppose first that there exists a region $(p,a)$ or $(a,q)$ such that $g(x)=g(a)$ for all $x$ in the region. Then we divide into two subcases (the region is of the form $(p,a)$ and the region is of the form $(a,q)$). If the region is of the form $(p,a)$, then we know that $(f\circ g)(x)=(f\circ g)(a)$ for all $x\in(p,a)$. It follows by an extension of Exercise \ref{exr:12.8} and Remark \ref{rmk:12.2} that $f\circ g$ is left-hand differentiable at $a$, $(f\circ g)_-'(a)=0$, and $g_-'(a)=0$, i.e., $g'(a)=0$ since $g$ is differentiable at $a$. Therefore, we have that
        % \begin{align*}
        %     (f\circ g)'(a) &= 0\\
        %     &= f'(g(a))\cdot 0\\
        %     &= f'(g(a))\cdot g'(a)
        % \end{align*}
        % as desired.\par
        % Now suppose that for all regions $R$ with $a\in\R$, there exist $x,y\in R$ such that $x<a<y$ and $g(x)\neq g(a)\neq g(y)$.

        % We now seek to use this result to verify that $\lim_{x\to a}\frac{f(g(x))-f(g(a))}{g(x)-g(a)}=f'(g(a))$. To do so, Definition \ref{dfn:11.1} tells us that it will suffice to show that for every $\epsilon>0$, there exists a $\delta>0$ such that if $x\in A$ and $0<|x-a|<\delta$, then $|\frac{f(g(x))-f(g(a))}{g(x)-g(a)}-f'(g(a))|<\epsilon$. Let $\epsilon>0$ be arbitrary. Since $\varphi$ is continuous at $a$, Theorem \ref{trm:11.5} asserts that there exists a $\delta_1>0$ such that if $x\in A$ and $|x-a|<\delta_1$, then $|\varphi(x)-\varphi(a)|<\epsilon$. Additionally, since it is true that for all regions $R$ with $a\in\R$, there exists $x\in R$ such that $g(x)\neq g(a)$, we have by the lemma that there exists a region $R$ with $a\in R$ such that $g(x)\neq g(a)$ for all $x\in R\setminus\{a\}$. It follows by Lemma \ref{lem:8.3} that $R$ is an interval and by Corollary \ref{cly:4.11} that $R$ is open. Thus, by Lemma \ref{lem:8.10}, there exists a number $\delta_2>0$ such that $(a-\delta_2,a+\delta_2)\subset R$. Choose $\delta=\min(\delta_1,\delta_2)$. Let $x$ be an arbitrary element of $A$ that satisfies $0<|x-a|<\delta$. It follows that $|x-a|<\delta_1$. Thus, by the above, $|\varphi(x)-\varphi(a)|<\epsilon$. Additionally, we have by the fact that $0<|x-a|<\delta$ that $|x-a|<\delta_2$ and $x\neq a$. It follows from the former result by Exercise \ref{exr:8.9} that $x\in(a-\delta_2,a+\delta_2)$, i.e., $x\in R$ by the above and Definition \ref{dfn:1.3}. It follows from the latter result by Script \ref{sct:1} that $x\notin\{a\}$. This combined with the previous result implies by Definition \ref{dfn:1.11} that $x\in R\setminus\{a\}$. Thus, by the above, $g(x)\neq g(a)$. This combined with the fact that $|\varphi(x)-\varphi(a)|<\epsilon$ implies by the definition of $\varphi$ that $|\frac{f(g(x))-f(g(a))}{g(x)-g(a)}-f'(g(a))|<\epsilon$, as desired.\par


        To prove that $f\circ g$ is differentiable at $a$, Theorem \ref{trm:12.4} tells us that it will suffice to show that $\lim_{x\to a}\frac{(f\circ g)(x)-(f\circ g)(a)}{x-a}$ exists. To do so, we will define a special function $\varphi$ and prove that it is continuous at $a$. It will follow that $(f\circ g)'(a)$ exists and equals $f'(g(a))\cdot g'(a)$. Let's begin.\par
        Let $\varphi:I\to\R$ be defined by
        \begin{equation*}
            \varphi(x) =
            \begin{cases}
                \frac{f(g(x))-f(g(a))}{g(x)-g(a)} & g(x)\neq g(a)\\
                f'(g(a)) & g(x)=g(a)
            \end{cases}
        \end{equation*}
        It is clear from the definition that the function is defined for all $x\in A$.\par
        To confirm that $\varphi$ is continuous at $a$, Theorem \ref{trm:11.5} tells us that it will suffice to demonstrate that for every $\epsilon>0$, there exists a $\delta>0$ such that if $x\in A$ and $|x-a|<\delta$, then $|\varphi(x)-\varphi(a)|=|\varphi(x)-f'(g(a))|<\epsilon$. Let $\epsilon>0$ be arbitrary. Since $f$ is differentiable at $g(a)$, Theorem \ref{trm:12.4} asserts that $\lim_{y\to g(a)}\frac{f(y)-f(g(a))}{y-g(a)}=f'(g(a))$. It follows by Definition \ref{dfn:11.1} that there is some $\delta'>0$ such that if $y\in I$ and $0<|y-g(a)|<\delta'$, then $|\frac{f(y)-f(g(a))}{y-g(a)}-f'(g(a))|<\epsilon$. Additionally, since $g$ is differentiable (hence continuous by Theorem \ref{trm:12.5}) at $a$, we have by Theorem \ref{trm:11.5} that there exists a $\delta>0$ such that if $x\in A$ and $|x-a|<\delta$, then $|g(x)-g(a)|<\delta'$.\par
        Using the above $\delta$, let $x$ be an arbitrary element of $A$ such that $|x-a|<\delta$. We now divide into two cases ($g(x)=g(a)$ and $g(x)\neq g(a)$). If $g(x)=g(a)$, then $|\varphi(x)-f'(g(a))|=|f'(g(a))-f'(g(a))|=0<\epsilon$, as desired. If $g(x)\neq g(a)$, then we continue. Since $|x-a|<\delta$, we have that $|g(x)-g(a)|<\delta'$. This combined with the fact that $g(x)\in I$ and $g(x)\neq g(a)$, i.e., $0<|g(x)-g(a)|$ illustrates that $|\frac{f(g(x))-f(g(a))}{g(x)-g(a)}-f'(g(a))|=|\varphi(x)-f'(g(a))|<\epsilon$. Therefore, $\varphi$ is continuous at $a$.\par
        It follows by Theorem \ref{trm:11.5} that $\lim_{x\to a}\varphi(x)=\varphi(a)=f'(g(a))$. Additionally, since $g$ is differentiable at $a$, Definition \ref{dfn:12.1} and Theorem \ref{trm:12.4} tell us that $\lim_{x\to a}\frac{g(x)-g(a)}{x-a}$ exists (and equals $g'(a)$). This combined with the previous result implies by Theorem \ref{trm:11.9} that the product of the limits exists and equals $f'(g(a))\cdot g'(a)$, i.e., we have that $\lim_{x\to a}\varphi(x)\cdot\frac{g(x)-g(a)}{x-a}=f'(g(a))\cdot g'(a)$.\par
        We now seek to confirm that $\lim_{x\to a}\frac{f(g(x))-f(g(a))}{x-a}=f'(g(a))\cdot g'(a)$. To do so, Definition \ref{dfn:11.1} tells us that it will suffice to demonstrate that for every $\epsilon>0$, there exists a $\delta>0$ such that if $x\in A$ and $0<|x-a|<\delta$, then $|\frac{f(g(x))-f(g(a))}{x-a}-f'(g(a))\cdot g'(a)|<\epsilon$. Let $\epsilon>0$ be arbitrary. Since $\lim_{x\to a}\varphi(x)\cdot\frac{g(x)-g(a)}{x-a}=f'(g(a))\cdot g'(a)$, we have by Definition \ref{dfn:11.1} that there exists a $\delta>0$ such that if $x\in A$ and $0<|x-a|<\delta$, then $|\varphi(x)\cdot\frac{g(x)-g(a)}{x-a}-f'(g(a))\cdot g'(a)|<\epsilon$. Choose this $\delta$ to be our $\delta$. Let $x$ be an arbitrary element of $A$ that satisfies $0<|x-a|<\delta$. We divide into two cases ($g(x)=g(a)$ and $g(x)\neq g(a)$). Suppose first that $g(x)=g(a)$. Since $0<|x-a|<\delta$, we have by the above and the definition of $\varphi$ that $|f'(g(a))\cdot\frac{g(x)-g(a)}{x-a}-f'(g(a))\cdot g'(a)|<\epsilon$. Additionally, it follows from the hypothesis that $g(x)-g(a)=0$ and $f(g(x))-f(g(a))=0$. Therefore,
        \begin{align*}
            \begin{split}
                \left| \frac{f(g(x))-f(g(a))}{x-a}-f'(g(a))\cdot g'(a) \right| \leq{}& \left| \frac{f(g(x))-f(g(a))}{x-a}-f'(g(a))\cdot\frac{g(x)-g(a)}{x-a} \right|\\
                &+ \left| f'(g(a))\cdot\frac{g(x)-g(a)}{x-a}-f'(g(a))\cdot g'(a) \right|
            \end{split}\tag*{Lemma \ref{lem:8.8}}\\
            \begin{split}
                ={}& \left| \frac{0}{x-a}-f'(g(a))\cdot\frac{0}{x-a} \right|\\
                &+ \left| f'(g(a))\cdot\frac{g(x)-g(a)}{x-a}-f'(g(a))\cdot g'(a) \right|
            \end{split}\\
            ={}& 0+\left| f'(g(a))\cdot\frac{g(x)-g(a)}{x-a}-f'(g(a))\cdot g'(a) \right|\\
            <{}& \epsilon
        \end{align*}
        as desired. Now suppose that $g(x)\neq g(a)$. Since $0<|x-a|<\delta$, we have by the above and the definition of $\varphi$ that $|\frac{f(g(x))-f(g(a))}{g(x)-g(a)}\cdot\frac{g(x)-g(a)}{x-a}-f'(g(a))\cdot g'(a)|<\epsilon$. Additionally, it follows from the hypothesis that $g(x)-g(a)\neq 0$. Therefore, we have by Script \ref{sct:7} that $|\frac{f(g(x))-f(g(a))}{x-a}-f'(g(a))\cdot g'(a)|<\epsilon$, as desired.\par
        It follows by Definition \ref{dfn:1.25} that $\lim_{x\to a}\frac{(f\circ g)(x)-(f\circ g)(a)}{x-a}=f'(g(a))\cdot g'(a)$. This proves that $f\circ g$ is differentiable at $a$. Lastly, it directly follows from Theorem \ref{trm:12.4} that
        \begin{equation*}
            (f\circ g)'(a) = \lim_{x\to a}\frac{(f\circ g)(x)-(f\circ g)(a)}{x-a}
            = f'(g(a))\cdot g'(a)
        \end{equation*}
        as desired.
    \end{proof}
\end{theorem}

We now come to the most important theorem in differential calculus, Corollary \ref{cly:12.16}.

\begin{definition}\label{dfn:12.11}
    Let $f:A\to\R$ be a function. If $f(a)$ is the last point of $f(A)$, then $f(a)$ is called the \textbf{maximum value} of $f$. If $f(a)$ is the first point of $f(A)$, then $f(a)$ is the \textbf{minimum value} of $f$. We say that $f(a)$ is a \textbf{local maximum value} of $f$ if there exists a region $R$ containing $a$ such that $f(a)$ is the last point of $f(A\cap R)$. We say that $f(a)$ is a \textbf{local minimum value} of $f$ if there exists a region $R$ containing $a$ such that $f(a)$ is the first point of $f(A\cap R)$.
\end{definition}

\begin{remark}\label{rmk:12.12}
    Equivalently, $f(a)$ is a local maximum (resp. minimum) value of $f$ if there exists $U$ open in $A$ such that $f(a)$ is the last (resp. first) point of $f(U)$.
\end{remark}

\begin{theorem}\label{trm:12.13}
    Let $f:A\to\R$ be differentiable at $a$. Suppose that $f(a)$ is the maximum value or minimum value of $f$. Then $f'(a)=0$.
    \begin{proof}
        Suppose first that $f(a)$ is the maximum value of $f$, and suppose for the sake of contradiction that $f'(a)\neq 0$. Then $f'(a)>0$ or $f'(a)<0$. We now divide into two cases. If $f'(a)>0$, then by Theorem \ref{trm:12.4}, $\lim_{x\to a}\frac{f(x)-f(a)}{x-a}>0$. Thus, by Lemma \ref{lem:11.8}, there exists a region $R$ with $a\in R$ such that $\frac{f(x)-f(a)}{x-a}>0$ for all $x\in R\cap A$. Additionally, since $A$ is open, Theorem \ref{trm:4.10} implies that there exists a region $S$ with $a\in S$ and $S\subset A$. It follows by Theorem \ref{trm:3.18} that $R\cap S=(c,d)$, where $(c,d)$ is a region containing $a$. Since $a\in(c,d)$, Theorem \ref{trm:5.2} implies that there exists a point $y\in\R$ such that $a<y<d$. Clearly, since $y\in(c,d)=R\cap S$ and $S\subset A$, we have that $y\in R\cap A$. It follows that $\frac{f(y)-f(a)}{y-a}>0$. Furthermore, $y>a$ implies that $y-a>0$. Therefore, by Definition \ref{dfn:7.21}, $f(y)-f(a)=\frac{f(y)-f(a)}{y-a}\cdot(y-a)>0$. But this means that $f(y)>f(a)$, i.e., that $f(a)$ is not the last point of $f(A)$ (by Definition \ref{dfn:3.3}), i.e., that $f(a)$ is not the maximum value of $f$ (by Definition \ref{dfn:12.11}), a contradiction. The argument is symmetric in the other case.\par
        The proof is symmetric in the other case.
    \end{proof}
\end{theorem}

\begin{corollary}\label{cly:12.14}
    Let $f:A\to\R$ be differentiable at $a$. Suppose that $f(a)$ is a local maximum or local minimum value of $f$. Then $f'(a)=0$.
    \begin{proof}
        Suppose first that $f(a)$ is a local maximum of $f$. Then by Definition \ref{dfn:12.11}, there exists a region $R$ containing $a$ such that $f(a)$ is the last point of $f(A\cap R)$. Now consider the restriction of $f$ to $A\cap R$. It follows from Definition \ref{dfn:9.6} that $f|_{A\cap R}$ is differentiable at $a$, that $f|_{A\cap R}(A\cap R)=f(A\cap R)$, and that $f|_{A\cap R}(a)=f(a)$ is the last point of $f|_{A\cap R}(A\cap R)$. The latter two results imply by Definition \ref{dfn:12.11} that $f|_{A\cap R}(a)$ is the maximum value of $f|_{A\cap R}$. This combined with the fact that $f|_{A\cap R}$ is differentiable at $a$ implies by Theorem \ref{trm:12.13} that $(f|_{A\cap R})'(a)=f'(a)=0$, as desired.
    \end{proof}
\end{corollary}

\begin{theorem}\label{trm:12.15}
    Suppose that $f:[a,b]\to\R$ is continuous, differentiable on $(a,b)$, and that $f(a)=f(b)=0$. Then there exists a point $\lambda\in(a,b)$ such that $f'(\lambda)=0$.
    \begin{proof}
        We divide into two cases ($f(x)=0$ for all $x\in[a,b]$, and $f(x)\neq 0$ for some $x\in[a,b]$).\par
        Suppose first that $f(x)=0$ for all $x\in[a,b]$. By Theorem \ref{trm:5.2}, we can choose a $\lambda\in(a,b)$. It follows from the hypothesis that $f(\lambda)=f(x)$ for all $f(x)\in f([a,b])$. This can be weakened to $f(\lambda)\geq f(x)$ for all $f(x)\in f([a,b])$. Thus, by Definition \ref{dfn:3.3}, $f(\lambda)$ is the last point of $f([a,b])$. Consequently, by Definition \ref{dfn:12.11}, $f(\lambda)$ is the maximum value of $f$. This combined with the fact that $f$ is differentiable at $\lambda$ (since $\lambda\in(a,b)$ and $f$ is differentiable on $(a,b)$) implies by Theorem \ref{trm:12.13} that $f'(\lambda)=0$, as desired.\par
        Now suppose that $f(x)\neq 0$ for some $x\in[a,b]$ which we shall call $x_0$. We divide into two cases again ($f(x_0)>0$ and $f(x_0)<0$). Suppose first that $f(x_0)>0$. Since $f:[a,b]\to\R$ is continuous, Exercise \ref{exr:10.21} asserts that there exists a point $\lambda\in[a,b]$ such that $f(\lambda)\geq f(x)$ for all $x\in[a,b]$. It follows that $f(\lambda)\geq f(x_0)>0$, so $f(\lambda)\neq f(a)=f(b)$. Thus, by Definition \ref{dfn:1.16}, $\lambda\neq a$ and $\lambda\neq b$. This combined with the fact that $\lambda\in[a,b]$ implies by Script \ref{sct:8} that $\lambda\in(a,b)$. Now as before, we can determine from the fact that $f(\lambda)\geq f(x)$ for all $x\in[a,b]$ that $f(\lambda)$ is the maximum value of $f$. This combined with the fact that $f$ is differentiable at $\lambda$ (since $\lambda\in(a,b)$ and $f$ is differentiable on $(a,b)$) implies by Theorem \ref{trm:12.13} that $f'(\lambda)=0$, as desired. The argument is symmetric in the other case.
    \end{proof}
\end{theorem}

\begin{corollary}\label{cly:12.16}
    Suppose that $f:[a,b]\to\R$ is continuous on $[a,b]$ and differentiable on $(a,b)$. Then there exists a point $\lambda\in(a,b)$ such that
    \begin{equation*}
        f(b)-f(a) = f'(\lambda)(b-a)
    \end{equation*}
    \begin{proof}
        Let $h:[a,b]\to\R$ be defined by
        \begin{equation*}
            h(x) = f(x)-\frac{f(b)-f(a)}{b-a}(x-a)-f(a)
        \end{equation*}
        By hypothesis, $f(x)$ is continuous on $[a,b]$. By Exercise \ref{exr:11.6} and Theorem \ref{trm:11.9}, $-\frac{f(b)-f(a)}{b-a}(x-a)-f(a)$ is continuous on $[a,b]$. Thus, by Theorem \ref{trm:11.9}, their sum (i.e., $h(x)$) is continuous on $[a,b]$. Additionally, by hypothesis, $f(x)$ is differentiable on $(a,b)$. By Exercises \ref{exr:12.8} and \ref{exr:12.9}, $-\frac{f(b)-f(a)}{b-a}(x-a)-f(a)$ is differentiable on $[a,b]$. Thus, by Exercise \ref{exr:12.9}, their sum (i.e., $h(x)$) is differentiable on $(a,b)$. Furthermore, by simple algebra, we can determine that
        \begin{align*}
            h(a) &= f(a)-\frac{f(b)-f(a)}{b-a}(a-a)-f(a)&
                h(b) &= f(b)-\frac{f(b)-f(a)}{b-a}(b-a)-f(a)\\
            &= -\frac{f(b)-f(a)}{b-a}\cdot 0&
                &= f(b)-(f(b)-f(a))-f(a)\\
            &= 0&
                &= 0
        \end{align*}
        Thus, by Theorem \ref{trm:12.15}, there exists a point $\lambda\in(a,b)$ such that $h'(\lambda)=0$.\par
        We can also calculate $h'(x)$ as follows.
        \begin{align*}
            h'(x) &= \left( f(x)-\frac{f(b)-f(a)}{b-a}(x-a)-f(a) \right)'\\
            &= \left( (f(x))+\left( -\frac{f(b)-f(a)}{b-a}\cdot x \right)+\left( \frac{f(b)-f(a)}{b-a}\cdot a-f(a) \right) \right)'\\
            &= f'(x)+\left( -\frac{f(b)-f(a)}{b-a}\cdot x \right)'+\left( \frac{f(b)-f(a)}{b-a}\cdot a-f(a) \right)'\tag*{Exercise \ref{exr:12.9}}\\
            &= f'(x)+\left( -\frac{f(b)-f(a)}{b-a} \right)'\cdot(x)+\left( -\frac{f(b)-f(a)}{b-a} \right)\cdot(x)'+\left( \frac{f(b)-f(a)}{b-a}\cdot a-f(a) \right)'\tag*{Exercise \ref{exr:12.9}}\\
            &= f'(x)+0\cdot x+\frac{f(b)-f(a)}{b-a}\cdot 1x^0+0\tag*{Exercise \ref{exr:12.8}}\\
            &= f'(x)+\frac{f(b)-f(a)}{b-a}
        \end{align*}
        But it follows that at $\lambda$,
        \begin{align*}
            0 &= f'(\lambda)-\frac{f(b)-f(a)}{b-a}\\
            \frac{f(b)-f(a)}{b-a} &= f'(\lambda)\\
            f(b)-f(a) &= f'(\lambda)(b-a)
        \end{align*}
        as desired.
    \end{proof}
\end{corollary}

\begin{corollary}\label{cly:12.17}\marginnote{\emph{4/6:}}
    Suppose that $f:[a,b]\to\R$ is continuous on $[a,b]$ and differentiable on $(a,b)$. Then
    \begin{enumerate}[label={\textup{(}\alph*\textup{)}}]
        \item If $f'(x)>0$ for all $x\in(a,b)$, then $f$ is strictly increasing on $[a,b]$.
        \begin{proof}
            To prove that $f$ is strictly increasing on $[a,b]$, Definition \ref{dfn:8.16} tells us that it will suffice to show that if $x,y\in[a,b]$ with $x<y$, then $f(x)<f(y)$. Let $x,y$ be arbitrary elements of $[a,b]$. WLOG, let $x<y$. Since $f$ is continuous  on $[a,b]$ and differentiable on $(a,b)$, Corollary \ref{cly:12.16} asserts that there exists a point $\lambda\in(x,y)$ such that $f(y)-f(x)=f'(\lambda)(y-x)$. But since $f'(\lambda)>0$ by hypothesis and $y-x>0$ because $y>x$, we have by Definition \ref{dfn:7.21} that $f'(\lambda)(y-x)>0$. It follows that $f(y)-f(x)>0$, i.e., that $f(x)<f(y)$, as desired.
        \end{proof}
        \item If $f'(x)<0$ for all $x\in(a,b)$, then $f$ is strictly decreasing on $[a,b]$.
        \begin{proof}
            The proof is symmetric to that of part (a).
        \end{proof}
        \item If $f'(x)=0$ for all $x\in(a,b)$, then $f$ is constant on $[a,b]$.
        \begin{proof}
            To prove that $f$ is constant on $[a,b]$, it will suffice to show that $f(x)=f(y)$ for all $x,y\in[a,b]$. Let $x,y$ be arbitrary elements of $[a,b]$. WLOG, let $x<y$. Since $f$ is continuous on $[a,b]$ and differentiable on $(a,b)$, Corollary \ref{cly:12.16} asserts that there exists a point $\lambda\in(x,y)$ such that $f(y)-f(x)=f'(\lambda)(y-x)$. But since $f'(\lambda)=0$ by hypothesis, $f(y)-f(x)=0$, i.e., $f(y)=f(x)$, as desired.
        \end{proof}
    \end{enumerate}
\end{corollary}

\begin{remark}\label{rmk:12.18}
    Corollary \ref{cly:12.17} also holds if instead of $[a,b]$, we have an arbitrary interval $I$; and instead of $(a,b)$, we have the interior of $I$.
\end{remark}

\begin{corollary}\label{cly:12.19}
    Suppose that $f:[a,b]\to\R$ and $g:[a,b]\to\R$ are continuous on $[a,b]$, differentiable on $(a,b)$, and $f'(x)=g'(x)$ for all $x\in(a,b)$. Then there is some $c\in\R$ such that for all $x\in[a,b]$, we have $f(x)=g(x)+c$.
    \begin{proof}
        Let $h:[a,b]\to\R$ be defined by $h(x)=f(x)-g(x)$. Since $f,g$ are continuous on $[a,b]$, Corollary \ref{cly:11.10} asserts that $h$ is continuous on $[a,b]$. Since $f,g$ are differentiable on $(a,b)$, Exercise \ref{exr:12.9} asserts that $h$ is differentiable on $(a,b)$. Since $f'(x)=g'(x)$ for all $x\in(a,b)$, Exercise \ref{exr:12.9} implies that $h'(x)=f'(x)-g'(x)=0$ for all $x\in(a,b)$. These three results satisfy the conditions of Corollary \ref{cly:12.17}, which means that $h$ is constant on $[a,b]$, i.e., that $h(x)=c$ for all $x\in[a,b]$ where $c\in\R$. But by the definition of $h$, this implies that for all $x\in[a,b]$, we have $f(x)-g(x)=c$, i.e., $f(x)=g(x)+c$.
    \end{proof}
\end{corollary}

\begin{corollary}\label{cly:12.20}
    Suppose that $f:[a,b]\to\R$ and $g:[a,b]\to\R$ are continuous on $[a,b]$ and differentiable on $(a,b)$. Then there is a point $\lambda\in(a,b)$ such that
    \begin{equation*}
        (f(b)-f(a))g'(\lambda) = (g(b)-g(a))f'(\lambda)
    \end{equation*}
    \begin{proof}
        Let $h:[a,b]\to\R$ be defined by $h(x)=(g(b)-g(a))f(x)-(f(b)-f(a))g(x)-f(a)g(b)+f(b)g(a)$. For the same reasons as in the proof of Corollary \ref{cly:12.19}, $h$ is continuous on $[a,b]$ and differentiable on $(a,b)$. Additionally, we can show with basic algebra that
        \begin{align*}
            h(a) &= (g(b)-g(a))f(a)-(f(b)-f(a))g(a)-f(a)g(b)+f(b)g(a)\\  
            &= f(a)g(b)-f(a)g(a)-f(b)g(a)+f(a)g(a)-f(a)g(b)+f(b)g(a)\\
            &= 0
        \end{align*}
        \begin{align*}
            h(b) &= (g(b)-g(a))f(b)-(f(b)-f(a))g(b)-f(a)g(b)+f(b)g(a)\\
            &= f(b)g(b)-f(b)g(a)-f(b)g(b)+f(a)g(b)-f(a)g(b)+f(b)g(a)\\
            &= 0
        \end{align*}
        These results satisfy the conditions of Theorem \ref{trm:12.15}, which means that there exists a point $\lambda\in(a,b)$ such that $h'(\lambda)=0$. We can also calculate that $h'(x)=(g(b)-g(a))f'(x)-(f(b)-f(a))g'(x)$ via a similar method to that used in the proof of Corollary \ref{cly:12.16}. But it follows that at $\lambda$,
        \begin{align*}
            0 &= (g(b)-g(a))f'(\lambda)-(f(b)-f(a))g'(\lambda)\\
            (f(b)-f(a))g'(\lambda) &= (g(b)-g(a))f'(\lambda)
        \end{align*}
    \end{proof}
\end{corollary}

Finally, we prove another very important theorem that tells us about inverse functions and their derivatives.

\begin{theorem}\label{trm:12.21}
    Suppose that $f:(a,b)\to\R$ is differentiable and that the derivative $f':(a,b)\to\R$ is continuous. Also suppose that there is a point $p\in(a,b)$ such that $f'(p)\neq 0$. Then there exists a region $R\subset(a,b)$ such that $p\in R$ and $f$ with domain restricted to $R$ is injective. Furthermore, $f^{-1}:f(R)\to R$ is differentiable at the point $f(p)$ and
    \begin{equation*}
        (f^{-1})'(f(p)) = \frac{1}{f'(p)}
    \end{equation*}
    \begin{proof}
        % \begin{itemize}
        %     \item WTS: $(f^{-1})'(f(p))$ exists and $=\frac{1}{f'(p)}$.
        %     \item Theorem \ref{trm:12.4}: Will suffice to show $\lim_{y\to f(p)}\frac{f^{-1}(y)-f^{-1}(f(p))}{y-f(p)}=\frac{1}{f'(p)}$.
        %     \item Theorem \ref{dfn:12.1}: Will suffice to confirm that for all $\epsilon>0$, there exists a $\delta>0$ such that if $y\in f(R)$ and $0<|y-f(p)|<\delta$, then $|\frac{f^{-1}(y)-f^{-1}(f(p))}{y-f(p)}-\frac{1}{f'(p)}|<\epsilon$.
        %     \item Let $\epsilon>0$ be arbitrary.
        %     \item Various:
        %     \begin{align*}
        %         \frac{1}{f'(p)} &= \frac{1}{\lim_{x\to p}\frac{f(x)-f(p)}{x-p}}\tag*{Theorem \ref{trm:12.4}}\\
        %         &= \lim_{x\to p}\frac{x-p}{f(x)-f(p)}\tag*{Theorem \ref{trm:11.9}}\\
        %         &= \lim_{x\to p}\frac{f^{-1}(f(x))-f^{-1}(f(p))}{f(x)-f(p)}\tag*{Definition \ref{dfn:1.18}}
        %     \end{align*}
        %     \item Definition \ref{dfn:11.1}: There exists $\delta'>0$ such that if $x\in R$ and $0<|x-p|<\delta'$, then $|\frac{f^{-1}(f(x))-f^{-1}(f(p))}{f(x)-f(p)}-\frac{1}{f'(p)}|<\epsilon$.
        %     \item \emph{Paragraph break/lane change}.
        %     \item Theorem \ref{trm:9.10}: $f^{-1}$ is continuous at $f(p)$.
        %     \item Theorem \ref{trm:11.5}: Either $f(p)\notin LP(f(R))$ or $\lim_{y\to f(p)}f^{-1}(y)=f^{-1}(f(p))$.
        %     \item $f(p)\in LP(f(R))$.
        %     \begin{itemize}
        %         \item Lemma \ref{lem:8.3}: $R$ is an interval.
        %         \item Theorem \ref{trm:8.15}: $R$ is connected.
        %         \item Theorem \ref{trm:9.11}: $f(R)$ is connected.
        %         \item Theorem \ref{trm:8.15}: $f(R)$ is an interval.
        %         \item Extension of Corollaries \ref{cly:5.5} and \ref{cly:5.14}: $f(p)\in LP(f(R))$.
        %     \end{itemize}
        %     \item Thus: $\lim_{y\to f(p)}f^{-1}(y)=f^{-1}(f(p))$.
        %     \item Definition \ref{dfn:11.1}: There exists a $\delta>0$ such that if $y\in f(R)$ and $0<|y-f(p)|<\delta$, then $|f^{-1}(y)-f^{-1}(f(p))|<\delta'$.
        %     \item Definition \ref{dfn:1.18}: There exists a $\delta>0$ such that if $y\in f(R)$ and $0<|y-f(p)|<\delta$, then $|x-p|<\delta'$.
        %     \item Choose this as our $\delta$. Let $y=f(x)$ be an arbitrary element of $f(R)$ such that $0<|y-f(p)|<\delta$.
        %     \item Then $|x-p|<\delta'$.
        %     \item $0<|x-p|$.
        %     \begin{itemize}
        %         \item Script \ref{sct:8}: $0<|y-f(p)|$ implies $f(x)\neq f(p)$.
        %         \item Definition \ref{dfn:1.20}: $x\neq p$.
        %         \item Script \ref{sct:8}: $0<|x-p|$.
        %     \end{itemize}
        %     \item Definition \ref{dfn:1.18}: $x\in R$.
        %     \item Thus, $x\in R$ and $0<|x-p|<\delta'$. Consequently, $|\frac{f^{-1}(f(x))-f^{-1}(f(p))}{f(x)-f(p)}-\frac{1}{f'(p)}|<\epsilon$.
        %     \item Definition \ref{dfn:1.18}: $|\frac{f^{-1}(y)-f^{-1}(f(p))}{y-f(p)}-\frac{1}{f'(p)}|<\epsilon$.
        %     % \item Various:
        %     % \begin{align*}
        %     %     (f^{-1})'(f(p)) &= \lim_{y\to f(p)}\frac{f^{-1}(y)-f^{-1}(f(p))}{y-f(p)}\tag*{Theorem \ref{trm:12.4}}\\
        %     %     &= \lim_{f(x)\to f(p)}\frac{f^{-1}(f(x))-f^{-1}(f(p))}{f(x)-f(p)}\tag*{Script \ref{sct:1}}\\
        %     %     &= \lim_{f(x)\to f(p)}\frac{x-p}{f(x)-f(p)}\tag*{Script \ref{sct:1}}
        %     % \end{align*}
        % \end{itemize}

        Since $f':(a,b)\to\R$ is continuous, Theorem \ref{trm:9.10} implies that it is continuous at $p\in(a,b)$. Thus, by Theorem \ref{trm:11.5}, $\lim_{x\to p}f'(x)=f'(p)$. This combined with the facts that $f'(p)\neq 0$ (i.e., $f'(p)>0$ or $f'(p)<0$) and $f'$ is continuous at $p$ implies by Lemma \ref{lem:11.8} that there exists a region $R$ with $p\in R$ such that $f'(x)>0$ for all $x\in R\cap(a,b)$ or $f'(x)<0$ for all $x\in R\cap(a,b)$. Clearly, $R\cap(a,b)=R$. Consequently, since $f'(x)>0$ for all $x\in R$ or $f'(x)<0$ for all $x\in R$, Corollary \ref{cly:12.17} asserts that $f$ is strictly increasing or strictly decreasing on $R$. Since $R$ is an interval by Lemma \ref{lem:8.3}, the previous result implies by Lemma \ref{lem:8.17} that $f$ is injective on $R$. Therefore, we have found an $R=R\cap(a,b)\subset(a,b)$ by Theorem \ref{trm:1.7} with $p\in R$ such that $f|_R$ is injective, as desired.\par
        From now on, we will denote $f|_R:R\to\R$ by $f$. Since $f$ is differentiable, Definition \ref{dfn:12.1} asserts that it is differentiable for all $x\in R$. Thus, by Theorem \ref{trm:12.5}, $f$ is continuous for all $x\in R$. Consequently, by Theorem \ref{trm:9.10}, $f$ is continuous. This combined with the previously proven fact that $f$ is injective implies by Theorem \ref{trm:9.14} that the inverse function $f^{-1}:f(R)\to R$ exists and is continuous.\par
        To prove that $(f^{-1})'(f(p))$ exists and equals $\frac{1}{f'(p)}$, Theorem \ref{trm:12.4} tells us that it will suffice to show that $\lim_{y\to f(p)}\frac{f^{-1}(y)-f^{-1}(f(p))}{y-f(p)}=\frac{1}{f'(p)}$. To do this, Definition \ref{dfn:11.1} tells us that it will suffice to confirm that for all $\epsilon>0$, there exists a $\delta>0$ such that if $y\in f(R)$ and $0<|y-f(p)|<\delta$, then $|\frac{f^{-1}(y)-f^{-1}(f(p))}{y-p}-\frac{1}{f'(p)}|<\epsilon$. Let $\epsilon>0$ be arbitrary. We have that
        \begin{align*}
            \frac{1}{f'(p)} &= \frac{1}{\lim_{x\to p}\frac{f(x)-f(p)}{x-p}}\tag*{Theorem \ref{trm:12.4}}\\
            &= \lim_{x\to p}\frac{x-p}{f(x)-f(p)}\tag*{Theorem \ref{trm:11.9}}\\
            &= \lim_{x\to p}\frac{f^{-1}(f(x))-f^{-1}(f(p))}{f(x)-f(p)}\tag*{Definition \ref{dfn:1.18}}
        \end{align*}
        Thus, by Definition \ref{dfn:11.1}, there exists a $\delta'>0$ such that if $x\in R$ and $0<|x-p|<\delta'$, then $|\frac{f^{-1}(f(x))-f^{-1}(f(p))}{f(x)-f(p)}-\frac{1}{f'(p)}|<\epsilon$.\par
        The previously proven fact that $f^{-1}$ is continuous implies by Theorem \ref{trm:9.10} that $f^{-1}$ is continuous at $f(p)$. It follows by Theorem \ref{trm:11.5} that either $f(p)\notin LP(f(R))$ or $\lim_{y\to f(p)}f^{-1}(y)=f^{-1}(f(p))$. However, as we will now see, $f(p)\in LP(f(R))$. To begin, the fact that $R$ is a region implies by Lemma \ref{lem:8.3} that $R$ is an interval. It follows by Theorem \ref{trm:8.15} that $R$ is connected, by Theorem \ref{trm:9.11} that $f(R)$ is connected, by Theorem \ref{trm:8.15} again that $f(R)$ is an interval, and finally by extensions of Corollaries \ref{cly:5.5} and \ref{cly:5.14} that $f(p)\in LP(f(R))$. Thus, with this case eliminated, we know that $\lim_{y\to f(p)}f^{-1}(y)=f^{-1}(f(p))$. Consequently, by Definition \ref{dfn:1.11}, there exists a $\delta>0$ such that if $y\in f(R)$ and $0<|y-f(p)|<\delta$, then $|f^{-1}(y)-f^{-1}(f(p))|<\delta'$. With a slight modification from Definition \ref{dfn:1.18} (and the definition that $y=f(x)$), we have that there exists a $\delta>0$ such that if $y\in f(R)$ and $0<|y-f(p)|<\delta$, then $|x-p|<\delta'$.\par
        Choose the above $\delta$ as our $\delta$. Let $y=f(x)$ be an arbitrary element of $f(R)$ satisfying $0<|y-f(p)|<\delta$. Then $|x-p|<\delta'$. We can also show that $0<|x-p|$: From Script \ref{sct:8}, the fact that $0<|y-f(p)|$ implies that $f(x)\neq f(p)$; hence by Definition \ref{dfn:1.20} and the fact that $f$ is injective, $x\neq p$; hence by Script \ref{sct:8} again, $0<|x-p|$. Continuing, since $y=f(x)\in f(R)$, we have by Definition \ref{dfn:1.18} that $x\in R$. Thus, we have that $x\in R$ and $0<|x-p|<\delta'$, so we know that $|\frac{f^{-1}(f(x))-f^{-1}(f(p))}{f(x)-f(p)}-\frac{1}{f'(p)}|<\epsilon$. Therefore, with a slight modification from Definition \ref{dfn:1.18}, we have that $|\frac{f^{-1}(y)-f^{-1}(f(p))}{y-f(p)}-\frac{1}{f'(p)}|<\epsilon$, as desired.
    \end{proof}
\end{theorem}

\begin{exercise}\label{exr:12.22}\marginnote{4/8:}
    Consider the function $f(x)=x^n$ for a fixed $n\in\N$. Show that if $n$ is even, then $f$ is strictly increasing on the set of nonnegative real numbers and that if $n$ is odd, then $f$ is strictly increasing on all of $\R$. For a given $n$, let $A$ be the aforementioned set on which $f$ is strictly increasing. Define the inverse function $f^{-1}:f(A)\to A$ by $f^{-1}(x)=\sqrt[n]{x}$, which we sometimes also denote $f^{-1}(x)=x^{1/n}$. Use Theorem \ref{trm:12.21} to find the points $y\in f(A)$ at which $f^{-1}$ is differentiable, and determine $(f^{-1})'(y)$ at these points.
    \begin{proof}
        % $x^n$ is strictly increasing on $\R^+$ and $f'(0)=0$, so $f'(x)>0$ for all $x\in(0,\infty)$. \ref{cly:12.17}: $f$ is strictly increasing.
        % \begin{itemize}
        %     % \item Inductive hypothesis: $x^n$ is strictly increasing on $\R^+$.
        %     \item Let $n$ be an arbitrary natural number.
        %     \item Divide into two cases ($n$ is even; $n$ is odd).
        %     \item If $n$ is even, WTS: Show that $f(x)=x^{n+1}$ is strictly increasing on $\R^+$.
        %     \item Lemma \ref{lem:8.17}: $x^n$ is injective on $\R^+$.
        %     \item $f'$ is injective on $\R^+$:
        %     \begin{align*}
        %         f'(x) &= f'(y)\\
        %         (n+1)x^n &= (n+1)y^n\\
        %         x^n &= y^n\tag*{Script \ref{sct:7}}\\
        %         x &= y
        %     \end{align*}
        %     \item Corollary \ref{cly:11.12}: $f'$ is continuous.
        %     \item Lemma \ref{lem:9.13}: $f'$ is strictly increasing or strictly decreasing.
        %     \item $f'$ is strictly increasing.
        %     \begin{itemize}
        %         \item 
        %     \end{itemize}
        %     \item Script \ref{sct:7}: $f'(0)=0$.
        % \end{itemize}

        For the first part of the question, we must prove that $f$ is strictly increasing on $\R^+$ if $n$ is even and that $f$ is strictly increasing on $\R$ if $n$ is odd. To do so, we induct on $n$. For the base case $n=1$, we must confirm that $f(x)=x$ is strictly increasing on $\R$ (since $n$ is odd). By Exercise \ref{exr:12.8}, $f'(x)=1>0$ for all $x\in\R$. Thus, by Corollary \ref{cly:12.17} and Remark \ref{rmk:12.18}, $f$ is strictly increasing on $\R$, as desired. Now suppose inductively that the claim holds for some natural number $n\in\N$; we wish to confirm it for $n+1$. We divide into two cases ($n+1$ is even and $n+1$ is odd).\par\smallskip
        If $n+1$ is even, then we must confirm that $f(x)=x^{n+1}$ is strictly increasing on $\R^+$. By Exercise \ref{exr:12.8}, $f'(x)=(n+1)x^n$, where $n$ is odd. To verify that $f'$ is strictly increasing on $\R^+$, Definition \ref{dfn:8.16} tells us that it will suffice to demonstrate that for all $x,y\in\R^+$ satisfying $x<y$, $f'(x)<f'(y)$. Let $x,y$ be arbitrary elements of $\R^+$ that satisfy $x<y$. Since $x^n$ is strictly increasing on $\R^+\subset\R$ by the inductive hypothesis, Definition \ref{dfn:8.16} tells us that $x^n<y^n$. This combined with the fact that $0<n+1$ by Script \ref{sct:0} implies by Lemma \ref{lem:7.24} that $(n+1)x^n<(n+1)y^n$. Thus, by the definition of $f'$, $f'(x)<f'(y)$, as desired. Note also that $f'(0)=(n+1)0^n=0$. Having established that $f'$ is strictly increasing on $\R^+$ and that $f'(0)=0$, we have that $f'(x)>0$ for all $x\in(0,\infty)$ by Definition \ref{dfn:8.16} because $0<x$ implies $0=f'(0)<f'(x)$. Thus, by Corollary \ref{cly:12.17} and Remark \ref{rmk:12.18}, $f$ is strictly increasing on $\R^+$, as desired.\par\smallskip
        If $n+1$ is odd, then we must confirm that $f(x)=x^{n+1}$ is strictly increasing on $\R$. By Exercise \ref{exr:12.8}, $f'(x)=(n+1)x^n$, where $n$ is even. With a symmetric argument to that used above, we can verify that $f'$ is strictly increasing on $\R^+$. Since $f'(-x)=(n+1)(-x)^n=(n+1)x^n=f'(x)$ by Script \ref{sct:7}, we can similarly prove that $f'$ is strictly \emph{decreasing} on $\R^-$ (essentially, if $x,y\in\R^-$, then $x<y$ implies $-y<-x$ implies $f(-y)<f(-x)$ implies $f(x)>f(y)$, as desired). These two results combined with the fact that we still have $f'(0)=0$ imply by a symmetric argument to the above that $f'(x)>0$ for all $x\in(-\infty,0)\cup(0,\infty)$. But it follows by consecutive applications of Corollary \ref{cly:12.17} and Remark \ref{rmk:12.18} that $f$ is strictly increasing on $\R$, as desired.\par\medskip
        For the second part of the question, we must find all of the points $y$ where $f^{-1}$ is differentiable and determine the derivative $(f^{-1})'(y)$ at these points for an arbitrary $n$. Let $n$ be an arbitrary element of $\N$. By Exercise \ref{exr:12.8}, $f$ is differentiable, and by both Exercise \ref{exr:12.8} and Corollary \ref{cly:11.12}, $f'$ is continuous. We divide into two cases ($n$ is even and $n$ is odd).\par\smallskip
        Suppose first that $n$ is even. To begin, we will verify that $f(\R^+)=\R^+$. By Definition \ref{dfn:1.2}, to do so it will suffice to show that every $y\in f(\R^+)$ is an element of $\R^+$ and vice versa. Let $y$ be an arbitrary element of $f(\R^+)$. Then by Definition \ref{dfn:1.18}, $y=f(x)$ for some $x\in\R^+$. It follows since $x\in\R^+$ that $x\geq 0$. Consequently, since $f$ is strictly increasing on $\R^+$, Definition \ref{dfn:8.16} implies that $f(x)\geq f(0)=0$. But if $y=f(x)\geq 0$, then $y\in\R^+$, as desired. Now let $y$ be an arbitrary element of $\R^+$. We divide into three cases ($y=0$, $0<y\leq 1$ and $1<y$). If $y=0$, then since $f(0)=0^n=0=y$ and $0\in\R^+$, $y=f(x)$ for an $x\in\R^+$. Therefore, Definition \ref{dfn:1.18} asserts that $y\in f(\R^+)$, as desired. If $0<y\leq 1$, then since $f$ is continuous (notably on $[0,2]$) and $f(0)=0<y<2^n=f(2)$ (by Script \ref{sct:7}), Exercise \ref{exr:9.12} asserts that there exists a point $x\in\R$ with $0<x<2$ such that $f(x)=y$. Therefore, since $y=f(x)$ for an $x\in\R^+$ (we do know that $x>0$), Definition \ref{dfn:1.18} asserts that $y\in f(\R^+)$, as desired. If $y>1$, then since $f(0)=0<1<y<y^n=f(y)$ (by Script \ref{sct:7}), Exercise \ref{exr:9.12} asserts that there exists a point $x\in\R$ with $0<x<y$ such that $f(x)=y$. Therefore, since $y=f(x)$ for an $x\in\R^+$ (we do know that $x>0$), Definition \ref{dfn:1.18} asserts that $y\in f(\R^+)$, as desired.\par
        Having established that $f(\R^+)=\R^+$, our task becomes one of finding all points $y\in\R^+$ at which $f^{-1}$ is differentiable, and determining $(f^{-1})'(y)$ at these points. By Theorem \ref{trm:12.21}, this means that we need only find all points $f(p)$ corresponding to a $p$ that satisfies $f'(p)\neq 0$. Since $f'(x)=nx^{n-1}$ by Exercise \ref{exr:12.8}, Script \ref{sct:7} implies that the only point $p$ where $f'(p)=0$ is $p=0$. Thus, we need only exclude $f(0)=0$ from our set of points at which $f^{-1}$ is differentiable. Therefore, we know that $f^{-1}$ is differentiable at every $y\in\R^+$ such that $y\neq 0$, or more simply, all $y$ in the interval $(0,\infty)$. Additionally, Theorem \ref{trm:12.21} implies that for any $y\in(0,\infty)$,
        \begin{align*}
            (f^{-1})'(y) &= \frac{1}{f'(f^{-1}(y))}\\
            &= \frac{1}{n(f^{-1}(y))^{n-1}}\\
            &= \frac{1}{n(y^{1/n})^{n-1}}\\
            &= \frac{1}{n}\cdot y^{\frac{1-n}{n}}
        \end{align*}\par\smallskip
        The proof is symmetric if $n$ is odd.
        
        
        % then by Theorem \ref{trm:12.21}, $f^{-1}$ is differentiable at all points $f(p)\in f(\R^+)$ corresponding to a $p$ that satisfies $f'(p)\neq 0$. $f^{-1}:f(\R^+)\to\R^+$.
        % If $n$ is even, then to find the points $f(p)$ where $f^{-1}$ is differentiable and determine the derivative $(f^{-1})'(f(p))$ at these points, Theorem \ref{trm:12.21} tells us that it will suffice to locate all points $f(p)$ corresponding to a $p$ that satisfies $f'(p)\neq 0$.
    \end{proof}
\end{exercise}

\subsection*{Additional Exercises}
\begin{enumerate}[ref={\thechapter.\arabic*}]
    \setcounter{enumi}{2}
    \item \label{axr:12.3}
    \begin{enumerate}[ref={\theenumi\alph*}]
        \item \label{axr:12.3a}Suppose $f:[a,b]\to\R$, $x_0\in(a,b)$, $f',f''$ exist and are continuoous oon soome interval containing $x_0$, and $f'(x_0)=0$, but $f''(x_0)\neq 0$. Prove that
        \begin{enumerate}[label={\roman*)},ref={\theenumii-\roman*}]
            \item \label{axr:12.3a-i}if $f''(x_0)>0$, then $f$ has a local minimum at $x_0$;
            \item \label{axr:12.3a-ii}if $f''(x_0)<0$, then $f$ has a local maximum at $x_0$.
        \end{enumerate}
        \item \label{axr:12.3b}\textbf{Bernoulli's inequality}: Prove that for $x\geq -1$ and $\alpha\geq 1$,
        \begin{equation*}
            (1+x)^\alpha \geq 1+\alpha x
        \end{equation*}
        and for $x\geq -1$ and $0\leq\alpha\leq 1$,
        \begin{equation*}
            (1+x)^\alpha \leq 1+\alpha x
        \end{equation*}
        \item \label{axr:12.3c}Prove that $x-\frac{x^3}{6}<\sin x<x$ for all $x>0$.
    \end{enumerate}
\end{enumerate}




\end{document}