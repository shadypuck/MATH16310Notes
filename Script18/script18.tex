\documentclass[../main.tex]{subfiles}

\pagestyle{main}
\renewcommand{\chaptermark}[1]{\markboth{\chaptername\ \thechapter}{}}
\setcounter{chapter}{17}

\begin{document}




\chapter[The Euclidean Space \texorpdfstring{$\protect\fakeboldb{\R}^{\bm{n}}$}{TEXT}]{The Euclidean Space \texorpdfstring{$\protect\fakeboldc{\R}^{\bm{n}}$}{TEXT}}\label{sct:18}
\marginnote{7/7:}For the next three sheets, we will be studying multivariable calculus, that is "calculus on $\R^n$." First, we need to understand the space $\R^n$.

\begin{definition}\label{dfn:18.1}
    The \textbf{Euclidean $\bm{n}$-space} $\R^n$ is the $n$-fold Cartesian product of $\R$. Symbolically,
    \begin{equation*}
        \R^n = \{(x_1,\dots,x_n)\mid x_1,\dots,x_n\in\R\}
    \end{equation*}
    is the set of $n$-tuples of real numbers. We often write
    \begin{equation*}
        \x = (x_1,\dots,x_n)
    \end{equation*}
    to denote an element, which is also referred to as a \textbf{vector}, in $\R^n$ and
    \begin{equation*}
        \bm{0} = (0,\dots,0)
    \end{equation*}
\end{definition}

\begin{definition}\label{dfn:18.2}
    Let $\x=(x_1,\dots,x_n),\y=(y_1,\dots,y_n)\in\R^n$ and $\lambda\in\R$. We define the following operations.
    \begin{enumerate}[label={(\alph*)}]
        \item (Addition) $\x+\y=(x_1+y_1,\dots,x_n+y_n)$.
        \item (Scalar Multiplication) $\lambda\x=(\lambda x_1,\dots,\lambda x_n)$.
    \end{enumerate}
\end{definition}

\begin{exercise}\label{exr:18.3}
    Prove that the addition on $\R^n$ satisfies FA1-FA4 (see Definition \ref{dfn:7.8}). Moreover, prove that
    \begin{enumerate}[label={VS\arabic*.}]
        \item (Associativity of Scalar Multiplication) If $\lambda,\mu\in\R$ and $\x\in\R^n$, then $(\lambda\mu)\x=\lambda(\mu\x)$.
        \item (Distributivity of Scalars) If $\lambda,\mu\in\R$ and $\x\in\R^n$, then $(\lambda+\mu)\x=\lambda\x+\mu\x$.
        \item (Distributivity of Vectors) If $\lambda\in\R$ and $\x,\y\in\R^n$, then $\lambda(\x+\y)=\lambda\x+\lambda\y$.
        \item (Scalar Multiplicative Identity) If $\x\in\R^n$, then $1\x=\x$.
    \end{enumerate}
    These eight properties together are called the \textbf{vector space axioms}.
    \begin{proof}
        To prove that $\R^n$ obeys FA1 from Definition \ref{dfn:7.8}, it will suffice to show that for all $\x,\y\in\R^n$, $\x+\y=\y+\x$. Let $\x,\y$ be arbitrary elements of $\R^n$. Then by Definition \ref{dfn:18.2},
        \begin{align*}
            \x+\y &= (x_1+y_1,\dots,x_n+y_n)\\
            &= (y_1+x_1,\dots,y_n+x_n)\\
            &= \y+\x
        \end{align*}
        as desired.\par
        To prove that $\R^n$ obeys FA2 from Definition \ref{dfn:7.8}, it will suffice to show that for all $\x,\y,\z\in\R^n$, $(\x+\y)+\z=\x+(\y+\z)$. Let $\x,\y,\z$ be arbitrary elements of $\R^n$. Then by Definition \ref{dfn:18.2},
        \begin{align*}
            (\x+\y)+\z &= (x_1+y_1,\dots,x_n+y_n)+\z\\
            &= ((x_1+y_1)+z_1,\dots,(x_n+y_n)+z_n)\\
            &= (x_1+(y_1+z_1),\dots,x_n+(y_n+z_n))\\
            &= \x+(y_1+z_1,\dots,y_n+z_n)\\
            &= \x+(\y+\z)
        \end{align*}
        as desired.\par
        To prove that $\R^n$ obeys FA3 from Definition \ref{dfn:7.8}, it will suffice to find an element $0\in\R^n$ such that $\x+0=0+\x=\x$ for all $\x\in\R^n$. Choose $\bm{0}$ to be our $0$. Let $\x$ be an arbitrary element of $\R^n$. Then by Definition \ref{dfn:18.2},
        \begin{align*}
            \x+\bm{0} &= (x_1+0,\dots,x_n+0)\\
            &= (x_1,\dots,x_n)\\
            &= \x\\
            &= (0+x_1,\dots,0+x_n)\\
            &= \bm{0}+\x
        \end{align*}
        as desired.\par
        To prove that $\R^n$ obeys FA4 from Definition \ref{dfn:7.8}, it will suffice to show that for all $\x\in\R^n$, there exists $\y\in\R^n$ such that $\x+\y=\y+\x=0$. Let $\x$ be an arbitrary element of $\R^n$. Choose $\y=(-x_1,\dots,-x_n)$. Then by Definition \ref{dfn:18.2},
        \begin{align*}
            \x+\y &= (x_1+(-x_1),\dots,x_n+(-x_n))\\
            &= (0,\dots,0)\\
            &= \bm{0}\\
            &= ((-x_1)+x_1,\dots,(-x_n)+x_n)\\
            &= \y+\x
        \end{align*}
        as desired.\par
        To prove that $\R^n$ obeys VS1, it will suffice to show that for all $\lambda,\mu\in\R$ and $\x\in\R^n$, we have $(\lambda\mu)\x=\lambda(\mu\x)$. Let $\lambda,\mu$ be arbitrary elements of $\R$, and let $\x$ be an arbitrary element of $\R^n$. Then by Definition \ref{dfn:18.2},
        \begin{align*}
            (\lambda\mu)\x &= ((\lambda\mu)x_1,\dots,(\lambda\mu)x_n)\\
            &= (\lambda(\mu x_1),\dots,\lambda(\mu x_n))\\
            &= \lambda(\mu\x)
        \end{align*}
        as desired.\par
        To prove that $\R^n$ obeys VS2, it will suffice to show that for all $\lambda,\mu\in\R$ and $\x\in\R^n$, we have $(\lambda+\mu)\x=\lambda\x+\mu\x$. Let $\lambda,\mu$ be arbitrary elements of $\R$, and let $\x$ be an arbitrary element of $\R^n$. Then by Definition \ref{dfn:18.2},
        \begin{align*}
            (\lambda+\mu)\x &= ((\lambda+\mu)x_1,\dots,(\lambda+\mu)x_n)\\
            &= (\lambda x_1+\mu x_1,\dots,\lambda x_n+\mu x_n)\\
            &= (\lambda x_1,\dots,\lambda x_n)+(\mu x_1,\dots,\mu x_n)\\
            &= \lambda\x+\mu\x
        \end{align*}
        as desired.\par
        To prove that $\R^n$ obeys VS3, it will suffice to show that for all $\lambda\in\R$ and $\x,\y\in\R^n$, we have $\lambda(\x+\y)=\lambda\x+\lambda\y$. Let $\lambda$ be an arbitrary element of $\R$, and let $\x,\y$ be arbitrary elements of $\R^n$. Then by Definition \ref{dfn:18.2},
        \begin{align*}
            \lambda(\x+\y) &= \lambda(x_1+y_1,\dots,x_n+y_n)\\
            &= (\lambda(x_1+y_1),\dots,\lambda(x_n+y_n))\\
            &= (\lambda x_1+\lambda y_1,\dots,\lambda x_n+\lambda y_n)\\
            &= (\lambda x_1,\dots,\lambda x_n)+(\lambda y_1,\dots,\lambda y_2)\\
            &= \lambda\x+\lambda\y
        \end{align*}
        as desired.\par
        To prove that $\R^n$ obeys VS4, it will suffice to show that for all $\x\in\R^n$, we have $1\x=\x$. Let $\x$ be an arbitrary element of $\R^n$. Then by Definition \ref{dfn:18.2},
        \begin{align*}
            1\x &= (1x_1,\dots,1x_n)\\
            &= (x_1,\dots,x_n)\\
            &= \x
        \end{align*}
        as desired.
    \end{proof}
\end{exercise}

\begin{remark}\label{rmk:18.4}
    Since $\R^n$ with the two operations defined as above satisfies these eight axioms, we call $\R^n$ a \textbf{vector space}.
\end{remark}

\begin{exercise}\label{exr:18.5}
    Prove that if $\x\in\R^n$, then $0\x=\bm{0}$.
    \begin{proof}
        By Definition \ref{dfn:18.2}, we have that
        \begin{align*}
            0\x &= (0x_1,\dots,0x_n)\\
            &= (0,\dots,0)\\
            &= \bm{0}
        \end{align*}
        as desired.
    \end{proof}
\end{exercise}

\begin{definition}\label{dfn:18.6}
    Let $\x\in\R^n$. The \textbf{norm} of $\x$ is defined as
    \begin{equation*}
        \norm{\x} = \sqrt{x_1^2+\cdots+x_n^2}
    \end{equation*}
\end{definition}

\begin{definition}\label{dfn:18.7}
    We call $\norm{\y-\x}$ the \textbf{distance} between $\x$ and $\y$.
\end{definition}

\begin{remark}\label{rmk:18.8}
    If $n=1$, the norm coincides with the definition of the absolute value in $\R$.
\end{remark}

\begin{lemma}\label{lem:18.9}
    \leavevmode
    \begin{enumerate}[label=\textup{(}\alph*\textup{)},ref={\thelemma\alph*}]
        \item \label{lem:18.9a}If $x,y\in\R$, then $xy\leq\frac{x^2+y^2}{2}$.
        \begin{proof}
            Let $x,y$ be arbitrary elements of $\R$. Then by Lemma \ref{lem:7.26}, $0\leq(x-y)^2$. Therefore, we have that
            \begin{align*}
                xy &= \frac{2xy+0}{2}\\
                &\leq \frac{2xy+(x-y)^2}{2}\\
                &= \frac{2xy+x^2-2xy+y^2}{2}\\
                &= \frac{x^2+y^2}{2}
            \end{align*}
            as desired.
        \end{proof}
        \item \label{lem:18.9b}If $\x,\y\in\R^n$, then $|x_1y_1+\cdots+x_ny_n|\leq\norm{\x}\cdot\norm{\y}$.
        \begin{proof}
            Suppose first that $\norm{\x}=\norm{\y}=1$. Then by Definition \ref{dfn:18.6}, $\norm{\x}=1=\sqrt{x_1^2+\cdots+x_n^2}$, from which it follows that $1=x_1^2+\cdots+x_n^2$. Therefore, we have that
            \begin{align*}
                |x_1y_1+\cdots+x_ny_n| &\leq |x_1y_1|+\cdots+|x_ny_n|\tag*{Lemma \ref{lem:8.8}}\\
                &= |x_1||y_1|+\cdots+|x_n||y_n|\\
                &\leq \frac{|x_1|^2+|y_1|^2}{2}+\cdots+\frac{|x_n|^2+|y_n|^2}{2}\tag*{Lemma \ref{lem:18.9a}}\\
                &= \frac{x_1^2+y_1^2}{2}+\cdots+\frac{x_n^2+y_n^2}{2}\\
                &= \frac{(x_1^2+\cdots+x_n^2)+(y_1^2+\cdots+y_n^2)}{2}\\
                &= \frac{1+1}{2}\\
                &= 1
            \end{align*}
            as desired.\par
            Now let $\x,\y$ be arbitrary elements of $\R^n$. Consider the vectors $\mathbf{u}_\x,\mathbf{u}_\y$ defined by $\mathbf{u}_\x=\frac{\x}{\norm{\x}}$ and $\mathbf{u}_\y=\frac{\y}{\norm{\y}}$. By the proof of the first case, we have that
            \begin{align*}
                |x_1y_1+\cdots+x_ny_n| &= \norm{\x}\cdot\norm{\y}\cdot\left| \frac{x_1y_1}{\norm{\x}\cdot\norm{\y}}+\cdots+\frac{x_ny_n}{\norm{\x}\cdot\norm{\y}} \right|\\
                &= \norm{\x}\cdot\norm{\y}\cdot|u_{\x_1}u_{\y_1}+\cdots+u_{\x_n}u_{\y_n}|\\
                &\leq \norm{\x}\cdot\norm{\y}\cdot 1\\
                &= \norm{\x}\cdot\norm{\y}
            \end{align*}
            as desired.
        \end{proof}
    \end{enumerate}
\end{lemma}

\begin{theorem}\label{trm:18.10}
    If $\x,\y\in\R^n$ and $\lambda\in\R$, then
    \begin{enumerate}[label=\textup{(}\alph*\textup{)},ref={\thetheorem\alph*}]
        \item \label{trm:18.10a}$\norm{\x}\geq 0$. Moreover, $\norm{\x}=0$ if and only if $\x=\bm{0}$.
        \begin{proof}
            Let $\x$ be an arbitrary element of $\R^n$.\par\smallskip
            We first prove that $\norm{\x}\geq 0$. By Lemma \ref{lem:7.26}, $x_i^2\geq 0$ for all $i\in[n]$. Thus, by Definition \ref{dfn:7.21}, $x_1^2+\cdots+x_n^2\geq 0$. Therefore, we have by Definition \ref{dfn:18.6} that $\norm{\x}=\sqrt{x_1^2+\cdots+x_n^2}\geq 0$, as desired.\par\smallskip
            We now prove that $\norm{\x}=0$ if and only if $\x=\bm{0}$. Suppose first that $\norm{\x}=0$. Then by Definition \ref{dfn:18.6} and Script \ref{sct:7}, $x_1^2+\cdots+x_n^2=0$. Now suppose for the sake of contradiction that $\x\neq\bm{0}$. Then there exists an $x_i$ such that $x_i\neq 0$. Thus, by Lemma \ref{lem:7.26}, $x_i^2>0$. Additionally, $x_j^2\geq 0$ for all $j\in[n]$. Thus, we have that $0<x_i^2\leq x_1^2+\cdots+x_n^2$. But by Definition \ref{dfn:3.1}, this implies that $x_1^2+\cdots+x_n^2\neq 0$, a contradiction.\par
            Now suppose that $\x=\bm{0}$. Then by Definition \ref{dfn:18.6}, $\norm{\x}=\sqrt{0^2+\cdots+0^2}=0$, as desired.
        \end{proof}
        \item \label{trm:18.10b}$\norm{\lambda\x}=|\lambda|\cdot\norm{\x}$.
        \begin{proof}
            Let $\lambda$ be an arbitrary element of $\R$, and let $\x$ be an arbitrary element of $\R^n$. Then we have that
            \begin{align*}
                \norm{\lambda\x} &= \sqrt{(\lambda x_1)^2+\cdots+(\lambda x_n)^2}\tag*{Definition \ref{dfn:18.6}}\\
                &= |\lambda|\cdot\sqrt{x_1^2+\cdots+x_n^2}\\
                &= |\lambda|\cdot\norm{\x}\tag*{Definition \ref{dfn:18.6}}
            \end{align*}
            as desired.
        \end{proof}
        \item \label{trm:18.10c}$\norm{\x+\y}\leq\norm{\x}+\norm{\y}$.
        \begin{proof}
            Let $\x,\y$ be arbitrary elements of $\R^n$. Then we have that
            \begin{align*}
                \norm{\x+\y} &= \sqrt{(x_1+y_1)^2+\cdots+(x_n+y_n)^2}\tag*{Definition \ref{dfn:18.6}}\\
                &= \sqrt{(x_1^2+\cdots+x_n^2)+(2x_1y_1+\cdots+2x_ny_n)+(y_1^2+\cdots+y_n^2)}\\
                &\leq \sqrt{\norm{\x}^2+2\norm{\x}\norm{\y}+\norm{\y}^2}\tag*{Lemma \ref{lem:18.9}}\\
                &= \sqrt{(\norm{\x}+\norm{\y})^2}\\
                &= \norm{\x}+\norm{\y}
            \end{align*}
            as desired.
        \end{proof}
    \end{enumerate}
\end{theorem}

\begin{corollary}\label{cly:18.11}
    If $\x,\y,\z\in\R^n$ and $\lambda\in\R$, then
    \begin{enumerate}[label=\textup{(}\alph*\textup{)}]
        \item $\norm{\x-\z}\leq\norm{\x-\y}+\norm{\y-\z}$.
        \item $|\norm{\x}-\norm{\y}|\leq\norm{\x-\y}$.
    \end{enumerate}
    \begin{proof}
        The proofs are symmetric to those of Lemma \ref{lem:8.8}.
    \end{proof}
\end{corollary}

\marginnote{7/10:}The next goal is to "topologize" $\R^n$. To discuss topology on $\R^n$, we first need to introduce notions for $\R^n$ that are analogous to open and closed intervals for $\R$.

\begin{remark}\label{rmk:18.12}
    For $\x=(x_1,\dots,x_n)\in\R^n$ and $\y=(y_1,\dots,y_m)\in\R^m$, we identify $(\x,\y)\in\R^n\times\R^m$ with $(x_1,\dots,x_n,y_1,\dots,y_m)\in\R^{n+m}$. So if $A\subset\R^n$ and $B\subset\R^m$, we can consider $A\times B$ to be a subset of $\R^{n+m}$.\par
    If also $C\in\R^k$, then $(A\times B)\times C$ and $A\times(B\times C)$ correspond to the same subset of $\R^{n+m+k}$ under this identification; we write $A\times B\times C$ for this set.
\end{remark}

\begin{definition}\label{dfn:18.13}
    An \textbf{open rectangle} in $\R^n$ is a set of the form $(a_1,b_1)\times\cdots\times(a_n,b_n)$, a product of open intervals. Similarly, a \textbf{closed rectangle} in $\R^n$ is a set of the form $[a_1,b_1]\times\cdots\times[a_n,b_n]$. We allow the possibility that $a_j=b_j$ (where $[a_j,a_j]=\{a_j\}$). If there is at least one $j$ with $a_j=b_j$, then we say that the rectangle is \textbf{degenerate}; otherwise, we say that the rectangle is \textbf{non-degenerate}.
\end{definition}

\begin{definition}\label{dfn:18.14}
    A subset $U\subset\R^n$ is \textbf{open} if for all $\x\in U$, there exists an open rectangle $R$ such that $\x\in R\subset U$. A subset $C\in\R^n$ is \textbf{closed} if its compliment is open.
\end{definition}

\begin{exercise}\label{exr:18.15}
    Decide whether each of the following is an open set in $\R^2$.
    \begin{enumerate}[label={\textup{(}\alph*\textup{)}}]
        \item $\{(x_1,x_2)\mid x_1,x_2\in\R,\ x_1>0,\ x_2>0\}$.
        \begin{proof}
            To prove that $U=\{(x_1,x_2)\mid x_1,x_2\in\R,\ x_1>0,\ x_2>0\}$ is open, Definition \ref{dfn:18.14} tells us that it will suffice to show that for all $\x\in U$, there exists an open rectangle $R$ such that $\x\in R\subset U$. Let $\x$ be an arbitrary element of $U$. Then by the definition of $U$, $0<x_1$ and $0<x_2$. It follows by Theorem \ref{trm:5.2} and Corollary \ref{cly:6.12}, there exist $a_1,b_1,a_2,b_2$ such that $0<a_1<x_1<b_1$ and $0<a_2<x_2<b_2$. Thus, by Equations \ref{eqn:8.1}, $x_1\in(a_1,b_1)$ and $x_2\in(a_2,b_2)$. Consequently, if we let $R=(a_1,b_1)\times(a_2,b_2)$, Definition \ref{dfn:18.13} guarantees that $R$ is an open rectangle. Additionally, Definition \ref{dfn:1.15} asserts that $(x_1,x_2)=\x\in R$, as desired. Additionally, if $\y$ is any vector in $R$, then by the definition of $R$, $0<a_1<y_1$ and $0<a_2<y_2$. Thus, by transitivity, $\y\in U$. Therefore, by Definition \ref{dfn:1.3}, $R\subset U$, as desired.
        \end{proof}
        \item $\{(x,0)\mid x\in\R\}$.
        \begin{proof}
            To prove that $U=\{(x,0)\mid x\in\R\}$ is not open, Definition \ref{dfn:18.14} tells us that it will suffice to find an $\x\in U$ such that for all open rectangles $R$ containing $\x$, $R\not\subset U$. Let $\x=(0,0)$, and let $R$ be an arbitrary open rectangle containing $\x$. By Definitions \ref{dfn:18.13} and \ref{dfn:1.15} along with Equations \ref{eqn:8.1}, $a_1<0<b_1$ and $a_2<0<b_2$. Thus, by consecutive applications of Theorem \ref{trm:5.2}, there exist points $y_1,y_2\in\R$ such that $a_1<y_1<0$ and $a_2<y_2<0$. It follows that $\y=(y_1,y_2)\in R$. However, since $y_2\neq 0$ by Definition \ref{dfn:3.1}, $\y\notin U$. Therefore, by Definition \ref{dfn:1.3}, $R\not\subset U$, as desired.
        \end{proof}
    \end{enumerate}
\end{exercise}

\begin{exercise}\label{exr:18.16}
    Show that if $R_1,\dots,R_m$ are open rectangles containing $\x\in\R^n$, then $R=R_1\cap\cdots\cap R_m$ is an open rectangle containing $\x\in\R^n$. If $R=(a_1,b_1)\times\cdots\times(a_n,b_n)$, derive formulas for $a_i$ and $b_i$ in terms of the corresponding quantities for $R_1,\dots,R_m$.
    \begin{proof}
        Let $R_i=(r_{ij},s_{ij})_{j=1}^n$ for all $i\in[m]$. To prove that $R=\bigcap_{i=1}^mR_i$ is an open rectangle containing $\x$, Definitions \ref{dfn:18.13} and \ref{dfn:1.15} tell us that it will suffice to show that $R$ is the Cartesian product of open intervals, each containing its respective $x_j$. Since $\x\in R_i$ for all $i\in[m]$, we have by Definition \ref{dfn:1.15} that $x_j\in(r_{ij},s_{ij})$ for all $i\in[m],\ j\in[n]$. Thus, by Corollary \ref{cly:3.19}, $\bigcap_{i=1}^m(r_{ij},s_{ij})$ is a region (hence an open interval by Corollary \ref{cly:4.11} and Lemma \ref{lem:8.3}) containing $x_j$ for all $j\in[n]$. Therefore, since $R=\bigcap_{i=1}^mR_i=\prod_{i=1}^n(\bigcap_{i=1}^m(r_{ij},s_{ij}))$ by Script \ref{sct:1}, we have that $R$ is the Cartesian product of open intervals, each containing its respective $x_j$, as desired.\par
        Let $a_j=\max_{i=1}^m(r_{ij})$ and let $b_j=\min_{i=1}^m(s_{ij})$ for all $j\in[n]$. To prove that $R=(a_j,b_j)_{j=1}^n$, Definition \ref{dfn:1.2} tells us that it will suffice to show that every $\x\in R$ is an element of $(a_j,b_j)_{j=1}^n$ and vice versa. Suppose first that $\x$ is an arbitrary element of $R$. Then by Definition \ref{dfn:1.6}, $\x\in R_i$ for all $i\in[m]$. It follows by Definition \ref{dfn:1.15} that $x_j\in(r_{ij},s_{ij})$ for all $i\in[m],\ j\in[n]$, including the $j,j'$ for which $r_{ij}$ is at its maximum and $s_{ij'}$ is at its minimum. In other words, $x_j\in(a_j,b_j)$ for all $j\in[n]$. Therefore, by Definition \ref{dfn:1.15}, $\x\in(a_j,b_j)_{j=1}^n$, as desired. The proof is symmetric in the other direction.
    \end{proof}
\end{exercise}

\begin{definition}\label{dfn:18.17}
    The \textbf{open ball} (in $\R^n$ with center $\mathbf{p}$ and radius $r>0$) is defined as
    \begin{equation*}
        B(\mathbf{p},r) = \{\x\in\R^n\mid\norm{\x-\mathbf{p}}<r\}
    \end{equation*}
    The \textbf{closed ball} (in $\R^n$ with center $\mathbf{p}$ and radius $r>0$) is defined as
    \begin{equation*}
        \overline{B}(\mathbf{p},r) = \{\x\in\R^n\mid\norm{\x-\mathbf{p}}\leq r\}
    \end{equation*}
\end{definition}

\begin{remark}\label{rmk:18.18}
    In $\R^1$, an open rectangle is also an open ball, and vice versa.
\end{remark}

The following results illustrate how open rectangles and open balls in $\R^n$ are "compatible" with each other.

\begin{lemma}\label{lem:18.19}
    Fix $\x\in\R$.
    \begin{enumerate}[label={\textup{(}\alph*\textup{)}},ref={\thelemma\alph*}]
        \item \label{lem:18.19a}If $R$ is an open rectangle containing $\x$, then there exists $r>0$ such that $B(\x,r)\subset R$.
        \begin{proof}
            Since $\x\in R$, Definitions \ref{dfn:18.13} and \ref{dfn:1.15} tell us that that $x_i\in(a_i,b_i)$ for all $i\in[n]$. Additionally, we know by Corollary \ref{cly:4.11} and Lemma \ref{lem:8.3} that each $(a_i,b_i)$ is an open interval. Combining the last two results, we have by Lemma \ref{lem:8.10} that for each $i\in[n]$, there exists $\delta_i>0$ such that $(x_i-\delta_i,x_i+\delta_i)\subset(a_i,b_i)$. Let $r=\min\{\delta_i\}_{i=1}^n$.\par
            To prove that $B(\x,r)\subset R$, Definition \ref{dfn:1.3} tells us that it will suffice to show that every $\y\in B(\x,r)$ is an element of $R$. Let $\y$ be an arbitrary element of $B(\x,r)$. Then by Definition \ref{dfn:18.17}, $\norm{\y-\x}<r$. It follows that
            \begin{align*}
                |y_i-x_i| &= \sqrt{(y_i-x_i)^2}\\
                &\leq \sqrt{(y_1-x_1)^2+\cdots+(y_n-x_n)^2}\tag*{Lemma \ref{lem:7.26}}\\
                &= \norm{\y-\x}\tag*{Definition \ref{dfn:18.6}}\\
                &< r
            \end{align*}
            for all $i\in[n]$. Thus, by the definition of $r$, $|y_i-x_i|\leq\delta_i$ for all $i\in[n]$. Consequently, by Exercise \ref{exr:8.9} and Definition \ref{dfn:1.3}, $y_i\in(a_i,b_i)$ for all $i\in[n]$. Therefore, by Definitions \ref{dfn:1.15} and \ref{dfn:18.13}, $\y\in R$, as desired.
        \end{proof}
        \item \label{lem:18.19b}If $B$ is an open ball containing $\x$, then there exists an open rectangle $R$ such that $\x\in R\subset B$.
        \begin{lemma*}
            If $\x\in\R^n$, then $\norm{\x}\leq\sum_{i=1}^n|x_i|$.
            \begin{proof}
                By Definition \ref{dfn:18.2}, we can decompose $\x$ into the sum of $n$ unit vectors $\mathbf{u_i}$ (where $\mathbf{u_i}$ points one unit in the $i^\text{th}$ direction), each scaled by $x_i$; symbolically, let $\x=\sum_{i=1}^nx_i\mathbf{u_i}$. Therefore,
                \begin{align*}
                    \norm{\x} &= \norm{\sum_{i=1}^nx_i\mathbf{u_i}}\\
                    &= \sum_{i=1}^n\norm{x_i\mathbf{u_i}}\tag*{Theorem \ref{trm:18.10c}}\\
                    &= \sum_{i=1}^n|x_i|\cdot\norm{\mathbf{u_i}}\tag*{Theorem \ref{trm:18.10b}}\\
                    &= \sum_{i=1}^n|x_i|\cdot\sqrt{1^2}\tag*{Definition \ref{dfn:18.6}}\\
                    &= \sum_{i=1}^n|x_i|
                \end{align*}
                as desired.
            \end{proof}
        \end{lemma*}
        \begin{proof}[Proof of Lemma \ref{lem:18.19b}]
            Suppose $\x\in B(\y,r)$. Then by Definition \ref{dfn:18.17}, $\norm{\x-\y}<r$. Thus, we can define $r'=r-\norm{\x-\y}$ such that $r'>0$. With this term defined, we can let $R=(x_i-\frac{r'}{n},x_i+\frac{r'}{n})_{i=1}^n$.\par
            To prove that $\x\in R$, Definition \ref{dfn:18.13} tells us that it will suffice to show that $x_i\in(x_i-\frac{r'}{n},x_i+\frac{r'}{n})$ for all $i\in[n]$. But since $|x_i-x_i|=0<\frac{r'}{n}$ for all $i\in[n]$, Exercise \ref{exr:8.9} asserts that this is true.\par
            To prove that $R\subset B$, Definition \ref{dfn:1.3} tells us that it will suffice to show that every $\z\in R$ is an element of $B$. Let $\z$ be an arbitrary element of $R$. Then by Definition \ref{dfn:18.13}, $z_i\in(x_i-\frac{r'}{n},x_i+\frac{r'}{n})$ for all $i\in[n]$. It follows by Exercise \ref{exr:8.9} that $|z_i-x_i|<\frac{r'}{n}$ for all $i\in[n]$. Consequently,
            \begin{align*}
                \norm{\z-\y} &\leq \norm{\z-\x}+\norm{\x-\y}\tag*{Corollary \ref{cly:18.11}}\\
                &\leq \sum_{i=1}^n|z_i-x_i|+\norm{\x-\y}\tag*{Lemma}\\
                &< \sum_{i=1}^n\frac{r'}{n}+\norm{\x-\y}\\
                &= r'+\norm{\x-\y}\\
                &= r-\norm{\x-\y}+\norm{\x-\y}\\
                &= r
            \end{align*}
            Therefore, by Definition \ref{dfn:18.17}, $\z\in B$, as desired.
        \end{proof}
    \end{enumerate}
\end{lemma}

\begin{corollary}\label{cly:18.20}
    A set $U\subset\R^n$ is open if and only if for every $\x\in U$, there exists $r>0$ such that $B(\x,r)\subset U$.
    \begin{proof}
        Suppose first that $U\subset\R^n$ is open. Let $\x$ be an arbitrary element of $U$. By Definition \ref{dfn:18.14}, there exists an open rectangle $R$ such that $\x\in R\subset U$. Therefore, by Lemma \ref{lem:18.19}, there exists $r>0$ such that $B(\x,r)\subset R\subset U$, as desired.\par
        Now suppose that for all $\x\in U$, there exists $r>0$ such that $B(\x,r)\subset U$. To prove that $U$ is open, Definition \ref{dfn:18.14} tells us that it will suffice to show that for all $\x\in U$, there exists an open rectangle $R$ such that $\x\in R\subset U$. Let $\x$ be an arbitrary element of $U$. Then there exists $r>0$ such that $B(\x,r)\subset U$. Therefore, by Lemma \ref{lem:18.19}, there exists an open rectangle $R$ such that $\x\in R\subset B\subset U$, as desired.
    \end{proof}
\end{corollary}

\begin{corollary}\label{cly:18.21}\marginnote{\emph{7/14:}}
    Open balls are open and closed balls are closed.
    \begin{proof}
        We will take this one claim at a time.\par\smallskip
        Let $B(\x,r)$ be an arbitrary open ball. To prove that $B$ is open, Definition \ref{dfn:18.14} tells us that it will suffice to show that for all $\y\in B$, there exists an open rectangle $R$ such that $\y\in R\subset B$. But by Lemma \ref{lem:18.19}, this is true.\par
        Let $\overline{B}(\x,r)$ be an arbitrary closed ball. To prove that $\overline{B}$ is closed, Definition \ref{dfn:18.14} tells us that it will suffice to show that $\R^n\setminus\overline{B}$ is open. To do this, Definition \ref{dfn:18.14} tells us again that it will suffice to verify that for all $\y\in\R^n\setminus\overline{B}$, there exists an open rectangle $R$ such that $\y\in R\subset\R^n\setminus\overline{B}$. Let $\y$ be an arbitrary element of $\R^n\setminus\overline{B}$. Then by Definition \ref{dfn:18.17}, $\norm{\y-\x}>r$. Thus, $\norm{\y-\x}-r>0$, so we may define $r'=\norm{\y-\x}-r$. Now consider $B(\y,r')$. By Lemma \ref{lem:18.19}, there exists an open rectangle $R$ such that $\y\in R\subset B$. Consequently, by Script \ref{sct:1}, the only thing left to do to verify that $R\subset\R^n\setminus\overline{B}$ is to show that $B\cap\overline{B}=\emptyset$. As such, suppose for the sake of contradiction that $B\cap\overline{B}\neq\emptyset$. Then there exists $\z\in\R^n$ such that $\z\in B$ and $z\in\overline{B}$. It follows by consecutive applications of Definition \ref{dfn:18.17} that $\norm{\z-\y}<r'$ and $\norm{\z-\x}\leq r$. But then we have that
        \begin{align*}
            \norm{\x-\y} &\leq \norm{\x-\z}+\norm{\z-\y}\tag*{Corollary \ref{cly:18.11}}\\
            &< r'+r\\
            &= \norm{\y-\x}-r+r\\
            &= \norm{\x-\y}
        \end{align*}
        a contradiction, as desired.
    \end{proof}
\end{corollary}

\begin{proposition}\label{prp:18.22}
    Let $U\subset\R^n$. The following are equivalent:
    \begin{enumerate}[label={\textup{(}\alph*\textup{)}}]
        \item $U$ is open.
        \item $U$ is a (possibly empty) union of open balls.
        \item $U$ is a (possibly empty) union of open rectangles.
    \end{enumerate}
    \begin{proof}
        As in Theorem \ref{trm:11.5}, to prove that statements a-c are equivalent, it will suffice to verify that $a\Rightarrow b$, $b\Rightarrow c$, and $c\Rightarrow a$. Let's begin.\par\smallskip
        First, suppose that $U$ is open. Then by Corollary \ref{cly:18.20}, for every $\x\in U$, there exists $r>0$ such that $B_\x(\x,r)\subset U$. Therefore, $U=\bigcup_{\x\in U}B_\x$, as desired.\par
        Second, suppose that $U$ is a union of open balls. Then for every open ball $B(\x,r)$ comprising $U$, Lemma \ref{lem:18.19} asserts that for every $\y\in B$, there exists an open rectangle $R_\y$ such that $\y\in R_\y\subset B$. Therefore, $U=\bigcup_{\y\in U}R_\y$, as desired.\par
        Third, suppose that $U$ is a union of open rectangles. Then for every $\x\in U$, there exists an open rectangle $R$ such that $\x\in R\subset U$. Therefore, by Definition \ref{dfn:18.14}, $U$ is open, as desired.
    \end{proof}
\end{proposition}

\begin{remark}\label{rmk:18.23}
    If $X\subset\R^n$, then $X$ is also a topolotical space with the \textbf{subspace topology}. That is, $A\subset X$ is \textbf{open} (in $X$) if there exists an open set $U\subset\R^n$ such that $X\cap U=A$. (See Script \ref{sct:8}.)
\end{remark}

We now discuss functions between Euclidean spaces.

\begin{definition}\label{dfn:18.24}
    Let $A\subset\R^n$ and let $f:A\to\R$. Define the \textbf{graph} of $f$ by
    \begin{equation*}
        \graph(f) = \{(x_1,\dots,x_n,f(x_1,\dots,x_n))\in\R^{n+1}\mid(x_1,\dots,x_n)\in A\}
    \end{equation*}
\end{definition}

\begin{exercise}\label{exr:18.25}
    For each of the following functions, describe the graph as a subset of $\R^3$.
    \begin{enumerate}[label={(\alph*)}]
        \item $f:\R^2\to\R$ given by $f(x,y)=2$ for all $(x,y)\in\R^2$.
        \begin{proof}[Description]
            For this function, we have $\graph(f)=\{(x,y,2)\in\R^3\mid(x,y)\in\R^2\}$. This makes the graph equal to the set of all points in $\R^3$ with $z=2$, which will be a planar, constant, infinite subspace of $\R^3$.
        \end{proof}
        \item $f:\R^2\to\R$ given by $f(x,y)=x+y+1$ for all $(x,y)\in\R^2$.
        \begin{proof}[Description]
            For this function, we have $\graph(f)=\{(x,y,x+y+1)\in\R^3\mid(x,y)\in\R^2\}$. Thus, the graph will be a planar, sloped, infinite subspace of $\R^3$ with gradient pointing in the $\mathbf{\hat{\imath}}+\mathbf{\hat{\jmath}}$ direction.
        \end{proof}
        \item $f:\R^2\to\R$ given by $f(x,y)=x^2+y^2$ for all $(x,y)\in\R^2$.
        \begin{proof}[Description]
            For this function, we have $\graph(f)=\{(x,y,x^2+y^2)\in\R^3\mid(x,y)\in\R^2\}$. Thus, the graph will be the paraboloid centered at the origin.
        \end{proof}
    \end{enumerate}
\end{exercise}

In Script \ref{sct:9}, we gave a definition of continuity that we can generalize to this case:

\begin{definition}\label{dfn:18.26}
    Let $X,Y$ be topological spaces. A function $f:X\to Y$ is \textbf{continuous} if for every open set $U\subset Y$, the preimage $f^{-1}(U)$ is open in $X$.\par
    The function $f:X\to Y$ is \textbf{continuous} at $x\in X$ if for every open set $U\subset Y$ containing $f(x)$, the preimage $f^{-1}(U)$ is open in $X$.
\end{definition}

\begin{theorem}\label{trm:18.27}\leavevmode
    \begin{enumerate}[label={\textup{(}\alph*\textup{)}}]
        \item A function $f:X\to Y$ is continuous if and only if it is continuous at every $x\in X$.
        \item A function $f:X\to Y$ is continuous if and only if $f^{-1}(B)$ is closed in $X$ whenever $B$ is closed in $Y$.
    \end{enumerate}
    \begin{proof}
        The proofs are symmetric to those of Theorem \ref{trm:9.10} and Proposition \ref{prp:9.5}, respectively.
    \end{proof}
\end{theorem}

\begin{remark}\label{rmk:18.28}\marginnote{7/17:}
    There is also a characterization of continuity in terms of limits, as in one variable, as we shall now see. First we need the definitions of limit point and limit.
\end{remark}

\begin{definition}\label{dfn:18.29}
    Let $A\subset\R^n$.
    \begin{enumerate}[label={(\alph*)}]
        \item We say that $\x$ is a \textbf{limit point} of $A$ if for every open set $U$ containing $\x$, $A\cap(U\setminus\{\x\})\neq\emptyset$.
        \item Let $\x\in LP(A)$ and $f:A\to\R^m$. We say $\mathbf{L}\in\R^m$ is the \textbf{limit} (of $f$ at $\x$) if for every $\epsilon>0$, there exists $\delta>0$ such that if $\y\in A$ and $0<\norm{\y-\x}<\delta$, then $\norm{f(\x)-\mathbf{L}}<\epsilon$. As in one variable, we can show that limits are unique. If $\mathbf{L}$ is the limit of $f$ at $\x$, we write $\lim_{\y\to\x}f(\x)=\mathbf{L}$.
    \end{enumerate}
\end{definition}

\begin{exercise}\label{exr:18.30}
    Compute the following limits if they exist, or prove that the limit does not exist.
    \begin{lemma*}
        Let $\x=(x_1,\dots,x_n)$ be an arbitrary element of $\R^n$. Then $\norm{\x}<\delta$ implies that $|x_i|<\delta$ for all $1\leq i\leq n$.
        \begin{proof}
            Suppose for the sake of contradiction that for some $1\leq i\leq n$, $|x_i|\geq\delta$. Note that since $0\leq\norm{\x}<\delta$ by Theorem \ref{trm:18.10}, $|\delta|=\delta$ by Definition \ref{dfn:8.4}. Then
            \begin{align*}
                \norm{\x} &= \sqrt{x_1^2+\cdots+x_{i-1}^2+x_i^2+x_{i+1}^2+\cdots+x_n^2}\tag*{Definition \ref{dfn:18.6}}\\
                &\geq \sqrt{x_1^2+\cdots+x_{i-1}^2+\delta^2+x_{i+1}^2+\cdots+x_n^2}\\
                &\geq \sqrt{\delta^2}\\
                &= \delta\\
                &> \norm{\x}
            \end{align*}
            a contradiction.
        \end{proof}
    \end{lemma*}
    \begin{enumerate}[label={(\alph*)},ref={\theexercise\alph*}]
        \item \label{exr:18.30a}$\lim_{(x,y)\to(a,b)}4xy$.
        \begin{proof}
            To prove that $\lim_{(x,y)\to(a,b)}4xy=4ab$, Definition \ref{dfn:18.29} tells us that it will suffice to show that for every $\epsilon>0$, there exists $\delta>0$ such that if $(x,y)\in\R^2$ and $0<\norm{(x,y)-(a,b)}<\delta$, then $\norm{4xy-4ab}<\epsilon$. Let $\epsilon>0$ be arbitrary. Choose $\delta=\min(\min(\frac{\epsilon}{8(|b|+1)},1),\frac{\epsilon}{8(|a|-1)})$. Then since $\norm{(x-a,y-b)}<\delta$ by hypothesis and Definition \ref{dfn:18.2}, the lemma asserts that $|x-a|<\delta$ and $|y-b|<\delta$. It follows that $|x-a|<\min(\frac{\epsilon}{8(|b|+1)},1)$ and $|y-b|<\frac{\epsilon}{8(|a|-1)}$. Consequently, by an argument symmetric to the proof of Theorem \ref{trm:11.9}, $|xy-ab|<\frac{\epsilon}{4}$. Therefore, $\norm{4xy-4ab}=|4xy-4ab|<4\cdot\frac{\epsilon}{4}=\epsilon$, as desired.
        \end{proof}
        % Prove that $(0,0)$ is a limit point of $\R^2\setminus\{(0,0)\}$.
        \item \label{exr:18.30b}$\lim_{(x,y)\to(0,0)}\frac{x^3-y^3}{x^2+y^2}$.
        \begin{proof}
            To ensure that $\lim_{(x,y)\to(0,0)}\frac{x^3-y^3}{x^2+y^2}$ is well-defined, Definition \ref{dfn:18.29} tells us that we must show that $(0,0)\in LP(\R^2\setminus\{(0,0)\})$, assuming that $\R^2\setminus\{(0,0)\}$ is the domain of $\frac{x^3-y^3}{x^2+y^2}$ since the domain is not explicitly specified. To do so, Definition \ref{dfn:18.29} tells us again that it will suffice to verify that for every open set $U$ containing $(0,0)$, $(\R^2\setminus\{(0,0)\})\cap(U\setminus\{(0,0)\})\neq\emptyset$. Let $U$ be an arbitrary open set containing $(0,0)$. By Definition \ref{dfn:18.14}, there exists an open rectangle $R$ such that $(0,0)\in R\subset U$. By Definition \ref{dfn:18.13}, $R$ is not a singleton set. Thus, there exist at least one point in $R$, i.e., in $U$ that is not equal to $(0,0)$ and is (naturally) in $\R^2$, as desired.\par
            To prove that $\lim_{(x,y)\to(0,0)}\frac{x^3-y^3}{x^2+y^2}=0$, Definition \ref{dfn:18.29} tells us that it will suffice to show that for every $\epsilon>0$, there exists $\delta>0$ such that if $(x,y)\in\R^2$ and $0<\norm{(x,y)-(0,0)}<\delta$, then $||\frac{x^3-y^3}{x^2+y^2}-0||=|\frac{x^3-y^3}{x^2+y^2}|<\epsilon$. Let $\epsilon>0$ be arbitrary. Choose $\delta=\frac{\epsilon}{3}$. Then from previous results, we can prove two important bounds on combinations of $x$ and $y$ that will be useful in the final inequality. Let's begin.\par
            First, since $0<\norm{(x,y)}$, Theorem \ref{trm:18.10} implies that $x\neq 0$ or $y\neq 0$. Thus, $x^2+y^2\neq 0$. Consequently, we may argue in a well-defined manner that
            \begin{align*}
                \left| \frac{xy}{x^2+y^2} \right| &= |xy|\cdot\left| \frac{1}{x^2+y^2} \right|\\
                &\leq \frac{x^2+y^2}{2}\cdot\left| \frac{1}{x^2+y^2} \right|\tag*{Lemma \ref{lem:18.9}}\\
                &= \frac{1}{2}
            \end{align*}
            Second, since we know from the lemma that $|x|<\frac{\epsilon}{3}$ and $|y|<\frac{\epsilon}{3}$, we have that
            \begin{align*}
                |x-y| &\leq |x|+|-y|\tag*{Lemma \ref{lem:8.8}}\\
                &< \frac{\epsilon}{3}+\frac{\epsilon}{3}\\
                &= \frac{2\epsilon}{3}
            \end{align*}\par
            Therefore, combining the last two results, we have that
            \begin{align*}
                \left| \frac{x^3-y^3}{x^2+y^2} \right| &= \left| \frac{(x-y)(x^2+xy+y^2)}{x^2+y^2} \right|\\
                &= |x-y|\cdot\left| \frac{xy}{x^2+y^2}+1 \right|\\
                &\leq |x-y|\cdot\left| \frac{xy}{x^2+y^2} \right|+|x-y|\tag*{Lemma \ref{lem:8.8}}\\
                &< \frac{2\epsilon}{3}\cdot\frac{1}{2}+\frac{2\epsilon}{3}\\
                &= \epsilon
            \end{align*}
            as desired.
        \end{proof}
        \item \label{exr:18.30c}$\lim_{(x,y)\to(0,0)}\frac{x^2-y^2}{x^2+y^2}$.
        \begin{proof}
            To prove that $\lim_{(x,y)\to(0,0)}\frac{x^2-y^2}{x^2+y^2}$ does not exist, Definition \ref{dfn:18.29} tells us that it will suffice to show that for every $L\in\R$, there exists an $\epsilon>0$ such that for all $\delta>0$, there exists $(x,y)\in\R^2$ satisfying $0<\norm{(x,y)-(0,0)}<\delta$ such that $||\frac{x^2-y^2}{x^2+y^2}-L||\geq\epsilon$. Let $L$ be an arbitrary element of $\R$. We divide into two cases ($L\geq 0$ and $L<0$). Suppose first that $L\geq 0$. Choose $\epsilon=1$. Let $\delta>0$ be arbitrary. Choose $(0,\frac{\delta}{2})\in\R^2$. By Definition \ref{dfn:18.6}, $0<\norm{(0,\frac{\delta}{2})}=\sqrt{\delta^2/4}=\frac{\delta}{2}<\delta$. Additionally,
            \begin{align*}
                \norm{\frac{0^2-\left( \frac{\delta}{2} \right)^2}{0^2+\left( \frac{\delta}{2} \right)^2}-L} &= \left| \frac{-1}{1}-L \right|\\
                &= |-1-L|\\
                &\geq 1-|L|\\
                &\geq 1\\
                &= \epsilon
            \end{align*}
            as desired. The proof is symmetric in the other case.
        \end{proof}
    \end{enumerate}
\end{exercise}

\begin{theorem}\label{trm:18.31}
    Let $A\subset\R^n$ and $\x\in A$. Let $f:A\to\R^m$. Then the following are equivalent:
    \begin{enumerate}[label={\textup{(}\alph*\textup{)}}]
        \item $f$ is continuous at $\x$.
        \item For every $\epsilon>0$, there exists $\delta>0$ such that if $\y\in A$ and $\norm{\y-\x}<\delta$, then $\norm{f(\y)-f(\x)}<\epsilon$.
        \item Either $\x\notin LP(A)$ or $\lim_{\y\to\x}f(\y)=f(\x)$.
    \end{enumerate}
    \begin{proof}
        The proof is symmetric to that of Theorem \ref{trm:11.9}.
    \end{proof}
\end{theorem}

\begin{exercise}\label{exr:18.32}
    For each of the following, prove that $f$ is continuous at every point in its domain.
    \begin{enumerate}[label={(\alph*)},ref={\theexercise\alph*}]
        \item \label{exr:18.32a}$A\subset\R^n$ and $f:A\to\R^m$ is a constant function.
        \begin{proof}
            Since $f$ is a constant function, we may let $f(\x)=\mathbf{c}$ for all $\x\in A$. To prove that $f$ is continuous at every $\x\in A$, let $\x$ be an arbitrary element of $A$; then Theorem \ref{trm:18.31} tells us that it will suffice to show that for every $\epsilon>0$, there exists $\delta>0$ such that if $\y\in A$ and $\norm{\y-\x}<\delta$, then $\norm{f(\y)-f(\x)}<\epsilon$. Let $\epsilon>0$ be arbitrary. Choose $\delta=1$. Let $\y$ be an arbitrary element of $A$ satisfying $\norm{\y-\x}<\delta$. Then
            \begin{align*}
                \norm{f(\y)-f(\x)} &= \norm{\mathbf{c}-\mathbf{c}}\\
                &= \norm{\mathbf{0}}\\
                &= 0\tag*{Theorem \ref{trm:18.10}}\\
                &< \epsilon
            \end{align*}
            as desired.
        \end{proof}
        \item \label{exr:18.32b}Fix $\mathbf{a}\in\R^m$. Define $f:\R\to\R^m$ by $f(h)=h\mathbf{a}$.
        \begin{proof}
            We divide into two cases ($\mathbf{a}=0$ and $\mathbf{a}\neq 0$). If $\mathbf{a}=0$, then by Exercise \ref{exr:18.32a}, $f$ is continuous at every point in its domain. If $\mathbf{a}\neq 0$, we continue.\par
            Let $x$ be an arbitrary element of $\R$. To prove that $f$ is continuous at $x$, Theorem \ref{trm:18.31} tells us that it will suffice to show that for every $\epsilon>0$, there exists $\delta>0$ such that if $y\in\R$ and $\norm{y-x}=|y-x|<\delta$, then $\norm{f(y)-f(x)}<\epsilon$. Let $\epsilon>0$ be arbitrary. Choose $\delta=\frac{\epsilon}{\norm{\mathbf{a}}}$ (by Theorem \ref{trm:18.10} and the supposition that $\mathbf{a}\neq 0$, we know that $\norm{\mathbf{a}}\neq 0$). Let $y$ be an arbitrary element of $\R$ satisfying $|y-x|<\delta$. Then
            \begin{align*}
                \norm{f(y)-f(x)} &= \norm{y\mathbf{a}-x\mathbf{a}}\\
                &= \norm{(y-x)\mathbf{a}}\\
                &= |y-x|\cdot\norm{\mathbf{a}}\tag*{Theorem \ref{trm:18.10}}\\
                &< \frac{\epsilon}{\norm{\mathbf{a}}}\cdot\norm{\mathbf{a}}\\
                &= \epsilon
            \end{align*}
            as desired.
        \end{proof}
        \item \label{exr:18.32c}Fix $\x\in\R^n$. Define $f:\R^n\to\R$ by $f(\y)=\norm{\y-\x}$.
        \begin{proof}
            Let $\y$ be an arbitrary element of $\R^n$. To prove that $f$ is continuous at $y$, Theorem \ref{trm:18.31} tells us that it will suffice to show that for all $\epsilon>0$, there exists $\delta>0$ such that if $\z\in\R^n$ and $\norm{\z-\y}<\delta$, then $\norm{f(\z)-f(\y)}<\epsilon$. Let $\epsilon>0$ be arbitrary. Choose $\delta=\epsilon$. Let $\z$ be an arbitrary element of $\R^n$ satisfying $\norm{\z-\y}<\delta$. Then
            \begin{align*}
                \norm{f(\z)-f(\y)} &= \norm{\norm{\z-\x}-\norm{\y-\x}}\\
                &\leq \norm{(\z-\x)-(\y-\x)}\tag*{Corollary \ref{cly:18.11}}\\
                &= \norm{\z-\y}\\
                &< \epsilon
            \end{align*}
            as desired.
        \end{proof}
        \item \label{exr:18.32d}$f:\R^2\to\R$ given by $f(x,y)=4xy$.
        \begin{proof}
            Let $(a,b)$ be an arbitrary element of $\R^2$. To prove that $f$ is continuous at $(a,b)$, Theorem \ref{trm:18.31} tells us that it will suffice to show that either $(a,b)\notin LP(\R^2)$ or $\lim_{(x,y)\to(a,b)}4xy=4ab$. But by Exercise \ref{exr:18.30a}, $\lim_{(x,y)\to(a,b)}4xy=4ab$, as desired.
        \end{proof}
    \end{enumerate}
\end{exercise}

\begin{exercise}\label{exr:18.33}
    Consider the function $f:\R^2\setminus\{(0,0)\}\to\R$ given by $f(x,y)=\frac{x^3-y^3}{x^2+y^2}$ (see Exercise \ref{exr:18.30b}). It can be shown that this function is continuous on its domain. Can you extend this function continuously to $\R^2$? More specifically, can you define a continuous function $g:\R^2\to\R$ such that $g(x,y)=f(x,y)$ for all $(x,y)\neq(0,0)$?
    \begin{proof}
        Let $g:\R^2\to\R$ be defined by
        \begin{equation*}
            g(x,y) =
            \begin{cases}
                f(x,y) & (x,y)\neq(0,0)\\
                0 & (x,y)=(0,0)
            \end{cases}
        \end{equation*}
        By the continuity of $f$ on $\R^2\setminus\{(0,0)\}$, $g$ is continuous on $\R^2\setminus\{(0,0)\}$. Additionally, by Exercise \ref{exr:18.30c}, $\lim_{(x,y)\to(0,0)}g(x,y)=0=g(0,0)$. Thus, by Theorem \ref{trm:18.31}, $g$ is continuous at $(0,0)$. Therefore, $g$ is continuous on $(\R^2\setminus\{(0,0)\})\cup\{(0,0)\}=\R^2$, as desired.
    \end{proof}
\end{exercise}

\begin{definition}\label{dfn:18.34}\marginnote{7/21:}
    Let $m\in\N$. Suppose $I=\{i_1,\dots,i_k\}\subset[m]$ with $i_1<\cdots<i_k$. We define the \textbf{projection function} $\pi_I:\R^m\to\R^k$ as
    \begin{equation*}
        \pi_I(\x) = (x_{i_1},\dots,x_{i_k})
    \end{equation*}
    If $I=\{i\}$ has only one element, we write $\pi_i$ instead of $\pi_{\{i\}}$.
\end{definition}

\begin{exercise}\label{exr:18.35}
    Prove that each $\pi_I$ is continuous.
    \begin{proof}
        Let $I$ be an arbitrary subset of $[m]$, and let $\x$ be an arbitrary element of $\R^m$. To prove that $\pi_I$ is continuous at $\x$, Theorem \ref{trm:18.31} tells us that it will suffice to show that for every $\epsilon>0$, there exists $\delta>0$ such that if $\y\in\R^m$ and $\norm{\y-\x}<\delta$, then $\norm{\pi_I(\y)-\pi_I(\x)}<\epsilon$. Let $\epsilon>0$ be arbitrary. Choose $\delta=\epsilon$. Let $\y$ be an arbitrary element of $\R^m$ such that $\norm{\y-\x}<\delta$. Then
        \begin{align*}
            \norm{\pi_I(\y)-\pi_I(\x)} &= \norm{(y_{i_1}-x_{i_1},\dots,y_{i_k}-x_{i_k})}\tag*{Definition \ref{dfn:18.2}}\\
            &= \sqrt{(y_{i_1}-x_{i_1})^2+\cdots+(y_{i_k}-x_{i_k})^2}\tag*{Definition \ref{dfn:18.6}}\\
            &\leq \sqrt{(y_1-x_1)^2+\cdots+(y_m-x_m)^2}\\
            &= \norm{(y_1-x_1,\dots,y_m-x_m)}\tag*{Definition \ref{dfn:18.6}}\\
            &= \norm{\y-\x}\tag*{Definition \ref{dfn:18.2}}\\
            &< \epsilon
        \end{align*}
        as desired.\footnote{This can also be done, without much difficulty, with the open preimage form of continuity.}
    \end{proof}
\end{exercise}

\begin{remark}\label{rmk:18.36}
    Let $A\subset\R^n$ be a rectangle (open or closed). Then
    \begin{equation*}
        A = \pi_1(A)\times\cdots\times\pi_n(A)
    \end{equation*}
\end{remark}

\begin{definition}\label{dfn:18.37}
    Let $f:A\to\R^m$. Its $i^\text{th}$ component function $f_i:A\to\R$ is defined as
    \begin{equation*}
        f_i = \pi_i\circ f
    \end{equation*}
    In other words,
    \begin{equation*}
        f(\x) = (f_1(\x),\dots,f_m(\x))
    \end{equation*}
\end{definition}

\begin{theorem}\label{trm:18.38}
    Let $A\subset\R^n$ and let $\x$ be a limit point of $A$. Suppose $f:A\to\R^m$. If $\lim_{\y\to\x}f(\y)=\z$ exists (with $\z=(z_1,\dots,z_m)$), then for all $i\in[m]$, $\lim_{\y\to\x}f_i(\y)$ exists and equals $z_i$. Conversely, if for all $i\in[m]$, $\lim_{\y\to\x}f_i(\y)=z_i$, then $\lim_{\y\to\x}f(\y)$ exists and equals $\z=(z_1,\dots,z_m)$.
    \begin{proof}
        Suppose first that $\lim_{\y\to\x}f(\y)=\z$. Let $i$ be an arbitrary element of $[m]$. To prove that $\lim_{\y\to\x}f_i(\y)=z_i$, Definition \ref{dfn:18.29} tells us that it will suffice to show that for every $\epsilon>0$, there exists $\delta>0$ such that if $\y\in A$ and $0<\norm{\y-\x}<\delta$, then $\norm{f_i(\y)-z_i}<\epsilon$. Let $\epsilon>0$ be arbitrary. Since $\lim_{\y\to\x}f(\y)=\z$, Definition \ref{dfn:18.29} asserts that there exists $\delta>0$ such that if $\y\in A$ and $0<\norm{\y-\x}<\delta$, then $\norm{f(\y)-\z}<\epsilon$. Choose this $\delta$ to be our $\delta$. Let $\y$ be an arbitrary element of $A$ satisfying $0<\norm{\y-\x}<\delta$. Then
        \begin{align*}
            \norm{f_i(\y)-z_i} &= \sqrt{(f_i(\y)-z_i)^2}\tag*{Definition \ref{dfn:18.6}}\\
            &\leq \sqrt{(f_1(\y)-z_1)^2+\cdots+(f_m(\y)-z_m)^2}\\
            &= \norm{(f_1(\y),\dots,f_m(\y))-(z_1,\dots,z_m)}\tag*{Definition \ref{dfn:18.6}}\\
            &= \norm{f(\y)-\z}\tag*{Definition \ref{dfn:18.37}}\\
            &< \epsilon
        \end{align*}
        as desired.\par
        Now suppose that for all $i\in[m]$, $\lim_{\y\to\x}f_i(\y)=z_i$. To prove that $\lim_{\y\to\x}f(\y)=\z$, Definition \ref{dfn:18.29} tells us that it will suffice to show that for all $\epsilon>0$, there exists $\delta>0$ such that if $\y\in A$ and $0<\norm{\y-\x}<\delta$, then $\norm{f(\y)-\z}<\epsilon$. Let $\epsilon>0$ be arbitrary. Since $\lim_{\y\to\x}f_i(\y)=z_i$ for all $i\in[m]$, Definition \ref{dfn:18.29} asserts that for all $i\in[m]$, there exists $\delta_i>0$ such that if $\y\in A$ and $0<\norm{\y-\x}<\delta$, then $\norm{f_i(\y)-z_i}<\frac{\epsilon}{m}$. Choose $\delta=\min(\delta_1,\dots,\delta_m)$. Let $\y$ be an arbitrary element of $A$ satisfying $0<\norm{\y-\x}<\delta$. Then
        \begin{align*}
            \norm{f(\y)-\z} &= \norm{(f_1(\y)-z_1)+\cdots+(f_m(\y)-z_m)}\\
            &\leq \norm{f_1(\y)-z_1}+\cdots+\norm{f_m(\y)-z_m}\tag*{Theorem \ref{trm:18.10}}\\
            &< \underbrace{\frac{\epsilon}{m}+\cdots+\frac{\epsilon}{m}}_{m\text{ times}}\\
            &= \epsilon
        \end{align*}
        as desired.
    \end{proof}
\end{theorem}

\begin{corollary}\label{cly:18.39}
    Let $A\subset\R^n$. A function $f:A\to\R^m$ is continuous if and only if $f_1,\dots,f_m$ are all continuous.
    \begin{proof}
        Suppose first that $f$ is continuous. Let $\x$ be an arbitrary element of $A$, and let $i$ be an arbitrary element of $[m]$. To prove that $f_i$ is continuous at $\x$, Theorem \ref{trm:18.31} tells us that it will suffice to show that either $\x\notin LP(A)$ or $\lim_{\y\to\x}f_i(\y)=f_i(\x)$. Since $f$ is continuous at $\x$ by hypothesis, Theorem \ref{trm:18.31} asserts that either $\x\notin LP(A)$ or $\lim_{\y\to\x}f(\y)=f(\x)$. We now divide into two cases. If $\x\notin LP(A)$, then we are done. On the other hand, if $\lim_{\y\to\x}f(\y)=f(\x)$, then by Theorem \ref{trm:18.38} and Definition \ref{dfn:18.37}, $\lim_{\y\to\x}f_i(\y)=f_i(\x)$, as desired.\par
        The proof is symmetric in the other direction.
    \end{proof}
\end{corollary}

\marginnote{7/24:}Now we revisit compactness, but in $\R^n$. For our purposes, the key result is Corollary \ref{cly:18.48}.

\begin{definition}\label{dfn:18.40}
    Let $A\subset\R^n$. Then $A$ is \textbf{compact} if every open cover $\mathcal{G}$ of $A$ has a finite subcover.
\end{definition}

\begin{proposition}\label{prp:18.41}
    Let $A\subset\R^n$. Then $A$ is compact if and only if every open cover $\mathcal{G}$ of $A$ consisting solely of open rectangles has a finite subcover.
    \begin{proof}
        Suppose first that $A$ is compact. Let $\mathcal{G}$ be an arbitrary open cover of $A$ consisting solely of open rectangles. Then since $A$ is compact, by Definition \ref{dfn:18.40}, $\mathcal{G}$ has a finite subcover.\par
        Now suppose that every open cover $\mathcal{G}$ of $A$ consisting solely of open rectangles has a finite subcover. To prove that $A$ is compact, Definition \ref{dfn:18.40} tells us that it will suffice to show that every open cover $\mathcal{G}$ of $A$ has a finite subcover. Let $\mathcal{G}=\{G_\lambda\mid\lambda\in\Lambda\}$ be an arbitrary open cover of $A$, and let $G_\lambda$ be an arbitrary element of $\mathcal{G}$. By Definition \ref{dfn:10.3}, $G_\lambda$ is open. Thus, by Proposition \ref{prp:18.22}, $G_\lambda=\bigcup_{\gamma\in\Gamma_\lambda}R_{\lambda_\gamma}$, where each $R_{\lambda_\gamma}$ is an open rectangle. Now let $\mathcal{H}=\{R_{\lambda_\gamma}\mid\lambda\in\Lambda,\gamma\in\Gamma_\lambda\}$. It follows by Script \ref{sct:1} that $\mathcal{G}=\mathcal{H}$. Additionally, by the hypothesis, there exists a finite subcover $\mathcal{H}'\subset\mathcal{H}$ of $A$. Finally, if $R_{\lambda_\gamma}\in\mathcal{H}'$, let $G_\lambda\in\mathcal{G}'$. It follows that $\mathcal{G}'$ is a finite subcover of $\mathcal{G}$, as desired.
    \end{proof}
\end{proposition}

\begin{definition}\label{dfn:18.42}
    Let $A\subset\R^n$ and $f:A\to\R^m$. We say that $f$ is \textbf{uniformly continuous} if for every $\epsilon>0$, there exists $\delta>0$ such that if $\x,\y\in A$ and $\norm{\x-\y}<\delta$, then $\norm{f(\x)-f(\y)}<\epsilon$.
\end{definition}

\begin{theorem}\label{trm:18.43}
    Let $A\subset\R^n$ be compact and $f:A\to\R^m$ be continuous. Then $f$ is uniformly continuous.
    \begin{proof}
        The proof is symmetric to that of Theorem \ref{trm:13.6}.
    \end{proof}
\end{theorem}

\begin{theorem}\label{trm:18.44}
    If $A\subset\R^n$ is compact and $f:A\to\R^m$ is continuous, then $f(A)$ is compact.
    \begin{proof}
        The proof is symmetric to that of Theorem \ref{trm:10.19}.
    \end{proof}
\end{theorem}

\begin{corollary}\label{cly:18.45}
    Let $\x\in\R^n$. If $B$ is a compact subset of $\R^m$, then $\{\x\}\times B$ is a compact subset of $\R^{n+m}$.
    \begin{proof}
        Let $f:B\to\R^{n+m}$ be defined by $f(\y)=(x_1,\dots,x_n,y_1,\dots,y_m)$ for all $\y\in B$. Let $\y$ be an arbitrary element of $B$. To prove that $f$ is continuous at $\y$, Theorem \ref{trm:18.31} tells us that it will suffice to show that for every $\epsilon>0$, there exists $\delta$ such that if $\z\in B$ and $\norm{\z-\y}<\delta$, then $\norm{f(\z)-f(\y)}<\epsilon$. Let $\epsilon>0$ be arbitrary. Choose $\delta=\epsilon$. Let $\z$ be an arbitrary element of $B$ satisfying $\norm{\z-\y}<\delta$. Then
        \begin{align*}
            \norm{f(\z)-f(\y)} &= \norm{(x_1-x_1,\dots,x_n-x_n,z_1-y_1,\dots,z_m-y_m)}\\
            &= \norm{(z_1-y_1,\dots,z_m-y_m)}\\
            &= \norm{\z-\y}\\
            &< \epsilon
        \end{align*}
        as desired. Therefore, since $B\subset\R^m$ is compact and $f:B\to\R^{n+m}$ is continuous, Theorem \ref{trm:18.44} asserts that $f(B)$ is compact. Naturally, $f(B)=\{\x\}\times B$, so the latter set is compact, too, as desired.
    \end{proof}
\end{corollary}

\begin{lemma}\label{lem:18.46}
    Let $\x\in\R^n$ and $B\subset\R^m$. If $\mathcal{G}$ is a finite set of open rectangles that covers $\{\x\}\times B\subset\R^{n+m}$, then there exists an open rectangle $R\subset\R^n$ containing $\x$ such that $\mathcal{G}$ covers $R\times B$.
    \begin{proof}
        % \begin{itemize}
        %     \item Let $\mathcal{G}=\{R_i\mid i\in[k]\}$.
        %     \item Every $\pi_{[n]}(R_i)$ is an open rectangle.
        %     \begin{itemize}
        %         \item Let $i\in[k]$ be arbitrary.
        %         \item Let $R_i=(r_{i_j},s_{i_j})_{j=1}^{n+m}$.
        %         \item WTS (Definition \ref{dfn:18.13}): $\pi_{[n]}(R_i)=(r_{i_j},s_{i_j})_{j=1}^n$.
        %         \item Let $\y\in\pi_{[n]}(R_i)$ be arbitrary.
        %         \item Definition \ref{dfn:1.18}: $\y=\pi_{[n]}(\z)$ for some $\z\in R_i$.
        %         \item Definition \ref{dfn:18.34}: $y_j=z_j$ for all $j\in[n]$.
        %         \item Definition \ref{dfn:18.13}: $y_j\in(r_{i_j},s_{i_j})$ for all $j\in[n]$.
        %         \item Definition \ref{dfn:1.15}: $\y\in(r_{i_j},s_{i_j})_{j=1}^n$.
        %         \item Symmetric in the other direction.
        %     \end{itemize}
        %     \item Every $\pi_{[n]}(R_i)$ contains $\x$.
        %     \begin{itemize}
        %         \item Let $i\in[k]$ be arbitrary.
        %         \item Let $\y\in\{\x\}\times B$ such that $\y\in R_i$.
        %         \item Definition \ref{dfn:1.18}: $\pi_{[n]}(\y)\in\pi_{[n]}(R_i)$.
        %         \item Definition \ref{dfn:1.15}: $\y=(x_1,\dots,x_n,y_1,\dots,y_m)$.
        %         \item Definition \ref{dfn:18.34}: $\pi_{[n]}(\y)=\x$.
        %         \item Therefore: $\x\in\pi_{[n]}(R_i)$.
        %     \end{itemize}
        %     \item Let $R=\bigcap_{i\in[k]}\pi_{[n]}(R_i)$.
        %     \item Exercise \ref{exr:18.16}: $R$ is an open rectangle containing $\x$.
        %     \item $\mathcal{G}$ covers $R\times B$.
        %     \begin{itemize}
        %         \item WTS (Definition \ref{dfn:10.3}): For all $\y\in R\times B$, $\y\in R_i$ for some $R_i\in\mathcal{G}$.
        %         \item Let $\y=(y_1,\dots,y_{n+m})\in R\times B$ be arbitrary.
        %         \item Definition \ref{dfn:1.15}: $(y_1,\dots,y_n)\in R$ and $(y_{n+1},\dots,y_{n+m})\in B$.
        %         \item Definition \ref{dfn:10.3} ($\mathcal{G}$ is a cover of $\{\x\}\times B$): $(x_1,\dots,x_n,y_{n+1},\dots,y_{n+m})\in R_i$ for some $i\in[k]$.
        %         \item Consider this $R_i$; we will show that $\y\in R_i$.
        %         \item WTS (Definition \ref{dfn:18.13}): $y_i\in(r_{i_j},s_{i_j})$ for all $j\in[n+m]$.
        %         \item Case 1: $j\in[n]$.
        %         \begin{itemize}
        %             \item Theorem \ref{trm:1.7} ($R=\bigcap_{i\in[k]}\pi_{[n]}(R_i)$): $R\subset\pi_{[n]}(R_i)$.
        %             \item Definition \ref{dfn:1.3} ($(y_1,\dots,y_n)\in R$): $(y_1,\dots,y_n)\in\pi_{[n]}(R_i)$.
        %             \item The above: $\pi_{[n]}(R_i)=(r_{i_j},s_{i_j})_{j=1}^n$.
        %             \item Definition \ref{dfn:1.2}: $(y_1,\dots,y_n)\in(r_{i_j},s_{i_j})_{j=1}^n$.
        %             \item Definition \ref{dfn:18.13}: $y_i\in(r_{i_j},s_{i_j})$ for all $j\in[n]$.
        %         \end{itemize}
        %         \item Case 2: $j\in[n+1:m]$.
        %         \begin{itemize}
        %             \item The above: $(x_1,\dots,x_n,y_{n+1},\dots,y_{n+m})\in R_i$.
        %             \item Definition \ref{dfn:18.13}: $y_i\in(r_{i_j},s_{i_j})$ for all $j\in[n+1:m]$.
        %         \end{itemize}
        %     \end{itemize}
        % \end{itemize}

        Let $\mathcal{G}=\{R_i\mid i\in[k]\}$. To begin, we will show that every $\pi_{[n]}(R_i)$ is an open rectangle containing $\x$. It will follow that the intersection of all $\pi_{[n]}(R_i)$ is an open rectangle $R$ containing $\x$. Thus, since this $R$ is a subset of each $R_i$ in dimensions 1 through $n$, we will be able to show that $\mathcal{G}$ covers $R\times B$. Let's begin.\par\smallskip
        First, we will show that every $\pi_{[n]}(R_i)$ is an open rectangle. Let $i$ be an arbitrary element of $[k]$, and let $R_i=(r_{i_j},s_{i_j})_{j=1}^{n+m}$. To show that $\pi_{[n]}(R_i)$ is an open rectangle, Definition \ref{dfn:18.13} tells us that it will suffice to verify that $\pi_{[n]}(R_i)=(r_{i_j},s_{i_j})_{j=1}^n$. Let $\y$ be an arbitrary element of $\pi_{[n]}(R_i)$. By Definition \ref{dfn:1.18}, $\y=\pi_{[n]}(\z)$ for some $\z\in R_i$. Thus, by Definition \ref{dfn:18.34}, $y_j=z_j$ for all $j\in[n]$. Consequently, since $z_j\in(r_{i_j},s_{i_j})$ for all $j\in[n]$ by Definition \ref{dfn:18.13}, we have that $y_j\in(r_{i_j},s_{i_j})$ for all $j\in[n]$. Therefore, by Definition \ref{dfn:1.15}, $\y\in(r_{i_j},s_{i_j})_{j=1}^n$. The argument is symmetric in the other direction. Both arguments, when combined, imply by Definition \ref{dfn:1.2} that $\pi_{[n]}(R_i)=(r_{i_j},s_{i_j})_{j=1}^n$, as desired.\par
        Next, we will show that every $\pi_{[n]}(R_i)$ contains $\x$. Let $i$ be an arbitrary element of $[k]$, and let $\y$ be an arbitrary element of $\{\x\}\times B$ satisfying $\y\in R_i$ (Definition \ref{dfn:10.3} guarantees that $\y$ is in some $R_i$). Thus, by Definition \ref{dfn:1.18}, $\pi_{[n]}(\y)\in\pi_{[n]}(R_i)$. Additionally, by Definition \ref{dfn:1.15}, $\y=(x_1,\dots,x_n,y_1,\dots,y_m)$. It follows by Definition \ref{dfn:18.34} that $\pi_{[n]}(\y)=\x$. Therefore, $\x\in\pi_{[n]}(R_i)$, as desired.\par
        Let $R=\bigcap_{i\in[k]}\pi_{[n]}(R_i)$. Consequently, by Exercise \ref{exr:18.16}, $R$ is an open rectangle containing $\x$.\par
        To prove that $\mathcal{G}$ covers $R\times B$, Definition \ref{dfn:10.3} tells us that it will suffice to show that for all $\y\in R\times B$, $\y\in R_i$ for some $R_i\in\mathcal{G}$. Let $\y=(y_1,\dots,y_{n+m})$ be an arbitrary element of $R\times B$. By Definition \ref{dfn:1.15}, $(y_1,\dots,y_n)\in R$ and $(y_{n+1},\dots,y_{n+m})\in B$. It follows from the latter statement and the fact that $\mathcal{G}$ is a cover of $\{\x\}\times B$ that $(x_1,\dots,x_n,y_{n+1},\dots,y_{n+m})\in R_i$ for some $i\in[k]$. Consider this $R_i$; we will confirm that $\y$ is an element of it. To do so, Definition \ref{dfn:18.13} tells us that it will suffice to demonstrate that $y_j\in(r_{i_j},s_{i_j})$ for all $j\in[n+m]$. We divide into two cases ($j\in[n]$ and $j\in[n+1:m]$). Suppose first that $j\in[n]$. Then since $R=\bigcap_{i\in[k]}\pi_{[n]}(R_i)$, Theorem \ref{trm:1.7} asserts that $R\subset\pi_{[n]}(R_i)$. Thus, since $(y_1,\dots,y_n)\in R$ by the above, Definition \ref{dfn:1.3} implies that $(y_1,\dots,y_n)\in\pi_{[n]}(R_i)$. Additionally, by the above, $\pi_{[n]}(R_i)$ can be written in the form $(r_{i_j},s_{i_j})_{j=1}^n$. Combining the last two results, we have by Definition \ref{dfn:1.2} that $(y_1,\dots,y_n)\in(r_{i_j},s_{i_j})_{j=1}^n$. Therefore, by Definition \ref{dfn:18.13}, $y_j\in(r_{i_j},s_{i_j})$ for all $j\in[n]$, as desired. Now suppose that $j\in[n+1:m]$. By the above, $(x_1,\dots,x_n,y_{n+1},\dots,y_{n+m})\in R_i$. Therefore, by Definition \ref{dfn:18.13}, $y_j\in(r_{i_j},s_{i_j})$ for all $j\in[n+1:m]$, as desired.
    \end{proof}
\end{lemma}

\begin{theorem}\label{trm:18.47}
    If $A\subset\R^n$ and $B\subset R^m$ are compact, then $A\times B\subset\R^{n+m}$ is also compact.
    \begin{proof}
        % \begin{itemize}
        %     \item WTS (Proposition \ref{prp:18.41}): Every open cover $\mathcal{G}$ of $A\times B$ consisting solely of open rectangles has a finite subcover.
        %     \item Let $\mathcal{G}$ be an arbitrary open cover of $A\times B$ consisting solely of open rectangles.
        %     \item Corollary \ref{cly:18.45}: For all $\x\in A$, $\{\x\}\times B$ is compact.
        %     \item Definition \ref{dfn:18.40}: For all $\x\in A$, there is a finite subcover $\mathcal{G}_\x\subset\mathcal{G}$ that covers $\{\x\}\times B$.
        %     \item Lemma \ref{lem:18.46}: For each $\mathcal{G}_\x$, there is an open rectangle $R_\x$ containing $\x$ such that $\mathcal{G}_\x$ covers $R_\x\times B$.
        %     \item Definition \ref{dfn:10.4}: The set $\{R_\x\mid\x\in A\}$ is an open cover of $A$.
        %     \item Definition \ref{dfn:18.40} ($A$ is compact): There exists a finite subcover $\{R_\x\mid \x\in I\}\subset\{\R_\x\mid\x\in A\}$ of $A$, where $I\subset A$.
        %     \item Let $\mathcal{G}'=\bigcup_{\x\in I}\mathcal{G}_\x$.
        %     \item Clearly, $\mathcal{G}'$ is a finite subset of $\mathcal{G}$.
        %     \item $\mathcal{G}'$ is an open cover of $A\times B$.
        %     \begin{itemize}
        %         \item WTS (Definition \ref{dfn:10.3}): Every $\y\in A\times B$ is an element of $G$ for some $G\in\mathcal{G}'$.
        %         \item Let $\y\in A\times B$ be arbitrary.
        %         \item Definition \ref{dfn:1.15}: $\y=(a_1,\dots,a_n,b_1,\dots,b_m)$, where $(a_1,\dots,a_n)\in A$ and $(b_1,\dots,b_m)\in B$.
        %         \item The above ($(a_1,\dots,a_n)\in A$): $(a_1,\dots,a_n)\in R_\x$ for some $\x\in I$.
        %         \item Definition \ref{dfn:1.15} ($(a_1,\dots,a_n)\in R_\x$ and $(b_1,\dots,b_m)\in B$): $\y\in R_\x\times B$.
        %         \item Definition \ref{dfn:10.3} ($R_\x\times B$ is covered by $\mathcal{G}_\x$): There exists $G\in\mathcal{G}_\x$ such that $\y\in G$.
        %         \item Theorem \ref{trm:1.7}: $\mathcal{G}_\x\subset\mathcal{G}'$.
        %         \item Definition \ref{dfn:1.3}: $G\in\mathcal{G}'$.
        %         \item Therefore: $\y\in G$ for some $G\in\mathcal{G}'$, as desired.
        %     \end{itemize}
        % \end{itemize}

        To prove that $A\times B$ is compact, Proposition \ref{prp:18.41} tells us that it will suffice to show that every open cover $\mathcal{G}$ of $A\times B$ consisting solely of open rectangles has a finite subcover. Let $\mathcal{G}$ be an arbitrary open cover of $A\times B$ consisting solely of open rectangles. By Corollary \ref{cly:18.45}, for all $\x\in A$, $\{\x\}\times B$ is compact. Thus, by Definition \ref{dfn:18.40}, for all $\x\in A$, there is a finite subcover $\mathcal{G}_\x\subset\mathcal{G}$ that covers $\{\x\}\times B$. Since $\mathcal{G}_\x$ is a finite set of open rectangles that covers $\{\x\}\times B$, it follows by Lemma \ref{lem:18.46} that for each $\mathcal{G}_\x$, there is an open rectangle $R_\x$ containing $\x$ such that $\mathcal{G}_\x$ covers $R_\x\times B$. Additionally, since each $R_\x$ is open and $\x\in R_\x$ for all $\x\in A$, Definition \ref{dfn:10.3} asserts that $\{\R_\x\mid\x\in A\}$ is an open cover of $A$. But since $A$ is compact, there exists a finite subcover $\{R_\x\mid\x\in I\}\subset\{R_\x\mid\x\in A\}$ of $A$, where $I\subset A$. We are now ready to define our finite subcover $\mathcal{G}'\subset\mathcal{G}$ of $A\times B$, and verify that it is such.\par
        Let $\mathcal{G}'=\bigcup_{\x\in I}\mathcal{G}_\x$. Since $\mathcal{G}'$ is the union of finitely many finite subsets of $\mathcal{G}$, Script \ref{sct:1} guarantees that $\mathcal{G}'$ is, itself, a finite subset of $\mathcal{G}$. To confirm that $\mathcal{G}'$ is an open cover of $A\times B$, Definition \ref{dfn:10.3} tells us that it will suffice to show that every $\y\in A\times B$ is an element of $G$ for some $G\in\mathcal{G}'$. Let $\y$ be an arbitrary element of $A\times B$. By Definition \ref{dfn:1.15}, $\y=(a_1,\dots,a_n,b_1,\dots,b_m)$, where $(a_1,\dots,a_n)\in A$ and $(b_1,\dots,b_m)\in B$. It follows from the former statement and the definition of $\{R_\x\mid\x\in I\}$ that $(a_1,\dots,a_n)\in R_\x$ for some $\x\in I$. This combined with the latter statement implies by Definition \ref{dfn:1.15} that $\y\in R_\x\times B$. Thus, since $\mathcal{G}_\x$ covers $R_\x\times B$, there exists $G\in\mathcal{G}_\x$ such that $y\in G$. Additionally, Theorem \ref{trm:1.7} implies that $\mathcal{G}_\x\subset\mathcal{G}$, so we have by Definition \ref{dfn:1.3} that $G\in\mathcal{G}'$. Therefore, $\y\in G$ for some $G\in\mathcal{G}'$, as desired.
    \end{proof}
\end{theorem}

\begin{corollary}\label{cly:18.48}
    If $A_1,\dots,A_n$ are all compact, then so is $A_1\times\cdots\times A_n$. In particular, a closed rectangle is compact.
    \begin{proof}
        We induct on $n$. For the base case $n=1$, if $A_1$ is compact, then $\prod_{i=1}^1A_i=A_i$ is trivially compact. Now suppose inductively that we have proven the claim for $n$; we now seek to prove it for $n+1$. Let $A_1,\dots,A_{n+1}$ be compact. By hypothesis, $\prod_{i=1}^nA_i$ is compact. Thus, by Theorem \ref{trm:18.47}, $\prod_{i=1}^{n+1}A_i=(\prod_{i=1}^nA_i)\times A_{n+1}$ is compact, as desired.\par
        Let $R$ be an arbitrary closed rectangle. By Definition \ref{dfn:18.13}, $R=[a_i,b_i]_{i=1}^n$. Additionally, by Theorem \ref{trm:10.14}, every $[a_i,b_i]$ is compact. Thus, since $R$ is the Cartesian product of $n$ compact sets, we have by the above that $R$ is compact, as desired.
    \end{proof}
\end{corollary}

\begin{theorem}\label{trm:18.49}
    If $A\subset X\subset\R^n$ with $X$ compact and $A$ closed in $\R^n$, then $A$ is compact.
    \begin{proof}
        The proof is symmetric to that of Theorem \ref{trm:10.15}.
    \end{proof}
\end{theorem}

\begin{theorem}\label{trm:18.50}
    Closed balls are compact.
    \begin{proof}
        Let $\overline{B}(\x,r)$ be an arbitrary closed ball. By Definition \ref{dfn:18.13}, $R=\prod_{i=1}^n[x_i-r,x_i+r]$ is a closed rectangle. Thus, to prove that $\overline{B}(\x,r)$ is compact, Theorem \ref{trm:18.49} tells us that it will suffice to show that $\overline{B}\subset R$, that $R$ is compact, and that $\overline{B}$ is closed. Let's begin.\par\smallskip
        To prove that $\overline{B}\subset R$, Definition \ref{dfn:1.3} tells us that it will suffice to show that every $\y\in\overline{B}$ is an element of $R$. Let $\y$ be an arbitrary element of $\overline{B}$. Then by Definition \ref{dfn:18.17}, $\norm{\y-\x}\leq r$. It follows by the lemma to Exercise \ref{exr:18.30} that $|y_i-x_i|\leq r$ for all $1\leq i\leq n$. Thus, by Exercise \ref{exr:8.9}, $y_i\in[x_i-r,x_i+r]$ for all $1\leq i\leq n$. Consequently, by Definition \ref{dfn:18.13}, $\y\in R$, as desired.\par
        By Corollary \ref{cly:18.48}, $R$ is compact, as desired.\par
        By Corollary \ref{cly:18.21}, $\overline{B}$ is closed, as desired.
    \end{proof}
\end{theorem}

\begin{definition}\label{dfn:18.51}
    A subset $A$ of $\R^n$ is bounded if there exists a closed rectangle $R$ such that $A\subset R$.
\end{definition}

\begin{theorem}[The Heine-Borel theorem in $\R^n$]\label{trm:18.52}
    A subset of $\R^n$ is compact if and only if it is closed and bounded.
    \begin{proof}
        Suppose first that $X$ is a compact subset of $\R^n$.\par
        Now suppose that $X$ is a closed and bounded subset of $\R^n$. Since $X$ is bounded, Definition \ref{dfn:18.51} implies that there exists a closed rectangle $R$ such that $X\subset R$. Additionally, since $R$ is a closed rectangle, Corollary \ref{cly:18.48} implies that $R$ is compact. Thus, since $X\subset R$ with $R$ compact and $X$ closed, Theorem \ref{trm:18.49} asserts that $X$ is compact.
    \end{proof}
\end{theorem}




\end{document}